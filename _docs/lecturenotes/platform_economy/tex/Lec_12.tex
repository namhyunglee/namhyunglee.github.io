\chapter{플랫폼과 반독점: 이론}\label{cha:antitrust}

\section*{학습개요}
현대 경쟁 정책의 기본 특징과 디지털 플랫폼 시장에서의 독점 문제를 학습한다.

\section*{학습목표}
\begin{enumerate}
\item 현대 경쟁 정책의 기본 특징을 이해한다.
\item 경쟁 제한 형태의 유형을 설명할 수 있다.
\item 디지털 플랫폼 시장의 독점화 경향을 설명할 수 있다.
\item 디지털 플랫폼 산업의 진입 장벽을 설명할 수 있다.
\item 디지털 플랫폼으로 인한 사회적 손실을 설명할 수 있다.
\item 디지털 플랫폼에 대한 규제가 필요하지 않다는 입장을 설명할 수 있다.
\end{enumerate}

\section*{주요 용어}
약탈 가격, 초기 위협의 배제, 수익 체증, 정보 비대칭, 소비자 잉여, 경쟁 약화, 혁신 지체


\pagebreak

%\section{양면 시장과 경쟁 정책}
\section{경쟁 정책의 기초}
\begin{itemize}
\item 현대 경쟁 정책의 기본 특징
	\begin{itemize}
	\item 기업의 전략적 의사 결정과 시장 구조의 동태적 변화를 확인 $\rightarrow$ 사안 별 분석
	\item 현재의 시장 구조보다 잠재적 시장 진입 가능성이 더 중요 $\rightarrow$ 정부 개입의 최소화
	\item 소비자 잉여의 감소 $\rightarrow$ 정부 개입의 근거
	\end{itemize}
\item 독점 시장의 가격과 생산량
	\begin{itemize}
	\item 완전경쟁시장과 비교해 높은 가격, 적은 생산량
	\item $\rightarrow$ 독점 시장이라고 하더라도 완전 경쟁 시장에서의 자원 배분 효율성을 달성할 수 있음
		\begin{itemize}
		\item[예)] 1차 가격 차별: 모든 소비자 각각의 지불 의사에 맞춰 차별화된 가격을 부과 
			\begin{itemize}
			\item $\rightarrow$ 단, 모든 소비자 잉여가 생산자 잉여로 흡수 
			\item $\rightarrow$ 소비자 잉여의 감소 문제는 남음
			\end{itemize}
		\item[예)] 잠재적 경쟁자: 항상 새로운 기업이 시장 진입을 할 수 있다면, 기존 독점 기업은 높은 독점 가격을 유지할 수 없음
			\begin{itemize}
			\item $\rightarrow$ 가격을 한계 비용과 같은 낮은 수준으로 유지할 수 있음
			\end{itemize}
		\end{itemize}
	\end{itemize}
\item 즉, 독점 생산자인 것이 문제가 아니라 경쟁을 제한하는 행태가 문제 \citep{Shapiro:2019aa}
	\begin{itemize}
	\item 제한 가격
		\begin{itemize}
		\item 기존 기업이 신규 기업의 시장 진입을 유도하지 않으면서 부과할 수 있는 가장 높은 가격
		\item 가정: 기존 기업의 생산 비용 $<$ 신규 기업의 생산 비용
			\begin{itemize}
			\item 기존 기업은 신규 기업에 비해 생산기술, 자원 획득, 광고, 정부 허가, 판매망, 신기술 개발 등에서 이점을 누릴 수 있기 때문
			\end{itemize}
		\item 신규 기업은 기존 기업보다 낮은 가격으로 시장 진입을 시도할 것
			\begin{itemize}
			\item  $\rightarrow$ 기존 기업은 신규 기업과 같은 수준으로 가격을 낮추어 대응
			\item $\rightarrow$ 신규 기업의 수요 감소 
			\item $\rightarrow$ 신규 기업이 시장 진입 포기
			\end{itemize}
		\end{itemize}
	\item 약탈 가격 (predatory pricing)
		\begin{itemize}
		\item 기존 기업이 일시적 이윤 감소를 감수하고 후발 기업의 시장 진입을 막은 후 다시 가격을 높이는 전략
		\item 현실적이지 않다는 반론이 있음
			\begin{itemize}
			\item 약탈 가격 전략 구사 후, 기존 기업이 감소한 이윤을 회복할 수 있을까?
			\item 기존 기업이 약탈 가격을 충분히 오랫동안 구사할 수 있는가?
			\item 약탈 가격 전략을 철회한 후 다른 신규 기업이 시장에 진입하지 않는다거나, 같은 신규 기업이 다시 시장에 진입하지 않는다는 보장이 있을까?
			\item 두 기업의 비용 구조가 유사하거나 동일하다면, 신규 기업도 약탈가격 전략을 구사할 수 있지 않을까?
			\end{itemize}
		\end{itemize}
	\item 초기 위협의 배제 (exclusion of nascent threats)
		\begin{itemize}
		\item 시장의 우월적인 지위를 이용하여, 핵심 사업을 위협할 정도로 성장할 것으로 예상되는 새로운 재화나 서비스를 배제시키는 전략
		\item[예)] 대상 기업의 인수합병, 대상 재화나 서비스의 판매 방해 등 
		\end{itemize}	
	\item 우월적 지위를 남용한 인접 시장 확장	
		\begin{itemize}
		\item 인접 시장에서 자사 상품이나 서비스를 타사 상품이나 서비스에 비해 우선 판매하는 경우
		\end{itemize}
	\end{itemize}
\item 독점의 측정
	\begin{itemize}
	\item 러너 지수 (Lerner Index) $= \dfrac{p-mc}{p}$
		\begin{itemize}
		\item 즉, 기업이 정한 가격($p$)에서 한계 비용($mc$)을 뺀 것을 가격에 대한 비율로 표시 
			\begin{itemize}
			\item $\rightarrow$ 0 $<$ 러너 지수 $<$ 1
			\end{itemize}
		\item 만약 완전경쟁시장이라면 $p = mc$ 
			\begin{itemize}
			\item $\rightarrow$ 러너 지수 $=$ 0
			\end{itemize}
		\item 독점 시장이라면 한계 비용보다 높은 가격을 정할 것이므로 $p > mc$ 
			\begin{itemize}
			\item $\rightarrow$ 러너 지수 $>$ 0
			\end{itemize}
		\item 만약 이윤 극대화를 추구하는 기업이라면, 러너 지수는 수요의 가격 탄력성의 음의 역수가 됨
			\begin{itemize}
			\item $\rightarrow$ 한계 비용을 관찰하지 않고, 가격 탄력성을 관찰하여 러너 지수를 측정할 수 있음
				\begin{align*}
				\pi & = p(q)q - c(q)q \\
				F.O.C.: \dfrac{d \pi}{d q} & = \left( \dfrac{d p}{d q} q + p \right) - \dfrac{d c}{d q}  = 0 \\
				p - mc & = - \dfrac{d p}{d q} q \quad \left( \text{정의 상 } mc = \dfrac{d c}{d q} \right) \\
				\dfrac{p - mc}{p} & = - \dfrac{d p}{d q} \dfrac{q}{p} \quad (\text{양변을 $p$로 나눠줌}) \\
				& = -1 / \left(  \dfrac{d q}{d p} \dfrac{p}{q} \right) \\
				& = - \dfrac{1}{\epsilon_{d}}
				\end{align*}
			\end{itemize}
		\end{itemize}
	\item 작지만 유의하며 비일시적인 가격 인상 (SSNIP: Small but significant and non-transitory increase in price) \citep{Johnson:1986ti}
		\begin{itemize}
		\item 1982년 미국 연방 거래 위원회(Federal Trade Commission) 제안
		\item 기업 가가 재화 A, 기업 나가 재화 B를 생산한다고 가정 
			\begin{itemize}
			\item 만약 재화 A와 재화 B가 대체재 관계에 있다면,
			\item 기업 가가 기업 나를 합병하여 가상의 독점 기업을 만든 후, 재화 A의 가격을 상승시키더라도 재화 B의 판매 증가로 수입을 늘릴 수 있음 $\rightarrow$ 기업 가는 시장을 독점할 유인이 있음
			\end{itemize}
		\end{itemize}
	\item 결정적 손실 공식 (Critical loss formulas) \citep{OBrien:2003us}
		\begin{itemize}
		\item 기업 결합 $\rightarrow$ 시장 지배력 확대 가능 $=$ 가격 상승 $\rightarrow$ 거래량 감소
			\begin{itemize}
			\item 상승한 가격은 가설적인 독점 가격 (hypothetical monopoly price)
			\item 기업 결합 이후 가격 상승이 있으리라는 또는 없으리라는 증거는 무엇인가?
			\end{itemize}
		\item 결정적 손실 분석
			\begin{itemize}
			\item 합병 이후 가격 상승으로 변화할 판매량으로 발생하는 이득($\Delta p (q+ \Delta q)$)과 손실 ( - $\Delta q (p-c)$)을 비교하면 됨
				\begin{align*}
				\Delta p (q+ \Delta q) & = - \Delta q (p-c) \\
				\dfrac{\Delta p}{q} \left( 1 + \dfrac{\Delta q} {q} \right) & = - \dfrac{\Delta q}{q} \left( \dfrac{p-c}{p} \right) \quad (\text{양변을 $pq$로 나눠줌}) \\
				- \dfrac{\Delta q}{q} & = \dfrac{\Delta p / p}{\Delta p / p+m}\text{, } \quad  m = \dfrac{p-c}{p}
				\end{align*}
			\end{itemize}	
		\item 결정적 손실 공식
			\begin{equation*}
			\text{결정적 손실} = - \dfrac{\Delta q}{q}  = \dfrac{X}{X+m}\text{, } \quad  X: X \text{퍼센트 가격 상승} = \dfrac{\Delta p}{p}
			\end{equation*}
		\end{itemize}
	\end{itemize}
\end{itemize}

\section{디지털 플랫폼의 독점화 경향}
\begin{itemize}
\item 디지털 플랫폼의 산업 구조는 독점 가능성이 높고, 시장 경쟁이 제한될 수 있음
\end{itemize}

\subsection{디지털 플랫폼의 산업 구조}
\begin{itemize}
\item 디지털 플랫폼은 다음의 이유로 하나의 기업이 시장을 지배 (승자 독식, Winner takes all markets)할 가능성이 높음 \citep{Zingales:2019aa, Furman:2019wl, OECD:2018wd}
	\begin{itemize}
	\item $\rightarrow$ 시장 내에서의 경쟁이 아니라, 시장 자체의 경쟁
	\end{itemize}	
\item 네트워크 효과
	\begin{itemize}
	\item 임계 질량과 네트워크 효과: 많은 수의 사용자에서 안정, \ref{cha:networktheory}장, 그림 \ref{fig:criticalmass} 참고
	\item $\rightarrow$ 충분한 사용자 수를 확보하기 위해 초기의 집중적인 투자가 필요
	\end{itemize}
\item 규모의 경제와 범위의 경제	
	\begin{itemize}	
	\item 규모의 경제(economies of scale): 높은 고정 비용과 낮은 한계 비용
		\begin{itemize}
		\item $\rightarrow$ 초기 투자로 많은 사용자 수를 확보하고, 좋은 상품 또는 서비스를 대량으로 공급
		\end{itemize}
	\item 범위의 경제 (economies of scope): 판매하는 상품이 다양하고, 서비스의 범위가 넓을 수록 비용 하락
		\begin{itemize}
		\item $\rightarrow$ 관련 시장으로 확장
		\item[예)] 지도 검색 $\rightarrow$ 식당 추천 
		\end{itemize}
	\end{itemize}
\item 데이터 사용에 대한 수익 체증
	\begin{itemize}
	\item 수익 체증(Increasing returns to scale):  $F(aK, aL) > a F(K, L)$ 즉, 늘어난 투입량보다 더 큰 비례로 생산량이 증가
	\item 데이터를 더 많이 확보하고 활용할 수록, 관련 상품이나 서비스는 더 좋아짐
	\end{itemize}
\item 전세계 시장으로의 낮은 운송 비용
	\begin{itemize}
	\item 여기서 운송은 온라인을 통한 디지털 전송 등을 포함
	\item 소비자가 인터넷 접속이 가능해야 한다는 점, 컨텐츠 사용 등의 라이센스 등 법적 사항, 언어 장벽 등의 문제가 있음
	\item 그럼에도 불구하고 기존 사업에 비해 전세계 시장 진출 비용은 낮다고 할 수 있음
	\end{itemize}
\end{itemize}

\subsection{디지털 플랫폼의 진입 장벽}
\begin{itemize}
\item 네트워크 효과와 규모의 경제
	\begin{itemize}
	\item 신규 기업이 기존 기업만큼의 사용자 수를 확보하는 데 시간과 비용이 필요
	\item 높은 고정 비용을 부담할 수 있는 기업만 진입 가능
	\item 데이터 구축의 부담
	\item $\rightarrow$ 시장에 먼저 진입한 기업의 작은 이득이 큰 장점으로 증폭될 수 있음
	\end{itemize}
\item 소비자 행동
	\begin{itemize}
	\item 소비자는 미래 가치보다 현재 가치를 과대 평가하는 경향이 있음 \citep{Thaler:2018aa}
		\begin{itemize}
		\item[예)] 첫 번째 검색 결과나 홈페이지 가장 위의 결과를 클릭
			\begin{itemize}
			\item 즉, 다른 검색 페이지나 홈페이지의 다른 내용을 보지 않음
			\item $\rightarrow$ 플랫폼 기업은 콘텐츠가 제시되는 화면을 통제
			\end{itemize}
		\item[예)] 소비자는 개인 정보 사용에 쉽게 동의
			\begin{itemize}
			\item 플랫폼 기업이 개인 정보의 사용 내역이나 사용 방법을 구체적으로 제시했다면 동의하지 않았을 수도 있음
			\item 또는 플랫폼 기업이 개인 정보 사용 동의를 기본으로 하고, 동의하지 않는 경우에만, 즉 소비자 입장에서는 추가적인 절차($=$ 비용)를 거쳐야만 사용하지 않도록 할 수도 있음
			\end{itemize} 
		\end{itemize}
	\item 하나의 플랫폼만 사용하는 소비자 (single-homing)
		\begin{itemize}
		\item[예)] 하나의 검색엔진만 사용, 하나의 소셜 네트워크 서비스만 사용 등
		\end{itemize}	
	\end{itemize}
\item 기존 플랫폼 기업이 만든 진입 장벽
	\begin{itemize}
	\item 정보 비대칭의 활용
		\begin{itemize}
		\item[예)] 맛집 검색: 기존 플랫폼이 신규 진입 기업의 콘텐츠와 비교하기 어렵게 할 수 있음
		\item $\rightarrow$ 기존 플랫폼은 소비자가 다른 서비스를 사용하는 전환 비용을 파악하기 어렵게 함
		\end{itemize}
	\item 신규 진입 기업이 가격으로 경쟁하기 어려울 수 있음
		\begin{itemize}
		\item[예)] 기존 플랫폼 기업이 제공하는 상품이나 서비스가 무료인 경우, 이 보다 낮은 가격으로 신규 기업이 시장에 진입할 수 없음
		\end{itemize}	
	\item 기술적으로는 선도 기업이 폐쇄성을 고집하거나, 기술 호환성을 제한하거나, 데이터 이동성을 낮출 유인이 있음
	\end{itemize}
\end{itemize}

\section{디지털 플랫폼과 사회적  손실}
\begin{itemize}
\item 독점과 일반적인 사회적 손실\footnote{거대 독점 디지털 플랫폼 기업과 관련된 사회 정치적 문제, 예를 들어, 소비자 사생활 보호, 데이터 보안, 혐오 발언, 가짜 뉴스 등도 분명 중요한 이슈이지만, 이 장과 다음 장에서는 경제적인 문제에 초점을 맞춘다.}
	\begin{itemize}
	\item 소비자 잉여 손실, 경쟁 약화, 혁신 지체
	\item 디지털 플랫폼에서도 나타날까?
	\end{itemize}
\item 추천 알고리듬
	\begin{itemize}
	\item 확증 편향 (confirmation bias): 기존의 믿음 또는 가설에 맞는 정보를 찾거나, 해석하거나, 선호하거나, 되새기는 경향
	\item $\rightarrow$ 플랫폼의 추천 알고리듬이 이윤 극대화를 목표로 설계되었을 때, 
	\item (알고리듬이 의도하지 않았더라도) 충동 구매, 가격 비교 중단, 구매 중독 등의 상태를 유지하도록 유도할 수도 있음 \citep{Allcott:2020wv,Allcott:2021vp}
	\item 구매 시점을 파악하여 추천 상품이나 서비스를 제공할 수도 있음
	\end{itemize}
\item 광고 알고리듬
	\begin{itemize}
	\item 플랫폼은 광고주가 목표로 하는 대상에 대해 광고를 할 것으로 계약을 체결 
		\begin{itemize}
		\item $\rightarrow$ 그러나 많은 경우, 이러한 광고를 집행하는 알고리듬은 그 투명성이 부족
		\item $\rightarrow$ 또한 광고 가격 결정 알고리듬이 공개되지 않는 경우가 많음
		\end{itemize}
	\item 다른 한편, 플랫폼은 사용자가 더 많은 광고에 노출되도록 플랫폼에 더 많은 시간 동안 머물러 있도록 할 유인이 있음
		\begin{itemize}
		\item $\rightarrow$ 경쟁 플랫폼 또는 서비스의 사용을 막기 위한 노력
		\end{itemize}
	\end{itemize}
\item 통상적인 소비자 잉여 측정이 어려움
	\begin{itemize}
	\item 많은 경우, 사용자는 무료로 서비스를 이용 $\leftrightarrow$ 하지만, 사실 상, 화면에 대한 집중 및 개인 정보와 교환
	\item $\rightarrow$ 무료이고 품질을 관찰하기 어려운 경우, 소비의 사회적 가치($=$ 소비자 잉여)를 측정하기 어려움
	\end{itemize}
\item 플랫폼, 특히 기술 플랫폼은 보완재의 이윤을 수취
	\begin{itemize}
	\item[예)] 애플과 구글은 운영체제를 제공하고 어플리케이션 스토어를 운영하면서, 판매 수입의 30\%를 수수료로 받음
	\item[예)] 구글은 검색 엔진을 제공하고 기사를 연결하면서 광고 수입을 얻지만 최근에 와서야 뉴스를 제공하는 기업과 콘텐츠 이용 계약을 체결
	\item 플랫폼 기업과 보완재를 판매하는 기업의 관계는 동태적
		\begin{itemize}
		\item 플랫폼 서비스의 초창기에는 상호 보완적: 플랫폼 기업은 사용자를 유인할 상품이나 서비스가 필요, 보완재 기업은 더 많은 소비자에게 접근 가능
		\item 소비자가 늘어남에 따라 보완재 기업은 투자를 늘릴 수 있음 $\rightarrow$ 하지만 투자로 인한 수입이 0이 된다면 투자를 하지 않을 것
		\item $\rightarrow$ 즉, 보완재 기업이 자신이 만드는 이윤의 대부분이 플랫폼 기업의 것이 된다고 판단한다면 더 이상의 투자는 발생하지 않게 됨
		\item $\rightarrow$ 하지만, 보완재 기업이 플랫폼 기업으로부터 독립할 수 없는 상태가 되었을 수 있음
		\end{itemize}
	\end{itemize}
\item 기술 혁신 양상의 변화
	\begin{itemize}
	\item 기존 플랫폼 기업이 갖고 있는 데이터, 기술적 장점 등을 뛰어 넘을 수 없다고 예상한다면, 투자자는 새로운 기술 기업에 투자하지 않을 것
		\begin{itemize}
		\item 기존 플랫폼 기업이 보유하고 있는 데이터의 공개 여부가 중요한 역할을 할 수 있음
		\end{itemize}
	\item 기존 플랫폼 기업이 새로운 기술 기업을 인수할 것으로 예상한다면, 투자자는 새로운 기술 기업에 투자할 것
	\item $\rightarrow$ 즉, 시장을 장악해서 얻는 이윤이 아니라 인수합병에서 발생하는 이윤이 될 것
	\end{itemize}
\item 규제가 필요하지 않다는 입장
	\begin{itemize}
	\item 동태적으로 경합성이 충분하므로 사후 규제만으로 충분
	\item 플랫폼 기업간 경쟁, 온라인과 오프라인 기업 간 경쟁이 있음
	\item 플랫폼 기업에 의한 소비자 후생 감소, 데이터 우위의 중요성, 동태적 경쟁/혁신 감소의 실증적 증거는 부족한 상태 
	\end{itemize}	
\end{itemize}


\pagebreak

\section*{정리하기}
\begin{enumerate}
\item 현대 경쟁 정책의 기본 방향은 기업의 전략적 의사 결정과 시장 구조의 동태적 변화를 고려하여 사안 별 분석, 잠재적 시장 진입 가능성을 검토함으로써 정부 개입을 최소화, 정부 개입은 소비자 잉여의 감소에 근거하는 특징을 갖고 있다.
\item 경쟁을 제한하는 행태가 중요한 문제점이고, 이에 해당하는 기업 전략은 약탈 가격, 초기 위협의 배제, 우월적 지위를 남용한 인접 시장 확장 등이 대표적이다.
\item 디지털 플랫폼은 네트워크 효과, 규모의 경제, 범위의 경제, 데이터 사용에 대한 수익 체증, 전세계 시장으로의 낮은 운송 비용으로 독점화 경향이 강하다.
\item 네트워크 효과와 규모의 경제로 인해 디지털 플랫폼 산업의 진입 장벽이 있으며, 소비자 행동, 기존 기업의 정보 비대칭 활용, 기술 및 데이터 호환성 전략도 진입 장벽으로 작동한다.
\item 디지털 플랫폼의 추천 알고리듬이나 광고 알고리듬은 각각 확증 편향으로 인한 비합리적 소비 유인과 플랫폼 이용 시간 등의 증가로 소비자 잉여를 감소시킬 가능성이 있다.
\item 또한, 플랫폼은 보완재 판매의 이윤을 수취하며 이로 인해 보완재 기업의 투자를 가로막을 수 있고, 기술 혁신의 목표도 시장 장악이 아니라 인수합병의 이윤 취득으로 바뀌게 됨에 따라 기술 혁신을 가로 막을 가능성도 높다. 
\item 그럼에도 불구하고, 시장의 동태적 경합성이 있음, 플랫폼 기업 간 경쟁 및 온라인과 오프라인 간 기업의 경쟁이 있음, 플랫폼 기업에 의한 사회적 잉여 감소의 실증적 증거 부족을 근거로 디지털 플랫폼 기업에 대한 규제가 필요하지 않다는 입장도 있다.
\end{enumerate}