\chapter{데이터 기반 의사 결정}\label{cha:databaseddecision}

\section*{학습개요}
 플랫폼 산업의 디지털화가 가져온 특징과, 경제학의 관점에서 기계 학습 또는 빅데이터 분석에 기반한 의사 결정이 어떤 의미가 있는 지 살펴본다.


\section*{학습목표}
\begin{enumerate}
\item 디지털 플랫폼의 특징을 설명할 수 있다.
\item 데이터 기반 의사 결정의 절차를 설명할 수 있다.
\item 데이터 기반 의사 결정으로 좋은 결과를 얻을 수 있는 조건을 설명할 수 있다.
\item 빅 데이터와 기계학습이 의사 결정에 미친 영향을 비용 하락으로 설명할 수 있다.
\item 데이터 활용 또는 빅데이터 분석이 경제의 어떤 비용을 낮추는 지 설명할 수 있다.
\end{enumerate}

\section*{주요 용어}
기계학습, 산업 간 데이터 마이닝 표준 절차, 예측, 의사결정, 정보의 디지털 화, 디지털 경제의 비용 하락

\pagebreak

\section{디지털 플랫폼과 데이터}
\begin{itemize}
\item 전통적인 플랫폼 산업의 변화
	\begin{itemize}
	\item 광고 산업
		\begin{itemize}
		\item 전통적으로 광고 효과를 높일 수 있는 콘텐츠(attention attractors)와
		\item 광고의 효과를 파악/예측할 수 있는 중개 수단(attention brokerage)이 필요
		\end{itemize}
	\item 디지털화 $\rightarrow$ 중개 수단의 광고 효과 파악/예측의 정확도가 올라감
	\end{itemize}
\item 디지털 플랫폼
	\begin{itemize}
	\item 과거의 플랫폼에 비해
		\begin{itemize}
		\item 사용자의 데이터를 더 많이 더 폭 넓게 수집, 유지, 활용
		\item 관련된 부가 서비스의 개발
		\end{itemize}
	\end{itemize}	
\item 뉴스 애그리게이터(News aggregator)
	\begin{itemize}
	\item 알고리듬을 이용, 뉴스를 자동으로 모아서 전달
	\item 어떤 기준으로 모을 것인가? $\rightarrow$ 독자의 취향에 따라야 함
		\begin{itemize}
		\item 이전의 뉴스 검색 기록을 토대로 추천 $\rightarrow$ 독자가 선호하는 콘텐츠를 검색할 시간을 줄여줌
		\end{itemize}
	\item 다양한 파생 문제 %$\rightarrow$ \ref{cha:newsandadvertisement}장
		\begin{itemize}
		\item 균형 잡힌 시각이 아니라, 독자의 선호에 맞는 편향된 시각의 뉴스만 제공
		\item 가십성 또는 감동적인 이야기 위주의 제공
		\item 저작권 침해 여부 및 이와 관련된 경제적 보상 문제
		\end{itemize}
	\end{itemize}
\end{itemize}

\section{데이터 기반 의사 결정}\label{sec:databaseddecision}
\subsection{데이터 기반 의사 결정}
\begin{itemize}
\item 의사 결정(decision making)
	\begin{itemize}
	\item 주어진 조건에서 어떤 행동의 실행 여부를 결정
	\item 차도의 신호등이 붉은 색, 인도의 신호등이 파란색 $\rightarrow$ 차를 멈추어야 함
	\end{itemize}
\item 경제 주체는 의사 결정에 항상 직면
	\begin{itemize}
	\item 기업: 주어진 재고 현황에서 제품 A의 생산을 늘려야할 지 줄여야할 지 등
	\item 중앙은행: 현재 경제 상황에서 이자율을 높여야할 지 낮추어야할 지 등
	\end{itemize}	
\item 데이터는 어디에나 있음
	\begin{itemize}
	\item 기업 내: 기업 운영, 공급 사슬, 작업 절차, 소비자 행동, 마케팅 성과 등
	\item 기업 외: 경쟁자 동향, 산업 동향, 거시 경제 상황 등
	\end{itemize}
\item 데이터 기반 의사 결정
	\begin{itemize}
	\item 주어진 조건 $\rightarrow$ 데이터 화 $\rightarrow$ 의사결정
	\end{itemize}	
\item 데이터 분석과 경제학
	\begin{itemize}
	\item 기계 학습 또는 빅데이터 분석의 최종 단계 $\rightarrow$ 결과를 어떻게 유용하게 사용할 것인지 생각해야 함
	\item 경제학은 제약 조건에서의 최적화라는 시각을 통해, 유용한 의사 결정의 방향을 제시
	\end{itemize}
\end{itemize}

\subsection{데이터 분석의 목적과 절차}
\begin{itemize}
\item 데이터 기반 의사 결정의 순서
	\begin{enumerate}
	\item 데이터 수집 및 가공
		\begin{itemize}
		\item 도로 이미지, 라이다 정보, 레이더 정보 등
		\item $\rightarrow$ 기계가 처리 가능한 형태로 전환
		\end{itemize}
	\item 기계 학습 (machine learning)
		\begin{itemize}
		\item 데이터를 분석하는 수학적 절차 (알고리듬)
		\item 주어진 사진 $\rightarrow$ 신호등과 현재 교통 신호를 학습
		\end{itemize}
	\item 데이터 분석
		\begin{itemize}
		\item 목적: 데이터가 표상하는 것 간의 유용한 패턴 또는 상관 관계를 찾는 것
		\item 차도의 신호등이 붉은색, 인도의 신호등이 파란색
			\begin{itemize}
			\item $\rightarrow$ 차는 멈추고 사람은 인도를 건너감
			\end{itemize}
		\end{itemize}
	\item 패턴 또는 상관 관계 파악
		\begin{itemize}
		\item 데이터가 표상하는 것의 새로운 사례에 대한 추론 (inference)
		\item 새로운 도로에서 무엇이 신호등이고, 신호는 무슨 색인가?
		\end{itemize}
	\item 의사결정
		\begin{itemize}
		\item 새로운 거리에서 차도의 신호등이 붉은 색, 인도의 신호등이 파란색
		\item  $\rightarrow$ 차를 멈추어야 함
		\end{itemize}	
	\end{enumerate}
\item 산업 간 데이터 마이닝 표준 절차(CRISP-DM: Cross-Industry Standard Process for Data Mining)
	\begin{enumerate}
	\item 사업에 대한 이해
		\begin{itemize}
		\item 가장 먼저, 그리고 가장 중요한 것은 올바른 질문을 하는 것 또는 문제를 정확히 찾는 것
		\item 창의적이고 개방적인 자세로 문제 해결을 위한 가설을 수립
		\end{itemize}
	\item 데이터에 대한 이해
		\begin{itemize}
		\item 가공할 수 있는 데이터를 정확히 이해해야 함
		\item 이전 단계에서 생각한 문제점 또는 해결 가설을 검토할 수 있는 데이터를 다각도로 확인해야 함
		\end{itemize}
	\item 데이터 가공
		\begin{itemize}
		\item 가장 많은 시간이 들어가는 작업이 될 수 있음
		\item 데이터와 사업 분야에 대한 이해가 충분한 인력으로 팀을 구성하는 것이 바람직
		\item 데이터 가공 기술이 있는 인력과 사업 운영 능력이 있는 인력이 한 팀이 되는 것을 검토
		\end{itemize}
	\item 모형화 
		\begin{itemize}
		\item 사용 가능한 데이터로 해결 가설을 지지할 수 있는 지, 다양한 알고리듬으로 분석
		\item 좋은 착안점을 줄 수 있을 때까지 인내심을 갖고 데이터를 계속 다룰 수 있어야 함
		\end{itemize}
	\item 모형 평가
		\begin{itemize}
		\item 첫번째 결과를 그대로 받아들이면 안됨
		\item 다양한 기법, 시나리오 분석 등을 통해, 결과에 대한 신뢰성을 높여야 함
		\end{itemize}
	\item 결과 활용
		\begin{itemize}
		\item 충분한 수준의 신뢰성을 확보했다면, 사업 또는 기업 운영에 데이터 분석 결과를 활용해야 함
		\end{itemize}
	\end{enumerate}	
\item 정확한 문제 진단 $+$ 정확한 가설 $+$ 정확한 데이터 $\rightarrow$ 좋은 결과
\item 연관 규칙 학습(association rule learning)
	\begin{itemize}
	\item 어떤 상품 또는 서비스를 다른 상품 또는 서비스와 묶어 파는 것이 유리한가? 
		\begin{itemize}
		\item $\rightarrow$ 교차 마케팅, 교차 판매, 카탈로그  설계, 쇼핑몰 사이트 디자인, 온라인 광고 최적화, 가격 결정, 프로모션 설계 등에 활용
		\end{itemize}
	\item 정의 및 유용한 개념
		\begin{itemize}
		\item 아이템 $I=\{i_{1}, i_{2}, \ldots, i_{n} \}$		
		\item 거래 데이터 베이스 $D = \{t_{1}, t_{2}, \ldots, t_{n} \}$
		\item 관계 규칙(association rule) $X \Rightarrow Y$, 단 $X,Y \subseteq I$
		\item 지지도(support) $supp(X) = \dfrac{|t \in T; X \subseteq t|}{|T|}$
		\item 신뢰도(confidence) $ conf(X \Rightarrow Y) = \dfrac{supp(X \cup Y)}{supp(X)} $
			\begin{itemize}
			\item $supp(X \cup Y) \rightarrow P(E_{X} \cap E_{Y})$
			\item $conf(X \Rightarrow Y) \rightarrow P(E_{Y}|E_{X}) $
			\end{itemize}
		\end{itemize}	
	\item 모두 1,000번의 거래 기록이 있고, 이 중 $X$가 팔린 기록은 300번, $X$와 $Y$가 같이 팔린 기록은 150번
		\begin{itemize}
		\item $supp(X) = \dfrac{150}{1000} = 15\%$
		\item $conf(X \Rightarrow Y) = \dfrac{150}{300} = 50\%$ 
		\end{itemize}		
	\end{itemize}	
\item 연관 규칙 학습의 예
	\begin{itemize}
	\item 판매 기록
		\begin{table}[htp]
		\caption{판매 기록의 예}
		\begin{center}
		\begin{tabular}{lllll}
		\toprule
		구매자	 & 구입품 1 & 2  & 3  & 4 \\
			\midrule
		 1 & 우유 & 계란 & 빵 & 딸기잼 \\
		 2 & 우유 & 딸기잼 & 계란 & 포도잼 \\
		 3 & 빵 & 딸기잼 & 포도잼 & \\
		 4 & 우유 & 빵 & 딸기잼 & \\
		 5 & 빵 & 딸기잼 & 소시지 & \\
		 6 & 우유 & 빵 & 딸기잼 & 소시지 \\
		 7 & 우유 & 소시지 & & \\
		 8 & 우유 & 빵 & 딸기잼 & \\
		 9 & 빵 & 딸기잼 & 계란 & 소시지 \\
		 10 & 우유 & 딸기잼 & 빵 & \\
		 11 & 우유 & 빵 & 딸기잼 & \\
		 12 & 우유 & 빵 & 소시지 & 포도잼 \\
		\bottomrule
		\end{tabular}
		\end{center}
		\label{tab:transactionlist}
		\end{table}%
	\item 주어진 데이터에서는 모든 발생 사건의 확률을 구할 수 있으므로 관심있는 사건으로 좁힘 $\rightarrow$ 관심있는 사건의 기준을 설정해야 함
		\begin{itemize}
		\item 어떤 제품 집합이 자주 팔림: $supp(X) \geq 33\%$
		\item 어떤 제품 집합 $X$가 판매되었을 때, 다른 제품 집합 $Y$가 팔릴 확률: $conf(X \rightarrow Y) \geq 50\%$
		\end{itemize}
	
\pagebreak	
		
	\item 판매 기록의 축약
		\begin{enumerate}
		\item 각 제품($x_{i}$)의 판매 횟수를 계산
			\begin{table}[htp]
%			\caption{default}
			\begin{center}
			\begin{tabular}{ll}
			\toprule
%			 & \\
%			\midrule
			우유& 9 \\
			빵 & 10 \\
			딸기잼 & 10 \\
			계란 & 3 \\
			포도잼 & 3 \\
			소시지 & 5 \\
			\bottomrule
			\end{tabular}
			\end{center}
			\label{tab:allitems}
			\end{table}%
		\item 자주 팔리는 제품($supp(x_{i}) \geq 33\%$)만  추출		
			\begin{table}[htp]
%			\caption{default}
			\begin{center}
			\begin{tabular}{ll}
			\toprule
%			 & \\
%			\midrule
			우유& 9 \\
			빵 & 10 \\
			딸기잼 & 10 \\
			소시지 & 5 \\
			\bottomrule
			\end{tabular}
			\end{center}
			\label{tab:featureditems}
			\end{table}%		
		\item 두 제품이 같이 팔린 경우를 추출 
			\begin{itemize}
			\item $\rightarrow$ 자주 팔리는 조합 4개: (우유, 빵), (우유, 딸기쨈), (빵, 딸기잼), (빵, 소시지)
			\end{itemize}	
				\begin{table}[htp]
	%			\caption{default}
				\begin{center}
				\begin{tabular}{ll}
				\toprule
	%			 & \\
	%			\midrule
				우유, 빵 & 7 \\
				우유, 딸기쨈 & 7 \\
				우유, 소시지 & 3 \\
				빵, 딸기잼 & 9 \\
				빵, 소시지 & 4 \\
				딸기잼, 소시지 & 3\\
				\bottomrule
				\end{tabular}
				\end{center}
				\label{tab:twoitemssets}
				\end{table}%			
		\item 세 제품이 팔린 경우를 추출
			\begin{itemize}
			\item $\rightarrow$ 자주 팔리는 조합 1개: (우유, 빵, 딸기쨈)		
				\begin{table}[h!]
	%			\caption{default}
				\begin{center}
				\begin{tabular}{ll}
				\toprule
	%			 & \\
	%			\midrule
				우유, 빵, 딸기쨈 & 6 \\
				우유, 빵, 소시지 & 1 \\
				빵, 딸기잼, 소시지 & 3 \\
				\bottomrule
				\end{tabular}
				\end{center}
				\label{tab:threeitemssets}
				\end{table}%		
			\end{itemize}									
		\item 소비 예측 $\rightarrow$ 모두 기준을 통과			
			\begin{enumerate}
			\item (빵, 딸기잼) $\rightarrow$ 우유
				\begin{itemize}
				\item $supp(A) = \dfrac{6}{12} = 50\%$
				\item $conf(A \Rightarrow B) = \dfrac{6}{9} = 67\%$
				\end{itemize}
			\item (우유, 빵) $\rightarrow$ 딸기잼
				\begin{itemize}
				\item $supp(C) = \dfrac{6}{12} = 50\%$
				\item $conf(C \Rightarrow D) = \dfrac{6}{7} = 86\%$
				\end{itemize}
			\item (우유, 딸기잼) $\rightarrow$ 빵
				\begin{itemize}
				\item $supp(E) = \dfrac{6}{12} = 50\%$
				\item $conf(E \Rightarrow F) = \dfrac{6}{7} = 86\%$
				\end{itemize}
			\end{enumerate}				
		\end{enumerate}	
	\item 상품 두 개의 조합을 검토할 수도 있음
		\begin{itemize}
		\item 많은 상품 조합이 가능 $\rightarrow$ 관심이 있는 조합을 먼저 정해야 함을 의미
		\end{itemize}
	\end{itemize}
\item $\rightarrow$ 구매 가능성 높은 상품의 추천, 재고 및 생산 관리 등
\item 주의할 점
	\begin{itemize}
	\item (우유, 딸기잼, 빵)을 묶어서 팔 때, 판매 가능성이 높다는 결론에 도달?
	\item 다른 요소를 고려해야 함
		\begin{itemize}
		\item 묶음 가격은 검토하지 않음 $\rightarrow$ 지불가능의사 파악 필요 
		\item 또한, 모든 조합이 사업에 의미가 있는 것은 아님 $\rightarrow$ 묶음의 비용 등도 고려해야 함
		\end{itemize}
	\end{itemize}
\end{itemize}

\section{예측의 경제적 효과}
\begin{itemize}
\item 예측
	\begin{itemize}
	\item 데이터 분석의 결과를 예측에 활용
	\item 빅데이터와 기계 학습 $\rightarrow$ 예측 비용의 하락
	\end{itemize}
\item 예측 $+$ 평가 $\rightarrow$ 의사결정
	\begin{itemize}
	\item 예측과 의사결정이 동일하지 않음
		\begin{itemize}
		\item 오후에 비가 내릴 확률이 80\%이면 아침에 우산을 갖고 출근
		\item 오후 7시에 비가 내릴 확률이 50\%이고, 오후 8시에 비가 내릴 확률이 80\%라면, 퇴근 시간에 따라 우산을 준비할 지 결정
		\item 하지만, 비를 맞는 것이 부담스럽지 않은 사람이면, 크게 신경쓰지 않고 우산을 준비하지 않을 것
		\end{itemize}	
	\item 평가 $\rightarrow$ 인간의 역할
	\end{itemize}	
\item 기계적 의사 결정과 도덕률
	\begin{itemize}
	\item 인간의 평가가 없는 의사 결정 $\rightarrow$ 기계적 규칙이 필요
	\item 소시지 소비자 분석 결과
		\begin{itemize}
		\item 맥주를 살 확률 $>$ 90\% 소비자 $\rightarrow$ 마케팅 필요 없음
		\item 1\% $\leq$ 맥주를 살 확률 $\leq$ 90\% $\rightarrow$ 할인 쿠폰 발송
		\item  맥주를 살 확률 $\leq$ 1\% $\rightarrow$ 마케팅 필요 없음
		\end{itemize}
	\item 충분?
		\begin{itemize}
		\item 미성년자에게 마케팅하면 안됨
		\item 기록이 있다면, 알코올 의존이 강한 소비자에게 마케팅하면 안 됨
		\end{itemize}
	\item $\rightarrow$ 기계적 규칙을 사회적 규범에 맞출 필요가 있음
	\end{itemize}	
\item 예측 비용 하락의 경제적 효과
	\begin{itemize}
	\item 예측을 중간재로서 사용하는 수요 증가
		\begin{itemize}
		\item 예
			\begin{itemize}
			\item 도로 상황을 파악 $\rightarrow$ 운전
			\item 자율 주행: 사람 운전자가 어떻게 하는 가를 예측하는 문제로 전환
			\end{itemize}
		\item $\rightarrow$ 보완재의 가치를 높임
			\begin{itemize}
			\item 데이터 수요 증가
			\item 평가(judgement)가 부가가치를 창출
			\end{itemize}
		\item $\rightarrow$ 대체되는 것의 가치를 낮출 것	
		\end{itemize}
	\item 불확실성 감소
		\begin{itemize}
		\item 기계학습은 어떤 결과가 나타날지의 가능성을 제시
		\item 어떤 구매자의 구매 기록에 쌀, 맥주, 닭강정이 있다고 가정
			\begin{itemize}
			\item 기계학습을 통해, 다음 구매에 
			\item 쌀을 살 확률 80\%, 맥주를 살 확률 15\%, 닭강정을 살 확률 5\%로 추정
			\end{itemize}
		\item $\rightarrow$ 불확실성을 낮춤
		\end{itemize}	
	\end{itemize}
\item 예측 방법
	\begin{itemize}
	\item 확률: 분류 모형(classification model)을 주로 사용
		\begin{itemize}
		\item (어떤 조건의) 소비자가 쌀을 살 확률
		\item $\rightarrow$ 질병 진단, 콘텐츠 분류, 소비자 이탈, 사기 감지, 상품 추천, 기계 고장 등
		\end{itemize}
	\item 규모: 회귀 모형(regression model)을 주로 사용
		\begin{itemize}
		\item 소비자가 한 번의 구매에서 사는 쌀의 양
		\item $\rightarrow$ 기대 수명, 이동 시간, 대출 손실, 소비 규모, 구매 간격, 통화 시간, 반응 시간 등
		\end{itemize}
	\end{itemize}	
\end{itemize}

\pagebreak

\section*{정리하기}
\begin{enumerate}
\item 디지털 플랫폼은 과거의 플랫폼에 비해 사용자의 데이터를 더 많이 더 폭 넓게 수집, 유지, 활용하고, 이를 바탕으로 관련된 부가 서비스를 개발할 수 있다는 특징이자 장점을 갖고 있다.
\item 정확한 문제 진단, 정확한 가설, 정확한 데이터가 결합될 때 좋은 결과를 얻을 수 있다.
\item 연관 규칙 학습은 지지도와 신뢰도를 기준으로, 어떤 상품 또는 서비스와 다른 상품 또는 서비스의 소비 관계를 설명한다.
\item 빅데이터의 기계 학습으로 예측 비용이 하락했다. 
\item 예측에 평가를 결합하여 의사 결정을 내린다.
\item 예측 비용이 하락하므로 예측을 중간재로 사용하는 수요가 늘어나고, 따라서 예측을 보완하는 상품이나 서비스의 가치는 높아지지만, 대체되는 상품이나 서비스의 가치는 낮아진다.
\item 데이터의 활용 또는 빅데이터 분석은 탐색, 추적, 인증 비용을 낮춘다.
\end{enumerate}