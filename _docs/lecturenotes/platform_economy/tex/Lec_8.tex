\chapter{뉴스와 광고: 구글과 페이스북}\label{cha:newsandadvertisement}

\section*{학습개요}
광고 수입 기반 플랫폼 비즈니스의 특징을 학습한다.

\section*{학습목표}
\begin{enumerate}
\item 광고시장에서 온라인 광고의 비중과 성장을 이해한다.
\item 온라인 광고의 종류와 가격 설정 방식을 이해한다.
\item 온라인 광고의 특징을 설명할 수 있다.
\item 플랫폼으로서 구글과 페이스북의 사업 특징을 설명할 수 있다.
\item 온라인 광고의 소비자 이득과 위험을 설명할 수 있다.
\item 온라인 광고의 광고주 및 콘텐츠 생산자 이득과 전통적인 경제주체의 위험을 설명할 수 있다.
\end{enumerate}

\section*{주요 용어}
온라인 광고, 맞춤형 광고, 순차 공개 가격, 유인 가격

\pagebreak

\section{온라인 광고 산업과 특징}\label{sec:onlinead}
\begin{itemize}
\item 표 \ref{tab:worldcontentsmarket}: 광고 시장의 높은 비중
	\begin{itemize}
	\item 광고 시장 전체의 2019--2024년의 연평균 성장률은 1.85\%로 예상 \cite[p. 11]{hangugkontencheujinheung-won:2020tl}
	\item 같은 기간, 온라인 광고 성장률은 4.68\%로 예상되어 시장 성장을 주도할 것으로 전망 
	\end{itemize}
\item 한국 광고 산업 \citep{hangugbangsong-gwang-gojinheung-gongsa:2020tl}
	\begin{itemize}
	\item 온라인 모바일 광고가 시장 성장을 주도
	\end{itemize}
			\begin{table}[htp]
			\begin{center}
			\begin{threeparttable}
			\caption{한국 매체유형별 광고비, 2016--2021}\label{tab:koreanadmarket}
%			\begin{tabularx}{\textwidth}{lrrrrrrrr}
			\begin{tabularx}{\textwidth}{lrrrrrr}
			\toprule
%			분류 & 2014 & 2015 & 2016 & 2017 & 2018 & 2019 & 2020\tnote{a} & 2021\tnote{a} \\
%			방송 & 4,185,239 & 4,463,966 &	4,135,069	 & 3,950,057 & 	3,931,829	 & 3,771,046	& 3,556,984	 & 3,468,550 \\
%			인쇄 & 2,323,891 & 2,329,706 &	2,319,341 & 	2,310,264 & 	2,347,956	 & 2,372,993	& 2,254,852	 & 2,232,215 \\
%			온라인 & 3,050,949 &  3,427,814 &	4,154,724	 & 4,775,137 & 	5,717,205 & 	6,521,929	 & 7,273,275	& 7,956,911\\
%			(PC) & 2,141,046&  2,053,373 & 	2,173,087	 & 1,909,192 & 	2,055,449 & 	1,871,643 & 	1,748,841	 & 1,799,048\\
%			(모바일) & 909,903& 1,374,442 & 	1,981,637	 & 2,865,945 & 	3,661,755 & 	4,650,286	 & 5,524,434	& 6,157,863\\
%			옥외 & 1,172,408 & 1,061,274	& 1,088,532 & 	1,305,948	 & 1,329,898 & 	1,256,765	 & 989,906	& 1,007,713 \\
%			기타 & 435,261 & 507,873	& 464,991 &	412,056	& 428,999	& 504,196 & 	474,518 & 	477,130 \\
%			총 광고비 & 11,167,749 & 11,790,634	& 12,162,657	 & 12,753,463	& 13,755,886	& 14,426,928	& 14,549,536	& 15,142,519 \\
			분류(중분류)  & 2016 & 2017 & 2018 & 2019 & 2020\tnote{a} & 2021\tnote{a} \\
			\midrule
			방송 &  4,135,069	 & 3,950,057 & 	3,931,829	 & 3,771,046	& 3,556,984	 & 3,468,550 \\
			인쇄 &	2,319,341 & 	2,310,264 & 	2,347,956	 & 2,372,993	& 2,254,852	 & 2,232,215 \\
			온라인 &	4,154,724	 & 4,775,137 & 	5,717,205 & 	6,521,929	 & 7,273,275	& 7,956,911\\
			(PC)  & 	2,173,087	 & 1,909,192 & 	2,055,449 & 	1,871,643 & 	1,748,841	 & 1,799,048\\
			(모바일) & 	1,981,637	 & 2,865,945 & 	3,661,755 & 	4,650,286	 & 5,524,434	& 6,157,863\\
			옥외 	& 1,088,532 & 	1,305,948	 & 1,329,898 & 	1,256,765	 & 989,906	& 1,007,713 \\
			기타 	& 464,991 &	412,056	& 428,999	& 504,196 & 	474,518 & 	477,130 \\
			총 광고비 	& 12,162,657	 & 12,753,463	& 13,755,886	& 14,426,928	& 14,549,536	& 15,142,519 \\
			\bottomrule
			\end{tabularx}
			\begin{tablenotes}
			\small
			\item[a] 추정치 
			\item 괄호 안의 소분류 업종은 일부만 제시
			\item 단위: 백만원
			\end{tablenotes}
			\end{threeparttable}
			\end{center}
			\end{table}%
\item 온라인 광고의 종류 \cite[pp. 15--22]{OECD:2019wn}
	\begin{itemize}
	\item 디스플레이 광고
		\begin{itemize}
		\item 웹사이트, 앱 등의 화면 상하좌우 또는 화면 위 등으로 노출되는 광고
		\end{itemize}
	\item 검색 광고: 보통 다음의 두 가지를 혼합
		\begin{itemize}
		\item 자연 검색(organic search): 질문에 대해 검색 알고리듬에 따라 결과 제시
		\item 광고 기반 검색(paid search): 질문에 대해, 광고비 지불에 근거한 우선 순위에 따라 결과 제시
		\end{itemize}
	\item 소셜 미디어 광고
		\begin{itemize}
		\item 맞춤형 광고(targeted advertising)에 유리
		\item 사용자 생산형 광고(user-generated ads) 가능: 인플루언서(influencer) 또는 사용자의 추천 등
		\end{itemize}
	\item 네이티브 광고
		\begin{itemize}
		\item 뉴스, 사용기 등 온라인 상의 일반적인 콘텐츠와 유사한 형식과 내용의 광고
		\end{itemize}
	\item 이메일 또는 문자 메시지 광고
	\end{itemize}
\item 온라인 광고의 가격 결정
	\begin{itemize}
	\item 가격 설계: 고정 가격 또는 경매 가격 
	\item 과금 방식
		\begin{itemize}
		\item 노출당 과금(CPM: cost per Mile), 유효 노출당 과금(vCPM: viewable)\footnote{구글에 따르면 광고의 56\%를 보지 않음. 1초 이상 시청한 광고만 과금}, 클릭당 과금(CPC: cost per click), 행위 당 과금(CPA: cost per act), 시청 당 과금(CPV: cost per view)\footnote{동영상 광고에 적용}	
		\end{itemize}
	\end{itemize}
\item 온라인 광고의 특징 \citep{Evans:2009aa, Goldfarb:2014aa}
	\begin{itemize}
	\item 맞춤형 광고의 비용을 낮춤
		\begin{itemize}
		\item 사용자(광고 시청자)의 정보를 수집할 수 있기 때문
			\begin{itemize}
			\item 연령, 성별, 거주지, 결혼 상태 등의 인구 통계학적 정보,
			\item 검색어를 이용 연관 광고를 제시,
			\item 클릭 흐름을 분석하여 광고를 제시 
			\item[예)] 서울에 거주하는 40대 남성, IT 기기, 논문 검색 등이 많음 $\rightarrow$ 가을 재킷을 검색어로 사용하면, 어떤 옷을 보여주는 것이 좋을까?
			\end{itemize}
		\end{itemize}
	\item 광고의 효과를 측정
		\begin{itemize}
		\item 광고 시청 시간, 클릭 수, 구매 등 특정 행동까지의 진행을 측정
		\end{itemize}
	\item 경매
		\begin{itemize}
		\item 기본적으로 제2가격 입찰제에서 출발
		\item 검색어 광고 경매는 goto.com 에서 시작 $\rightarrow$ 이후 오버추어(Overture)로 회사 명을 변경, 야후(Yahoo!)에서 인수
		\item 단순 모형부터, 클릭 수에 따른 가중치 부여, 소비자 선택 포함 등으로 모형을 확장
		\end{itemize}
	\end{itemize}
\end{itemize}

\section{광고 기반 플랫폼}
\subsection{검색: 구글}
\begin{itemize}
\item 짧은 연혁
	\begin{itemize}
	\item 1996년 스탠포드 대학(Stanford University) 박사과정 학생인 래리 페이지(Larry Page)와 세르게이 브린(Gergey Brin)의 연구 프로젝트로 시작
		\begin{itemize}
		\item `페이지랭크(PageRank)' 알고리듬 기반의 검색 엔진을 개발
		\item 연관된 웹 페이지에서의 링크 수로 웹 페이지의 중요도를 측정
		\end{itemize}
	\item 1998년 주식회사 설립, 2004년 상장
	\item 이후 일련의 서비스 출시
		\begin{itemize}
		\item 2000년 애드워즈(AdWords): 온라인 검색 광고
		\item 2002년 뉴스
		\item 2004년 지메일(Gmail)
		\item 2005년 구글 지도(Google Maps)
		\item 2008년 구글 크롬(Chrome) 인터넷 브라우저, 안드로이드(Android) 모바일 기기 운영체제, 구글 플레이(Google Play) 앱 및 콘텐츠 스토어
		\item 2011년 구글 플러스(Google$+$)		
		\end{itemize}
	\item 2015년 지주회사인 알파벳(Alphabet)을 설립하고, 완전 자회사가 됨
	\end{itemize}
\item 구글의 경영성과
			\begin{table}[htp]
			\begin{center}
			\begin{threeparttable}
			\caption{구글의 경영성과, 2016--2020}\label{tab:google}
			\begin{tabularx}{\textwidth}{lrrrrrrrr}
			\toprule
			연도 & 총 수입 & 광고   &  &   & 클라우드 & 그외 구글 서비스 & 기타  & 헷징\\
			& & 검색과 기타 & 유튜브  & 네트워크 멤버 & &  & & \\
			\midrule
			2016 & 90,702 & 79,383 & 10,601 & 288 \\
			2017 & 110,855 & 69,811 & 8,150 & 17,616 & 4,056 & 10,914 & 477  & -169\\
			2018 & 136,819 &  85,296 & 11,155 & 20,010 & 5,838 & 14,063 & 595 &-138\\
			2019 & 161,857 & 98,115 & 15,149 & 21,547 & 8,919 & 17,014& 659 & 455 \\
			2020 & 182,527 & 104,062 & 19,772 & 23,090 & 13,059 & 21,711 & 657 & 176 \\
			\bottomrule
			\end{tabularx}
			\begin{tablenotes}
			\small
			\item 단위: 백만달러
			\item 출처: 알파벳 10-K, \url{https://abc.xyz/investor/}
			\end{tablenotes}
			\end{threeparttable}
			\end{center}
			\end{table}%
\item 플랫폼으로서의 구글 \citep{OECD:2019tx}
	\begin{itemize}
	\item 사용자
		\begin{itemize}
		\item 구글이 제공하는 서비스를 사용, 하드웨어나 일부 서비스를 제외하고, 대부분 무료
			\begin{itemize}
			\item 검색, 도서, 금융, 이미지, 지도, 뉴스, 스콜라, 쇼핑, 트렌드, 크롬 브라우저, 크롬 OS, 안드로이드 OS, 워크플레이스, 블로거, 유튜브, 구글 플레이, 구글 어시스턴트, 구글 페이, 하드웨어(픽셀, 크롬북, 크롬캐스트 등)
			\item 서비스 간, 기기간 높은 호환성
			\end{itemize}
		\item 댓가로 광고 시청, 데이터 제공 등
		\end{itemize}			
	\item 광고주
		\begin{itemize}
		\item 애드워즈(AdWords)
			\begin{itemize}
			\item 전세계 광고 프로그램: 구글의 서비스 사용자에게 광고를 전달
			\item 경매 기반: 문자 또는 제시 광고, 광고를 제시하도록 하는 핵심단어에 대한 호가 및 일일 예산을 설정, 광고주의 지불가능의사, CTR, 그리고 다른 상관 요소에 의해 광고 노출 횟수를 결정 
			\item 다른 상관 요소: 웹 페이지의 콘텐츠, 사용자의 이전 방문 기록, 사용자의 흥미, 사용 기기, 위치, 인구학적 정보, 언어, 고객 매칭 등
			\end{itemize}
		\item 더블클릭 애드 익스체인지(DoubleClick Ad Exchange)
			\begin{itemize}
			\item 실시간 광고 노출 공간 경매 시장
			\item 다른 광고 매체를 연결하는 광고 브로커 역할
			\end{itemize}
		\end{itemize}
	\item 콘텐츠 공급자
		\begin{itemize}
		\item 애드센스(AdSense)
			\begin{itemize}
			\item 검색용: 자신의 홈페이지에 구글 검색 창을 설치. 동시에 검색 결과에 대한 광고가 노출됨
			\item 콘텐츠용: 구글이 웹 콘텐츠를 분석해 관련 광고를 노출
			\end{itemize}	
		\item 애드몹(AdMob)
			\begin{itemize}
			\item 앱 개발자가 애드워즈와 더블클릭 애드 익스체인지의 광고를 사용할 수 있도록 해주는 프로그램
			\end{itemize}
		\end{itemize}
	\item 수입
		\begin{itemize}
		\item 수입의 대부분은 광고 수수료
		\item 애드워즈 광고는 CPC 기반
			\begin{itemize}
			\item 광고에 대한 클릭이 발생할 때에만 광고주가 지불
			\end{itemize}
		\item 광고가 제시되는 횟수에 따라서 수수료를 지불할 수도 있음
		\item 검색 외에도, 지도, 플레이, 유튜브 등 다른 서비스에서도 광고 수입이 발생
		\end{itemize}
	\end{itemize}
\end{itemize}

\subsection{소셜 네트워크: 페이스북}
\begin{itemize}
\item 짧은 연혁
	\begin{itemize}
	\item 마크 저커버그(Mark Zuckerberg), 2003년 페이스매쉬(Facemash) 웹사이트 개설
		\begin{itemize}
		\item 페이스매쉬: 하버드 재학생의 사진과 기본 정보를 담은 온라인 명부
		\end{itemize}
	\item 더페이스북(thefacebook.com)으로 변화, 2004년 기업화
	\item 다른 대학이나 학교로 가입자 확대 $\rightarrow$ 2005년 말 가입자 600만명으로 증가
	\item 2006년 9월, 공개 가입으로 확대 $\rightarrow$ 2006년 12월 가입자 1,200만명
	\item 2008년, 페이스북 챗(Facebook Chat) 출시 $\rightarrow$ 이후, 메신저(Messenger)로 변경
	\item 2010년 7월 가입자 5억명 돌파
	\item 2012년 사진 공유 플랫폼 인스타그램(Instagram) 인수, 주식 공개
	\item 2014년 메시징 앱 왓츠앱(WhatsApp) 인수
	\end{itemize} 
\item 페이스북의 경영성과

			\begin{table}[htp]
			\begin{center}
			\begin{threeparttable}
			\caption{페이스북의 경영성과, 2012--2020}\label{tab:facebook}
			\begin{tabularx}{\textwidth}{lrrrrr}
			\toprule
			연도 & 총 수입 & 광고 & 기타 & 일 활성 사용자 & 월 활성 사용자 \\
			\midrule
			2012 & 5,089 & 4,279 & 810 & 618 & 1,056 \\
			2013 & 7,872 & 6,986 & 886 & 757 & 1,228 \\
			2014 & 12,466 & 11,492 & 974 & 890 & 1,393 \\
			2015 & 17,928 & 17,079 & 849 & 1,038 & 1,591 \\
			2016 & 27,638 & 26,885 & 753 & 1,227 & 1,860 \\
			2017 & 40,653 & 39,942 & 711 & 1,401 & 2,129 \\
			2018 & 55,838 & 55,013 & 825 & 1,523 & 2,320 \\
			2019 & 70,697 & 69,655 & 1,042 & 1,657 & 2,498 \\
			2020 & 85,965 & 84,169 & 1,796 & 1,845  & 2,797 \\
			\bottomrule
			\end{tabularx}
			\begin{tablenotes}
			\small
			\item 단위: 백만달러, 백만명(12월 31일 기준)
			\item 출처: 페이스북 10-K, \url{https://investor.fb.com/financials/default.aspx}
			\end{tablenotes}
			\end{threeparttable}
			\end{center}
			\end{table}%
\item 플랫폼으로서의 페이스북 \citep{OECD:2019tx}
	\begin{itemize}
	\item 사용자
		\begin{itemize}
		\item 페이스북의 소비자 대응 서비스를 이용하는 사용자 만을 지칭
		\item 개인, 집단, 기업 등
		\item 프로필(사진, 동영상, 연락처, 행사, 시간 대별 정보 등), 관계망(친구), 콘텐츠에 대한 코멘트, 다른 사용자와의 의사소통 등을 게시하고 공유할 수 있음
		\item 뉴스 피드(News Feed)
			\begin{itemize}
			\item 연결된 친구, 기업, 앱에서 제공하는 콘텐츠
			\item 광고도 포함됨
			\item 사용자 데이터에 기반한 알고리듬으로 개인화되어 제공
			\end{itemize}
		\item 무료로 사용
		\item 하지만, 사용하는 서비스에 대한 대가로 데이터, 트래픽, 정보(사진, 좋아요, 의견 등) 등의 비화폐적 거래가 이루어지는 셈
		\end{itemize}
	\item 광고주
		\begin{itemize}
		\item 전통적인 광고 산업에 비해, 페이스북은 광고주에게 광고 대상에 대해 많은 정보를 제공할 수 있음
		\item 사용자가 페이스북에 공유한 연령, 위치, 성별, 교육 수준, 경력, 취향 등의 정보, 
		\item 그리고, 사용자가 만든 콘텐츠를 이용해 실시간으로 사용자의 선호와 시장 동향에 대한 정보도 취합할 수 있음
		\item 이러한 정보를 토대로 광고주는 광고가 더 많은 주목을 받도록 만들도록 조정할 수 있음
		\end{itemize}
	\item 개발자/콘텐츠 공급자
		\begin{itemize}
		\item 페이스북: 개발 도구와 API를 제공 $\rightarrow$ 페이스북과 다른 앱, 웹사이트, 페이스북의 친구와의 활동 공유 등을 연결
			\begin{itemize}
			\item API (Application programming interfaces): 플랫폼의 내부 통신 프로토콜에 접속하거나 정보를 공유할 수 있도록 하는 도구
			\end{itemize}
		\item 사용자가 소비하고 있는 기사, 책, 영화, 음악 등의 공유롤 통해 판매를 촉진하고 이윤을 높일 수 있는 기회를 만듬
		\item 또한 대규모의 소비자 집단에 접근할 수 있는 기회를 얻음
		\end{itemize}
	\item 수입
		\begin{itemize}
		\item 광고
			\begin{itemize}
			\item CPC, CPM 기반
			\item 페이스북 사용자, 관련 웹사이트 또는 모바일 앱 사용자에게 광고를 보여줌
			\end{itemize}
		\item 개발자로부터의 수수료
			\begin{itemize}
			\item 사용자가 개발자에게 지불할 수 있음(게임 등)
			\item 이 중 최대 30\%를 수수료로 수취
			\end{itemize}
		\end{itemize}	
	\end{itemize}
\end{itemize}

\section{온라인 광고의 명과 암}
\begin{itemize}
\item 소비자 이득
	\begin{itemize}
	\item 적시의 맞춤형 광고 $\rightarrow$ 소비자 탐색 비용의 하락
	\item 광고 수입으로 유지되는 무료 서비스의 이용: 구글, 페이스북
	\end{itemize}
\item 소비자 위험
	\begin{itemize}
	\item 잘못된 정보의 전달
		\begin{itemize}
		\item 부정확한 가격 비교: 배송비 미포함 가격, 세일 전 또는 세일 후 가격 등 다양한 가격을 제시 $\rightarrow$ 가격에 대한 소비자 인식을 왜곡할 수 있음
		\item 순차 공개 가격(drip pricing): 낮은 가격으로 소비자를 유인한 후, 반드시 지불해야 하는 추가 비용을 제시(예) 배송비, 필수 액세서리 등)
		\item 유인 가격(bait pricing): 소량의 상품만을 낮은 가격으로 제시
		\item $\rightarrow$ 탐색과 가격 비교를 어렵게 함. 추가 요금에 대한 반응을 둔화시키고, 총 가격을 더 낮게 인식하도록 왜곡 $\rightarrow$ 주어진 총가격 대비 수요가 증가하도록 소비자 행동 편향을 유발 \citep{gimminjeong-ihwalyeong:2018vk}
		\end{itemize} 
	\item 광고를 광고로 인식하지 못할 수 있음
		\begin{itemize}
		\item 특히, 네이티브 광고나 사용자 생산형 광고
		\item 닻 내림(anchoring) 또는 틀짜기(framing) 효과로 $\rightarrow$ 광고가 없었을 때보다 더 높은 가치를 부여하는 소비자 행동 편향 유발 가능
			\begin{itemize}
			\item 닻 내림: 소비자는 다른 정보를 무시하고, 자신이 가장 중요하다고 생각하는 특정 정보를 중심으로 판단
			\item 틀짜기: 정보의 내용이 그 자체가 아니라 정보가 제시되는 방식으로부터 영향을 받음
			\end{itemize}
		\end{itemize}
	\item 결과적으로, 시장에서의 신뢰를 낮출 수 있음 $\rightarrow$ 온라인 거래의 감소 유발 가능
	\end{itemize}
\item 광고주 및 콘텐츠 생산자 이득 \citep{Evans:2009aa}
	\begin{itemize}
	\item 광고 기회의 확대
	\item 소규모 소액 광고의 확대
	\item 콘텐츠 생산에 대해 광고 수입을 얻을 수 있는 기회가 늘어남
	\end{itemize}
\item 전통적인 경제 주체의 위험
	\begin{itemize}
	\item 전통적인 중개인의 경제적 중요성 감소
		\begin{itemize}
		\item 온라인 광고가 오프라인 광고에 비해 판매자와 소비자의 직접 매칭, 소비자에게 광고 메시지 전달 등의 효율성이 높아진 것이 원인
		\item 전통적인 언론사 등
		\end{itemize}
	\item 온라인 광고로 광고 공간의 공급이 증가 $\rightarrow$ 광고 가격 하락 $\rightarrow$ 전통적인 광고 기반 매체의 수입 하락	
		\begin{itemize}
		\item 수입이 광고 기반에서 구독료 기반으로 전환
		\item $\rightarrow$ 구독료를 지불하고서라도 볼 만큼 매력적인 콘텐츠를 제공해야 함
		\end{itemize}
	\end{itemize}
\end{itemize}

\pagebreak

\section*{정리하기}
\begin{enumerate}
\item 전세계나 국내에서나 온라인 광고 시장의 비중이 높아지고 있으며, 특히 모바일 광고 시장의 성장이 두드러진다.
\item 온라인 광고 방식에는 크게 디스플레이 광고, 검색 광고, 소셜 미디어 광고, 네이티브 광고 등이 있다.
\item 온라인 광고 가격은 보통 고정 가격 또는 경매로 결정되며, 과금 방식은 노출당 과금, 유효 노출당 과금, 클릭당 과금, 행위 당 과금, 시청 당 과금 등 추적 가능한 행위에 따라 다양하다.
\item 온라인 광고는 전통적인 광고에 비해 맞춤형 광고의 비용을 낮추고, 광고의 효과를 더 정교하게 측정할 수 있으며, 경매를 통해 이윤을 높일 수 있는 특징이 있다.
\item 구글은 검색 등 자사의 서비스를 사용자에게 무료로 제공하고 광고를 노출시키고, 사용자는 광고 시청 및 데이터를 제공하게 된다. 구글은 광고주에게는 경매 기반의 광고 프로그램을 제공하고, 콘텐츠 공급자에게는 광고 분석 결과를 제공한다. 콘텐츠 공급자는 광고를 노출할 수 있는 공간을 제공하고 광고 수입의 일부를 받는다.
\item 페이스북은 자사의 서비스를 사용자에게 무료로 제공한다. 사용자는 데이터, 트래픽, 정보 등을 비화폐적 대가로 제공한다. 광고주는 페이스북을 통해 전통적인 광고 산업에 비해 광고 대상에 대한 세부적인 정보를 얻게 된다. 개발자와 콘텐츠 공급자는 페이스북이 제공하는 개발 도구와 API로 다른 앱과 활동 공유를 연결하고, 대규모 소비자 집단에 접근할 수 있는 기회를 얻는다.
\item 온라인 광고로 소비자는 적시의 맞춤형 정보를 제공함으로써 소비자 탐색 비용을 낮추고, 광고 수입으로 유지되는 서비스를 무료로 이용할 수 있는 이득을 누린다. 다른 한 편, 부정확한 가격 정보로 인해 실제의 총가격 대비 수요가 증가하는 소비자 행동 편향, 그리고 닻 내림 또는 틀짜기 효과로 광고가 없었을 때보다 더 높은 가치를 부여하는 소비자 행동 편향을 유발할 수 있다. 이는 결과적으로 시장의 신뢰를 낮추어 온라인 거래를 감소시킬 위험이 있다.
\item 광고주와 콘텐츠 생산자는 광고 기회가 늘어나고, 소규모 소액 광고가 가능해지며, 콘텐츠 생산에 대해 광고 수입을 얻을 기회를 갖게 된다. 다른 한편, 전통적으로 언론사와 같은 광고 기반의 경제 주체는 경제적 중요성이 감소하고, 광고 가격이 하락함에 따라 수입이 감소하는 어려움을 겪게 된다.
\end{enumerate}