\chapter{플랫폼의 성공과 실패}\label{cha:chikenandeggproblem}

\section*{학습개요}
성공적인 플랫폼 사업을 위한 고려 사항을 학습한다.


\section*{학습목표}
\begin{enumerate}
\item 비즈니스 모델 캔버스로 사업 전 확인해야할 요소를 정리할 수 있다.
\item 플랫폼이 성공하기 위해 필요한 요소를 설명할 수 있다.
\item 준비 -- 사업 초기 -- 임계 질량 -- 성숙의 각 단계에서 해결해야할 문제를 이해한다.
\item 기술 플랫폼과 거래 플랫폼이 각 단계에서 해결해야할 문제를 이해한다.
\end{enumerate}

\section*{주요 용어}
수입 흐름, 거버넌스, 사용자 경험

\pagebreak

\section{플랫폼 성공 전략}\label{sec:}
\begin{itemize}
\item 비즈니스 모델 캔버스 \cite[pp. 43--45]{Reillier:2017tt}
	\begin{itemize}
	\item 에어비앤비(AirBnB)의 경우
		\begin{itemize}
		\item 플랫폼이 상대해야하는 경제 주체의 분류(customer segments): 여행자(관광 또는 출장), 집 주인
		\item 플랫폼의 역할(value proposition): 전세계의 숙박 장소를 등록, 검색, 예약할 수 있는 시스템
		\item 홍보(channels): 지인 알림 또는 유료 광고
		\item 경제 주체 간 관계 발생 여부(customer Relationships): 집 주인과 여행자 간의 직접 관계 발생 여부
		\item 수입 흐름(revenue streams): 집 주인 수수료 3\%, 여행자 6\%--12\%
		\item 핵심 자원(key resources): 에어비앤비의 디지털 인프라
		\item 핵심 역량(key activities): 집 주인과 여행자의 매칭, 안전 거래, 신뢰 환경 개발, 커뮤니티 관리
		\item 주요 관계자(key partners): 기술 지원(호스팅, 데이터베이스 관리, 지불 관리), 보험사, 숙박 시설 청소 서비스, 벤처 투자자
		\item 비용 구조(cost structure): 마케팅, 기술 개발 및 유지, 법률
		\end{itemize}
	\end{itemize}
\item 플랫폼 성공의 과정 \cite[ch. 5]{Reillier:2017tt}
	\begin{itemize}
	\item 모든 면의 경제 주체를 유인해야 함(attract)
		\begin{itemize}
		\item 양면 시장 사용자를 임계 질량 수준까지 확보해야 함 $\rightarrow$ 충분한 크기의 네트워크 효과가 나타날 수 있도록
		\item 장기 성장의 토대가 됨
		\end{itemize}
	\item 각 면의 경제 주체를 매칭(match)
		\begin{itemize}
		\item 매칭의 품질이 중요 $\rightarrow$ 정보는 어디에나 많기 때문에, 원하는 상대를 정확히 매칭할 수 있어야 함
		\item 정확한 검색 결과, 취향에 따른 추천 등을 적시에 제공할 수 있어야 함
		\item 여기에는 제공하는 정보의 양이 적절해야 함도 의미
			\begin{itemize}
			\item[예)] 내일까지 과제를 해야하는 데 10권의 추천 도서를 제시하는 것이 의미가 있을까? 
			\end{itemize}
		\item 추천의 기준이 하나일 필요도 없음
			\begin{itemize}
			\item[예)] 최저가로 판매하지만 평판(리뷰/별점)이 나쁜 판매자를 최상단에 추천할 필요가 있을까?
			\end{itemize}
		\end{itemize}
	\item 거래하도록 함(transact)
		\begin{itemize}
		\item 핵심 (화폐) 거래 $+$ 기타 활동
			\begin{itemize}
			\item[예)] 페이스북이나 유튜브의 좋아요, 배달 앱에서의 별점 평가, 구글 검색 광고에서의 클릭 등 
			\end{itemize}
		\item 핵심 (화폐) 거래: 수수료 부과 등 수입 창출 방법이 상대적으로 명확
		\item 기타 활동: 수입을 늘릴 방법이 있을지 세밀하게 검토할 필요 있음
			\begin{itemize}
			\item 광고를 검색 결과 최상단에 배치하는 데 추가 과금
			\item 회원제 가입 유도 등
			\end{itemize}
		\end{itemize}
	\item 운영 최적화(optimise)
		\begin{itemize}
		\item 특히 디지털 플랫폼의 경우, 데이터 기반 활동이 가능
		\item 매칭과 거래의 모든 기록을 확보 $\rightarrow$ 플랫폼에서 의도한 결과를 끊임없이 시험하고 결과를 확인할 수 있음
			\begin{itemize}
			\item[예)] 광고를 검색 결과 화면의 최상단, 중간, 가장 아래, 왼쪽 중 어디에 배치하는 것이 가장 효과적인가?  
			\end{itemize}  
		\end{itemize}
	\end{itemize}
\item 플랫폼 성공을 위한 추가 고려 사항
	\begin{itemize}
	\item 거버넌스(governance)\footnote{거버넌스 또는 지배구조, 통치구조 등으로 번역되며 아직 합의된 단일한 번역어는 없음. 또 학문 분야마다 다소 다른 의미로 사용하기도 함. 경제적 거버넌스 특히 기업지배구조에 대한 이론적 기여로 2009년 노벨 경제학상을 수상한 윌리엄슨(Oliver E. Williamson, 1932-2020)은 거버넌스의 경제학을 ``좋은 질서와 작동 가능한 조정에 관한 연구(the study of good order and workable arrangements"로 정의 \citep{Williamson:2005ws}.}
		\begin{itemize}
		\item 플랫폼의 이해관계자, 이해관계자의 역할과 이득의 배분을 결정하는 문제
		\item 누가 플랫폼에 참여할 것인가? 어떤 행위를 보상해야할 것인가? 플랫폼 참여자 간의 분쟁은 어떻게 해결해야할 것인가 등을 다루어야 함
		\end{itemize}
	\item 신뢰(trust)
		\begin{itemize}
		\item 플랫폼 참여자가 믿을만하고 정직한 행동을 한다는 신뢰가 있어야 함
		\item 플랫폼 참여자가 안전한 환경에서 상호 작용 및 거래하도록 하는 원칙 및 규칙, 절차, 도구 등이 필요
		\item 정보 비대칭을 줄이는 것과도 밀접한 관련이 있음
		\item[예)] 어떻게 비대면으로 한 번도 만나지 않은 판매자의 물건을 구매하는가, 어떻게 남의 집에서 잠을 자는 가 등
		\item $\rightarrow$ 브랜드 가치 형성과 연결됨 
		\end{itemize}
	\item 정보 인프라(IT infrastructure)
		\begin{itemize}
		\item 디지털 플랫폼에서는 소프트웨어 스택, 데이터베이스, 서버, API, 클라우드 접속 등의 관리가 중요
		\item 이와 동시에 기계 학습, 애널리틱스 등의 분석 역량도 포함됨
		\end{itemize}
	\item 사용자 경험(user experience)
		\begin{itemize}
		\item 온라인/오프라인의 경험을 모두 고려해야 함
			\begin{itemize}
			\item 구글, 페이스북 등: 온라인 경험만
			\item 아마존, 에어비앤비 등: 온라인과 오프라인 경험을 동시에
			\end{itemize}
		\item 사용자 경험의 개선 $\rightarrow$ 네트워크 효과에 중요
		\end{itemize}
	\item 지불 수단(payments)
		\begin{itemize}
		\item 플랫폼이 사업적 성공을 거두기 위해서는 거래가 원할해야 함
		\item $\rightarrow$ 가능한 가장 편리한 방식이어야 하며, 글로벌 진출까지 고려할 수 있어야 함
		\end{itemize}
	\end{itemize}
\item 준비 단계에서 해결해야할 질문들 \cite[ch. 7]{Reillier:2017tt}
	\begin{itemize}
	\item 유인
		\begin{itemize}
		\item 주요 행위자는 누구인가?
		\item 각 면에 대해 플랫폼이 해야할 역할은 무엇인가?
		\item 각 면에 대해 플랫폼이 반드시 제공해야하는 도구나 서비스는 무엇인가?
		\end{itemize}
	\item 매칭
		\begin{itemize}
		\item 어떻게 매칭할 것인가?
		\item 매칭을 위한 주요 기준은 무엇인가?
		\end{itemize}	
	\item 거래
		\begin{itemize}
		\item 거래의 속성과 유형은 무엇인가?
		\item 핵심 거래는 무엇인가?
		\item 어느 쪽에 과금할 것인가?
		\item 어떤 방식으로 과금할 것인가?
		\end{itemize}	
	\item 운영 최적화	
		\begin{itemize}
		\item 최초에 관리해야할 데이터와 핵심 성과 지표(KPIs)는 무엇인가?
		\end{itemize}	
	\item 추가 고려사항
		\begin{itemize}
		\item 생산자와 사용자가 플랫폼에 접근하는 규칙은 무엇인가?
		\item 생산자와 사용자가 플랫폼과 관계를 맺는 규칙은 무엇인가?
		\item 플랫폼의 경제 주체와 신뢰를 만들기 위해 필요한 제도는 무엇인가?
		\item 기술 인프라를 직접 구축할 것인가 외부에서 구매할 것인가?
		\end{itemize}
	\end{itemize}	
\item 사업 초기 단계에서 해결해야할 질문들 \cite[ch. 8]{Reillier:2017tt}
	\begin{itemize}
	\item 유인
		\begin{itemize}
		\item 어느 면의 사용자가 먼저 증가하는가?
		\item 초기 사용자를 늘릴 수 있는 홍보 방법은 무엇인가?
		\item 사용자를 플랫폼에 머물러 있도록 하는 플랫폼의 역할은 무엇인가?
		\end{itemize}
	\item 매칭
		\begin{itemize}
		\item 사용자를 가장 잘 매칭시키는 규칙이나 필터는 무엇인가?
		\item 어떻게 플랫폼에서 쉽게 상호작용할 수 있도록 할 것인가?
		\item 최적화된 긍정적 상호 작용을 이끌어낼 수 있는 규칙은 무엇인가?
		\end{itemize}	
	\item 거래
		\begin{itemize}
		\item 언제 어떻게 과금할 것인가?
		\item 핵심 거래의 병목을 일으키는 것은 무엇인가?
		\end{itemize}	
	\item 운영 최적화	
		\begin{itemize}
		\item 계속 필요한 핵심 성과 지표는 무엇인가?
		\end{itemize}	
	\item 추가 고려사항
		\begin{itemize}
		\item 어떤 거버넌스, 신뢰, 브랜드, 사용자 경험으로 플랫폼에 머물러 있게 할 수 있을까?
		\end{itemize}
	\end{itemize}	
\item 임계질량에 도달하기 위해 해결해야할 질문들 	\cite[ch. 9]{Reillier:2017tt}
	\begin{itemize}
	\item 유인
		\begin{itemize}
		\item 기존의 플랫폼 사용자에게 최상의 관계를 어떻게 제공할 것인가?
		\item 플랫폼 사용자를 유인하고 유지하기 위한 방법은 무엇인가?
		\item 새로운 소비자 집단을 충족할 플랫폼의 역할은 무엇인가?
		\end{itemize}
	\item 매칭
		\begin{itemize}
		\item 규모가 커지면서도 적시에 최적의 매칭을 효율적으로 제공할 수 있는 방법은 무엇인가?
		\end{itemize}	
	\item 거래
		\begin{itemize}
		\item 긍정적인 관계는 늘리면서, 부정적인 관계는 줄일 수 있는 방법은 무엇인가?
		\item 어떻게 매칭을 거래로 이끌 것인가?
		\item 네트워크 효과에 부정적 결과를 주지 않으면서 수입을 늘릴 수 있는 방법은 무엇인가?
		\end{itemize}	
	\item 운영 최적화	
		\begin{itemize}
		\item 규모가 커지면서 필요한 성과 관리 지표는 무엇인가?
		\end{itemize}	
	\item 추가 고려사항
		\begin{itemize}
		\item 규모가 커지면서 변해야할 거버넌스 규칙은 무엇인가?
		\item 안전과 신뢰를 어떻게 유지할 것인가?
		\end{itemize}
	\end{itemize}	
\item 성숙 단계에서 해결해야할 질문들 	\cite[ch. 10]{Reillier:2017tt}
	\begin{itemize}
	\item 유인
		\begin{itemize}
		\item 어떻게 사용자를 늘리고 유지하면서 플랫폼의 역할을 확대해나갈 것인가?
		\end{itemize}
	\item 매칭
		\begin{itemize}
		\item 적재 적소 적시의 매칭을 어떻게 계속 개선시킬 것인가?
		\end{itemize}	
	\item 거래
		\begin{itemize}
		\item 거래를 어떻게 최적화할 것인가?
		\item 수입을 어떻게 최적화할 것인가?
		\end{itemize}	
	\item 운영 최적화	
		\begin{itemize}
		\item 성숙 단계에서 필요한 성과 관리 지표는 무엇인가?
		\end{itemize}	
	\item 추가 고려사항
		\begin{itemize}
		\item 신뢰와 안전을 어떻게 강화할 것인가?
		\item 거버넌스 원칙을 어떻게 변화시킬 것인가?
		\item 다양한 참여자에게 차별화된 브랜드를 어떻게 강조할 것인가?
		\end{itemize}
	\end{itemize}		
\end{itemize}

\section{기술 플랫폼과 거래 플랫폼}
\subsection{준비 단계}
\begin{itemize}
\item 기술 플랫폼 \cite[ch. 3]{Cusumano:2019aa}
	\begin{itemize}
	\item 플랫폼에 대한 수요를 늘릴 수 있는 보완재를 확인하고, 이를 새로운 상품 또는 서비스로 만드는 것
		\begin{itemize}
		\item[예)] 마이크로소프트 윈도우즈 $+$ 마이크로소프트 오피스(엑셀) 
		\end{itemize}
	\item API를 활용하기도 함
		\begin{itemize}
		\item $\rightarrow$ 다른 개발자가 보완재를 만들 수 있도록 지원함
		\end{itemize}
	\end{itemize}
\item 거래 플랫폼
	\begin{itemize}
	\item 대부분의 거래 플랫폼은 양면에서 출발, 시간이 지남에 따라 상대하는 면을 늘려 나감
	\item 언제 얼마나 면을 늘리는 지는 전략적으로 중요 $\rightarrow$ 하지만, 사전에 이 답을 찾기는 어려움
	\item 광고에 기반한 서비스의 경우, 광고를 너무 일찍 유치하면 사용자 경험을 떨어뜨림 $\rightarrow$ 사용자 증가의 장애 요소 $\rightarrow$ 광고주에게 매력을 잃게 됨
	\end{itemize}
\end{itemize}

\subsection{초기 단계}
\begin{itemize}
\item 공통적으로 다음 전략을 생각할 수 있음
	\begin{itemize}
	\item 어느 한 면에서 먼저 투자하고 먼저 수입을 창출할 수 있을 정도가 되면, 다른 한 면에 투자하며, 지그재그로 성장을 유도
	\item 양면에 동시에 투자하고 수입을 창출 
		\begin{itemize}
		\item 플랫폼의 투자금이 충분해야 함 
		\item 승자 독식의 가능성이 높아야 함
		\item 경쟁자가 시장에 남아 있더라도 신규 진입의 가능성이나, 전환의 가능성이 낮아야 함
		\item $\rightarrow$ 모두 만족하기 힘든 조건으로, 따라서 비용이 많이 들고 위험한 전략
		\end{itemize}
	\end{itemize}
\item 기술 플랫폼 
	\begin{itemize}
	\item 플랫폼이 외부 개발자의 보완재가 필요없는 강력한 상품이나 서비스를 제공해야 함
	\item 외부 개발자의 보완재가 있더라도 다음 조건을 만족해야 함
		\begin{itemize}
		\item 플랫폼에 종속될 수 있는 기술적 장치를 마련해야 함 $\rightarrow$ 예를 들어, 플랫폼이 API를 차단하면 외부 개발자의 보완재가 작동하지 않아야 함
		\item 플랫폼이 기술 혁신을 할 때, 외부 개발자의 보완재가 영향을 미치지 않도록 충분히 모듈화되도록 설계되어야 함
		\item 외부 개발자의 핵심 기능에 플랫폼이 쉽게 접근할 수있어야 함
		\end{itemize}
	\item 만약 외부 개발자의 보완재가 충분하지 않을 때 소비자를 어떻게 유인할 것인가? 반대로 소비자가 충분하지 않을 때, 외부 개발자를 어떻게 유인할 것인가?
		\begin{itemize}
		\item 외부 개발자는 보완재를 생산하는 공급자에 그치지 않고, 그 자체로 혁신가 
		\item $\rightarrow$ 외부 개발자의 보완재 개발에 투자하거나, 결과물을 구매
		\item[예)] 애플은 자사의 지도 서비스를 출시하기 전까지 구글의 지도 서비스를 아이폰에 탑재 
		\end{itemize}
	\end{itemize}
\item 거래 플랫폼
	\begin{itemize}
	\item 무에서 유를 창조할 필요는 없음 $\rightarrow$ 공개 데이터를	수집, 분석하는 것으로부터 출발할 수 있음
	\item 어느 면에 먼저 투자할 것인가를 결정 $\rightarrow$ 다른 면을 유인할 수 있는 요소를 발굴
		\begin{itemize}
		\item[예)] 에어비앤비, 집 주인의 홍보를 지원하기 위해 전문 사진가를 고용, 호텔과 비슷한 느낌의 숙소 전경을 촬영, 호텔과는 다른 환대를 제공함을 강조 $\rightarrow$ 다른 집주인과의 차별화를 위해 집주인이 스스로 촬영 및 위생 등에 투자
		\end{itemize}
	\end{itemize}
\end{itemize}

\subsection{임계 질량에 도달}
\begin{itemize}
\item 기술 플랫폼 
	\begin{itemize}
	\item 플랫폼 자체에 대한 지불 가능 의사를 높임
		\begin{itemize}
		\item 새로운 기능을 추가
		\item 또는 외부 개발자의 보완재 개발
		\end{itemize}
	\item 보완재 판매의 이득을 플랫폼이 수취
		\begin{itemize}
		\item[예)] 구글 안드로이드 배포 후, 검색 광고 수입의 이득을 얻음 
		\end{itemize}
	\end{itemize}
\item 거래 플랫폼
	\begin{itemize}
	\item 수입 창출 방법
		\begin{itemize}
		\item 매칭에 대한 비용 부과
		\item 거래 비용 절감에 대한 수수료 부과
		\item 보완 서비스 제공에 대한 이용료 부과
		\item 보완 기술 제공에 대한 이용료 부과
		\item 광고
		\end{itemize}
	\end{itemize}
\end{itemize}

\subsection{성숙 단계}
\begin{itemize}
\item 기술 플랫폼 
	\begin{itemize}
	\item 소비자나 개발자의 기대가 상승
		\begin{itemize}
		\item 플랫폼에 책임있는 행동을 요구하기도 함
		\item $\rightarrow$ 기술 개방 수준을 결정하는 문제로 이어질 수도 있음
		\end{itemize}
	\end{itemize}
\item 거래 플랫폼
	\begin{itemize}
	\item 리뷰, 평가, 알고리듬, 플랫폼 참여자들에 대한 사회적 고려 등이 이슈화됨
	\end{itemize}
\end{itemize}

\pagebreak

\section*{정리하기}
\begin{enumerate}
\item 플랫폼이 상대해야하는 경제 주체, 플랫폼의 역할, 홍보, 경제 주체 간의 관계, 수입 흐름, 핵심 자원, 핵심 역량, 주요 관계자, 비용 구조를 파악하여 사업 전 필요한 요소를 확인할 수 있다.
\item 플랫폼이 성공하기 위해서는 모든 면의 경제주체를 유인, 각 면의 경제 주체를 매칭, 경제 주체 간의 거래, 운영 최적화를 해야 한다.
\item 이외에도 거버넌스, 신뢰 구조, 정보 인프라, 사용자 경험, 지불 수단도 고려해야 한다.
\item 플랫폼의 준비, 사업 초기, 임계 질량 도달, 성숙의 각 단계에서 해결해야하는 문제는 각기 다르다.
\item 기술 플랫폼은 준비 단계에서는 보완재와 API, 초기 단계에서는 외부 개발자와의 관계, 임계 질량 도달 단계에서는 플랫폼 자체에 대한 지불가능의사를 높이기 위한 방법, 성숙 단계에서는 기대 및 기술 개방에 대한 요구를 검토해야 한다.
\item 거래 플랫폼은 준비 단계에서는 어느 면의 사용자를 먼저 늘릴지, 초기 단계에서는 어느 면에 먼저 투자할 것인지, 임계 질량 도달 단계에서는 수입 창출 방법, 성숙 단계에서는 신뢰를 높이기 위한 방법을 검토해야 한다.
\end{enumerate}