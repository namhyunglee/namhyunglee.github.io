\chapter{플랫폼과 노동: 현실과 정책}\label{cha:laborpolicy}

\section*{학습개요}
플랫폼 노동과 관련된 국제 및 국내 통계, 관련 문제에 대한 해외 정책 방향과 사례를 학습한다.

\section*{학습목표}
\begin{enumerate}
\item 플랫폼 중개 노동자 관련 통계 작성이 어려움을 이해한다.
\item 국제노동기구와 우리나라의 플랫폼 중개 노동자 통계 작성 기준을 이해한다.
\item 관련 통계에서 나타나는 플랫폼 중개 노동의 특징을 이해한다.
\item 해외 각 국의 플랫폼 노동 관련 정책 방향과 사례를 이해한다.
\end{enumerate}

\section*{주요 용어}
노동 3권, 안전 및 보건, 데이터 접근과 사생활 보호

\pagebreak

\section{플랫폼 노동 관련 통계}\label{sec:}
\subsection{국제노동기구}
\begin{itemize}
\item 디지털 노동 중개 플랫폼 기업 수 \citep{International-Labour-Office:2021uk}
	\begin{itemize}
	\item 2021년 1월 현재 총 777개로 추정
	\item 배달 383개, 프리랜서 181개, 택시 106개, 미세 과업 46개, 경연 기반 37개, 경쟁 프로그래밍 19개
	\end{itemize}
\item 디지털 노동 중개 플랫폼 기업에 대한 투자와 수입
	\begin{itemize}
	\item 투자: 아시아 570억 달러, 북미 460억 달러, 유럽 120억 달러로 전체 96\%를 차지; 택시/ 배달에 투자 집중
	\item 수입: 대상 기업 중 243개만 확인 가능, 미국(49\%), 중국(23\%), 유럽(11\%), 기타 지역(17\%)
	\end{itemize}
\item 디지털 노동 중개 플랫폼을 사용하는 노동자(이하 플랫폼 노동자로 약칭)
	\begin{itemize}
	\item 온라인 웹 기반: 전세계 100개국, 2,900명 $+$ 중국(1,107명), 우크라이나(761명) 부가 조사
	\item 위치 기반: 11개국, 5,000명 조사 $+$ 비교를 위해 전통적인 택시 (9개국)와 배달(4개국) 총 2,200명 부가 조사
	\item 여성은 전체 38\%, 하지만 경쟁 프로그래밍은 2\% 등으로 분야별 차이가 큼
	\item 연령
		\begin{itemize}
		\item 온라인 웹 기반 평균 31세(선진국 35세, 개발도상국 30세)
		\item 택시/배달: 플랫폼 택시 36세(전통 44세), 배달 29세(31세)
		\end{itemize}
	\item 학력 수준별
		\begin{itemize}
		\item 온라인 웹 기반: 대졸 이상 남성 62\%, 여성 65\%; 개발도상국(73\%)이 선진국(61\%)보다 대졸 이상 비중이 높았으며, 개발도상국 여성의 대졸 이상 비중이 매우 높음(80\%)
		\item 택시/배달: 각각 24\%와 21\%가 대졸 이상. 통상 저학력 저숙련 노동자가 많은 것으로 알려진 것에 비해 학력 수준이 높은 것으로 나타남
		\end{itemize}
	\end{itemize}
\item 플랫폼 노동 조건
	\begin{itemize}
	\item 수입과 노동 시간
	
			\begin{table}[htp]
			\begin{center}
			\begin{threeparttable}
			\caption{온라인 웹 기반 플랫폼 노동자 소득과 노동 시간}\label{tab:ILO2021income}
			\begin{tabularx}{\textwidth}{lrrrr}
			\toprule
			 구분 & 주 소득원  & 시간 당 평균 임금\tnote{a} & 시간 당 평균 총 임금\tnote{b} & 주당  \\
			 &  비율(\%) & (중위 임금)(달러) & (중위 임금)(달러) & 노동시간\tnote{b} \\
			\midrule
			미세 과업 & 32 & 4.4(3.0) & 3.3 (2.2) & 24  \\
			프리랜서 & 59 & 11.2(7.2) & 7.6 (5.3)  & 30 \\
			경쟁 프로그래밍 & 3 & - & - & 18 \\
			\midrule 
			선진국 & 29 & 6.1(4.5) & 4.5(3.3)  & 20  \\
			개발도상국 & 44 & 4.1(2.0) & 2.8(1.4) & 32  \\
			\midrule
			남성 & 29 & 5.0(3.0) & 3.5(2.2)  & 28 \\
			여성 & 32 & 4.8(2.9) & 3.4(1.9)  & 26 \\
			\midrule
			전체 & 36 & 4.9(3.0) & 3.4(2.1)  & 27 \\
			\bottomrule
			\end{tabularx}
			\begin{tablenotes}
			\small
			\item[a] 지급
			\item[b] 지급 $+$ 미지급 
			\end{tablenotes}
			\end{threeparttable}
			\end{center}
			\end{table}%
	
		\begin{itemize}
		\item 온라인 웹 기반 플랫폼 노동자의 약 1/3이 플랫폼 노동이 주 수입원. 하지만 경쟁 프로그래밍 노동자는 3\% 만이 주 수입원이라고 응답. 상금 수상 경험자는 12\%뿐이었으며, 상금은 수 달러에서 10,000 달러까지 다양.
		\item 매칭을 찾는 등 지급되지 않는 시간까지 포함한 임금은 평균 3.4달러. 평균 임금에 못 미치는 임금을 받는 플랫폼 노동자는 프리랜서와 미세과업 각각 64\%와 63\%. 프리랜서 플랫폼 노동자가 미세과업 플랫폼 노동자보다 2배 이상 더 높은 임금을 받고 있음
		\item 부가 조사에 따르면, 인도와 미국의 미세 과업 플랫폼 노동자는 동일한 업무의 전통 노동에 비해 각각 64\%와 81\% 낮은 임금을 받고 있음
		\item 플랫폼 상의 경력을 제시하기 위해 프리랜서 플랫폼 노동자의 62\%가 낮은 임금의 일을 받아들였으며, 60\%는 낮은 보수를 제시했고, 13\%는 무료로 일을 하기도 했음
		\item 온라인 웹 기반 플랫폼 노동자의 통상적인 주당 노동시간은 27시간으로 조사됨. 프리랜서 노동자와 개발도상국 노동자의 노동시간이 상대적으로 긴 것으로 나타남
		\item 택시와 배달 플랫폼 노동이 주 수입원으로 응답한 경우는 각각 84\%와 90\%이며 남성에 비해 여성이 응답이 더 많았음
		\item 임금은 국가별로 많은 차이를 보였는데, 택시의 경우 레바논이 8.2달러로 가장 높고 인도가 1.1달러로 가장 낮았음. 배달은 경우 우크라이나가 3.5달러로 가장 높고 가나가 0.9달러로 가장 낮았음
		\item 전통적인 서비스에 비해 플랫폼 택시 노동 수입이 더 높은 것으로 나타났지만 나라마다 그 차이는 크며(우크라이나 22\%, 가나 86\%), 배달 노동의 경우 높은 경우(케냐 39\%, 레바논 25\%)와 낮은 경우 (칠레 $-$ 24\%)도 확인됨
		\item 플랫폼 택시 노동자의 경우 차량 구입 대출 또는 임대 비용이 큰 부담이었으며, 이러한 금융 비용을 지원한 플랫폼에 고착되는 경우도 많았음. 그외 사고 시 보험 적용이 제한을 받는 것이 재정적으로 큰 제약 요소로 나타남
		\item 플랫폼 배달 노동자의 경우, 플랫폼과의 계약에 따라 소득이 달라질 수 있었음. 최소 의무 노동 및 배달을 완료하면 소득 보장 계약을 맺는 경우 최소 소득이 보장되지만, 그렇지 않은 경우 소득이 불안정함
		\item 노동시간의 경우 플랫폼 기반 택시와 배달 노동은 각각 주 65시간과 59시간 노동을 하며, 전통적인 택시와 배달 노동은 70시간과 57시간임
		\item 임금과 마찬가지로 국가별 차이가 큰데, 인도의 플랫폼 기반 택시 노동자는 82시간 일하지만, 케냐와 레바논의 주 노동시간은 63시간임
		\end{itemize}
	
	\item 사회 보장 제도의 적용
		\begin{itemize}
		\item 온라인 웹 기반 미세과업 노동자의 61\%가 건강 보험의 적용을 받고 있는 것으로 나타났는데, 이는 자신의 주업에서 보장을 받고 있거나, 배우자로부터 보장 받는 것으로 추측할 수 있음. 프리랜서와 경쟁 프로그래밍에서의 수급 응답자 비율은 낮았음. 다른 사회 보험의 적용여부에서도 이와 유사한 경향이 확인됨
		\item 개발도상국에 비해서는 선진국의 온라인 웹 기반 노동자가 더 많은 사회보장을 받고 있지만, 선진국의 사회보험 수급자 범위를 고려하면, 모든 국가군에서 부족한 상태라고 할 수 있음
		\item 이는 위치 기반 플랫폼 노동자의 경우에도 마찬가지임
		
			\begin{table}[htp]
			\begin{center}
			\begin{threeparttable}
			\caption{플랫폼 노동자의 사회 보장 급여 수급 비율}\label{tab:ILO2021socialprotection}
			\begin{tabularx}{\textwidth}{lrrrrrr}
			\toprule
			 구분 & 건강 보험 & 산재 보험 & 고용 보험 & 장애 보험 & 연금 \\
			\midrule
			미세 과업 & 61 & 21 & 16 & 13 & 35 \\
			프리랜서 & 16 & 1 & 2 & 2 & 6 \\
			경쟁 프로그래밍 & 9 & 6 & 4 & 2 & 6 \\
			\midrule 
			선진국 & 61 & 17 & 17 & 15 & 35 \\
			개발도상국 & 43 & 18 & 9 & 7 & 23 \\
			\midrule
			남성 & 42 & 18 & 13 & 12 & 21 \\
			여성 & 39 & 11 & 10& 11 & 18 \\
			\midrule
			전체 & 41 &  15 & 12 & 12 & 20 \\
			\bottomrule
			앱 기반 택시 & 51 & 27 & 5 & 4 & 18 \\
			전통 택시 & 52 & 23 & 3 & 3& 14 \\
			\midrule
			앱 기반 배달 & 53 & 31 & 7 & 6 & 17 \\
			전통 배달 & 40 & 31 & 16 & 4 & 23 \\
			\bottomrule
			\end{tabularx}
			\begin{tablenotes}
			\small
			\item 단위: \%
			\end{tablenotes}
			\end{threeparttable}
			\end{center}
			\end{table}%
		
		\end{itemize}
		\item 분쟁 해결 제도의 유무
		\begin{itemize}
		\item 프리랜서 플랫폼 노동자의 83\%는 높은 업무 평가가 새로운 업무를 받는 데 중요한 요소라고 응답. 낮은 업무 평가는 계정 정지로 이어지기도 함
		\item 업무 평가는 의뢰인과 플랫폼의 알고리듬에 의해 이루어짐. 업무 수행 결과에 대해 의뢰인이 부당하게 수령을 거부하기도 하며 이는 부정적 평가와 함께 보수 지급 정지, 그리고 미래 업무 기회 또는 계정 정지로 이어짐
		\item 약 절반 정도의 응답자만 부당한 평가에 대한 공식 이의 신청 절차가 있다는 것을 알고 있었으며, 이 절차를 안다고 응답한 경우 이를 이용한 경우는 31\% 였음. 이의를 제기한 경우 중 77\%는 만족스러운 결과를 얻었지만, 이의 제기 전보다 악화된 경우도 5\%가 있었음
		
			\begin{table}[htp]
			\begin{center}
			\begin{threeparttable}
			\caption{플랫폼 노동 분쟁 해결 제도의 활용}\label{tab:appealmechanism}
			\begin{tabularx}{\textwidth}{lccccc}
			\toprule
			 구분 & 이의 신청 & 이의 신청 & 결과 만족 & 이의 거부 & 부정적 영향\\
			 &  제도 알고 있음 & 경험 있음 & & & \\
			\midrule
			프리랜서 & 48 & 31 & 77 & 18 & 5 \\
			\bottomrule
			앱 기반 택시 & 58 & 28 & 51 & & \\
			앱 기반 배달 & 68 & 36 & 63 &  & \\
			\bottomrule
			\end{tabularx}
			\begin{tablenotes}
			\small
			\item 단위: \%
			\end{tablenotes}
			\end{threeparttable}
			\end{center}
			\end{table}%
		\item 위치 기반 플랫폼 노동에서도 평가는 매우 중요하며, 플랫폼 택시 노동자의 72\%가 평가가 업무량에 영향을 미치고, 58\%는 수입 및 이동 거리 등 업무 성격에 영향을 미친다고 응답. 플랫폼 배달 노동자의 경우 각각 65\%와 47\% 였음
		\item 공식적인 이의 신청 절차를 알고 있다는 응답자는 플랫폼 택시 노동자 58\%, 플랫폼 배달 노동자 68\% 였음. 하지만 구체적인 계약 약관을 보지 못했다는 응답자가 상당수였음 (각각 58\%, 49\%)
		\item 이의 신청 내용은 택시의 경우 지불 (48\%) -- 고객 분쟁 (35\%) -- 기술 문제 (23\%) -- 취소 (12\%), 배달의 경우 지불 (42\%) -- 취소 (36\%) -- 기술 문제 (31\%) -- 고객 불만 (24\%) 순이었음
		\item 이의 신청 결과에 불만족한 비중은 택시 49\%와 배달 37\%로 높게 나타났음
		\item 계정 정지를 경험한 사례는 19\%(택시)와 15\%(배달)였으며, 이 중 65\%(모두)가 계정 정지에 반대했고, 이 중 69\%(택시)와 83\%(배달)가 이의 신청을 했음. 결과에 만족한 것은 52\%(택시)와 41\%(배달)로 나타남
		\end{itemize} 
	\end{itemize}
\end{itemize}	
	
	
%\item 플랫폼 노동의 세분화 \citep{Schmidt:2017ws}
%\item 미국 \citep{BLS:2018ut}
%\item 유럽연합 \citep{Groen:2021uc}	
	
\subsection{한국}
\begin{itemize}
\item \cite{gimjun-yeong-gwonhyeja-choegiseong-yeonbola-bagbigon:2018aa}
	\begin{itemize}
	\item 2단계 조사
		\begin{itemize}
		\item 1단계: 만 15세 이상 3만명 표본조사 $\rightarrow$ 플랫폼 경제 종사자 규모 추정
		\item 2단계: 1단계 조사 결과 토대로, 퀵서비스, 음식배달, 대리운전, 택시 운전의 4개 직종 종사자 363명 추가 조사
		\end{itemize}
	\item 플랫폼 경제 종사자
		\begin{itemize}
		\item 정의 1: 지난 한 달 동안 디지털 플랫폼의 중개를 이용하여 유급 노동을 제공하고 수입을 얻은 경우 $+$ 지난 한 달 동안 일거리 1건당 수수료나 수수료와 정액급여 혼합 방식으로 소득이 결정되는 단기 아르바이트 앱/웹 이용자 $\rightarrow$ 46만 9천명으로 추정 (취업자 중 1.7\%)
		\item 정의 2: 위 정의 $+$ 플랫폼 경제 종사자와 지난 1년 동안 디지털 플랫폼의 중개를 이용하여 유급 노동을 제공하고 수입을 얻는 고용 형태 $\rightarrow$ 53만 8천명 (취업자 중 2.0\%)
		\end{itemize}
	\item 성별 주요 직업 (정의 1 기준)
		\begin{itemize}
		\item 남성(66.7\%): 대리운전, 화물운송, 택시운전, 판매/영업, 청소/건물 관리 순
		\item 여성(33.3\%): 음식점 보조/서빙, 가사육아도우미, 요양의료, 청소/건물관리, 판매/영업 순
		\end{itemize}	
	\item 연령별 (정의 1 기준): 40대 이상 	72.9\%
	\item 주업 53.7\% (남녀 전체)
	\item 플랫폼 경제 종사 소득과 관련 지출: 표 \ref{tab:kim2018incomeandexpenditure}
		\begin{table}[htp]
		\caption{플랫폼 경제 종사 소득과 관련 지출}
		\begin{center}
		\begin{tabular}{lccc}
		\toprule
		 직업 & 월 평균 소득 & 총 소득 중 비중(평균) & 소득을 얻기 위한 지출 비중(평균) \\
		\midrule
		대리운전 & 159만 4천원 & 63.8\% & 24.8\% \\ 
		음식배달 & 218만 3천원 & 86.6\% & 13.9\% \\ 
		퀵서비스 & 230만 1천원 & 90.3\% & 20.6\% \\ 
		택시운전 & 73만 7천원 & 30.8\% & 5.6\% \\ 
		\bottomrule
		\end{tabular}
		\end{center}
		\label{tab:kim2018incomeandexpenditure}
		\end{table}%
	\item 사회보험 가입 비율: 표 \ref{tab:kim2018socialsecurity}
		\begin{table}[htp]
		\caption{플랫폼 경제 종사자 사회보험 가입 비율}
		\begin{center}
		\begin{tabular}{lrrr}
		\toprule
		 직업 & 고용 보험 & 국민연금 & 건강보험 \\
		\midrule
		대리운전 & 27.5\% & 53.9\% & 71.6\% \\ 
		음식배달 & 10.2\% & 37.8\% & 48.0\% \\ 
		퀵서비스 & 19.6\% & 34.0\% & 54.6\% \\ 
		택시운전 & 70.4\% & 77.6\% & 98.4\% \\ 
		\bottomrule
		\end{tabular}
		\end{center}
		\label{tab:kim2018socialsecurity}
		\end{table}%		
	\item 보상 만족도: 표 \ref{tab:kim2018satisfaction}
		\begin{table}[htp]
		\caption{플랫폼 경제 종사자 보상 만족도}
		\begin{center}
		\begin{tabular}{lrrrrrr}
		\toprule
		 직업 & 수입/소득 & & 일거리 안정성 & & 일과 적성/흥미의 일치 \\
		 & 불만족 & 만족 & 불만족 & 만족 & 불만족 & 만족 \\
		\midrule
		대리운전 & 36.3\% & 16.7\% & 57.8\% & 16.7\% & 29.4\% & 22.5\% \\ 
		음식배달 & 7.1\% & 54.1\% & 24.5\% & 31.6\% & 11.2\% & 36.7\% \\ 
		퀵서비스 & 29.9\% & 27.8\% & 36.1\% & 21.6\% & 12.4\% & 23.7\% \\ 
		택시운전 & 20.8\% & 27.2\% & 16.0\% & 36.8\% & 12.0\% & 32.0\% \\ 
		\bottomrule
		\end{tabular}
		\end{center}
		\label{tab:kim2018satisfaction}
		\end{table}%				
	\end{itemize}
	
\pagebreak	
	
	
\item \cite{jangjiyeon:2020ab}
	\begin{itemize}
	\item 만 15세 이상 65세 미만 인구 9만명 설문조사
	\item 노동 중개 플랫폼 이용자 개념 구체화 
		\begin{itemize}
		\item 노동의 대가를 중개하는 플랫폼 (단순 구인 게시판 역할 사이트 이용자는 제외) $\rightarrow$ 단순 구인/구직 앱 이용자: 추정 157만명 (취업자 중 6.54\%)
		\item 플랫폼을 통한 과업이 불특정 다수에게 공개
		\item 3개월 이내 노동 중개 플랫폼 이용 $\rightarrow$ 플랫폼 노동자: 추정 22만명 (취업자 중 0.92\%)
		\end{itemize}
	\item 온라인과 오프라인 업무 구분
		\begin{itemize}
		\item 온라인: 전자 상거래, 단순 작업, 창작, IT, 전문 서비스 순
		\item 오프라인: 배달/운송, 전문서비스, 가사 , 주문 제작, 기타 순
		\end{itemize}
	\item 남성 65.3\%, 40대 이상 50.3\% (이하 응답자 중)
	\item 주업 49.7\%
		\begin{itemize}
		\item 온라인 37.8\%, 오프라인 53.2\%
		\end{itemize}
	\item 근로 일수, 시간, 소득: 표 \ref{tab:jang2020income}		
		\begin{table}[htp]
		\caption{플랫폼 노동 일, 노동 시간, 월 평균 소득, 총 소득 비중}
		\begin{center}
		\begin{tabular}{lrrrrr}
		\toprule
		& 주업 & 부업 & 온라인 & 오프라인 & 전체 \\
		\midrule
		한 달 중 노동 일 & 19.4 & 10.3 & 14.1 & 15.1 & 14.8 \\
		하루 중 노동 시간 & 8.7 & 4.3 & 5.3 & 6.9 & 6.5 \\
		월 평균 소득(만원) & 238.4 & 54.8 & 116.1 & 154.9 & 145.9 \\
		총 소득 비중(\%) & 90.9 & 21.7 & 42.1 & 60.3 & 56.0 \\
		\bottomrule
		\end{tabular}
		\end{center}
		\label{tab:jang2020income}
		\end{table}%	
		
	\item 자율성
		\begin{itemize}
		\item 가격 결정: 플랫폼 (41.7\%) -- 소속회사 (16.6\%) -- 본인 (14.8\%) -- 본인과 고객 협의 (13.7\%) -- 고객 (12.4\%) 순
		\item 업무 배정: 본인 선택 58\%
		\item 노동 시간: 본인 선택 59.5\% 
		\item 성과 평가: 있음 46.5\% $\rightarrow$ 성과 평가로 과업 양이 줄어드는 경우 52.0\%
		\item 노동 자율성 지수화: 전체 0점(22.0\%) -- 4점(26.6\%), 오프라인 0점(26.2\%) -- 4점(26.6\%)
		\end{itemize}	
	\end{itemize}
\end{itemize}



\section{관련 정책 방향}
\begin{itemize}
\item \cite{International-Labour-Office:2021uk} 권고
	\begin{itemize}
	\item 계약 상의 지위와 관계 없이 모든 플랫폼 노동자에게 적용되어야 할 요소
		\begin{itemize}
		\item 기본 원칙 및 권리 (fundamental principles and rights)
			\begin{itemize}
			\item 결사의 자유와 단체 교섭을 효과적으로 인정받을 권리 (freedom of association and effective recognition of the right to collective bargaining)
			\item 차별 금지와 동일 임금 (non-discrimination and equal remuneration)
			\item 강제 노동 철폐 (elimination of forced labor)
			\item 아동 노동 철폐 (elimination of child labor)
			\end{itemize}
		\item 노동 기준 (labour standards)
			\begin{itemize}
			\item 안전과 보건 (occupational safety and health)
			\item 사회 보장 (social security)
			\item 고용과 일자리 정책 (employment and job creation policy)
			\item 노동 감사 (labour inspection)
			\end{itemize}
		\item 플랫폼 노동에서 지켜져야할 관행적 원칙 (convention principles)
			\begin{itemize}
			\item 지불 시스템: 법정 화폐로 지급, 노동자의 자유로운 임금 처분, 부적절한 감액의 금지, 정기적이며 적시의 지급, 계약 종료 시의 지급 완료, 지급 기록 보관 등
			\item 공정한 중단: 노동 계약은 자의적이며 부당하게 중지되어서는 안됨
			\item 데이터에 대한 접근 및 사생활 보호: 사생활 보호를 원칙으로 모든 개인 데이터가 처리되어야 함, 데이터의 복사 및 열람에 대한 권리, 부정확한 데이터의 삭제 또는 수정 등도 포함
			\item 계약서의 명확한 조항: 노동자에게 노동 계약의 조항과 조건은 적절하고, 검증 가능하며, 쉽게 이해할 수 있는 방식의 문서로서 전달되어야 함
			\item 일자리 이동성: 플랫폼과의 계약 중지, 다른 플랫폼과의 계약, 플랫폼과 독립적인 노동의 보장
			\item 불만 및 분쟁 해결: 효율적이며 모든 관계자가 참여 가능해야함. 노동에 대한 평가도 대상에 포함됨
			\end{itemize}
		\end{itemize}
	\item 추가적인 고려 사항 
		\begin{itemize}
		\item 온라인 웹 기반 노동 중개 플랫폼의 특성 상, 국제 공조가 필요할 수 있음
		\item 경쟁
			\begin{itemize}
			\item 경쟁법에 의해 자영업자가 단체 교섭의 적용에서 제외되는 경우, 이에 대한 재검토 필요
			\item 경쟁법 상의 경쟁 금지 조항, 배타적 계약, 높은 수수료, 차별 대우 등에 대한 검토도 필요
			\end{itemize}
		\item 인공지능
			\begin{itemize}
			\item 알고리듬으로 과업 또는 업무의 할당, 성과 평가, 사용자 비활성화, 동적 가격이 결정
			\item $\rightarrow$ 자동화된 의사 결정으로 노동자와 기업의 위험이 결정될 수 있음: 성, 인종, 지리적 위치 등에 의한 차별 등
			\item 알고리듬 소스 코드의 접근 및 분석이 필요 $\rightarrow$ 인공지능의 투명성과 책무성 보장
			\end{itemize}
		\item 조세
			\begin{itemize}
			\item 국제적 조세 회피 $\rightarrow$ 사회 보장 제도를 위한 재원을 침해 (\ref{cha:taxation}장 참고)
			\end{itemize}
		\end{itemize}
	\item 구체적 권고 사항 
		\begin{itemize}
		\item 공정한 경쟁의 보장 및 지속 가능한 기업 활동을 위한 환경 조성
		\item 노동법 및 소비자법의 개정을 포함하여, 노동자와 기업 간의 계약 투명성 증진
		\item 노동자의 고용 상 지위를 명확히 분류하고 국가 분류 체계와 조화시킬 것
		\item 플랫폼을 이용하는 노동자나 기업에 대한 평가 투명성 제고
		\item 플랫폼을 이용하는 노동자나 기업에 대한 알고리듬의 투명성과 책무성 보장
		\item 플랫폼을 이용하는 노동자나 기업의 개인 정보, 작업 데이터, 기업 정보 보호
		\item 경쟁법과 노동법과 조화롭게, 자영업자도 단체 협약의 적용을 받을 수 있도록, 관련 제도 개정
		\item 플랫폼을 이용하는 노동자에 대한 차별 반대 및 노동 안전 보장
		\item 플랫폼을 이용하는 노동자를 포함하여 모든 노동자에게 적절한 사회 보장 급여를 지급할 것을 보장
		\item 플랫폼을 이용하는 노동자의 공정한 업무 중단을 보장
		\item 독립적인 분쟁 해결 기구의 접근을 보장
		\item 플랫폼을 이용하는 노동자가 자신이 원하는 지역에서의 사법권을 적용받을 권리를 보장
		\item 임금 보호, 공정 지불, 표준 노동 시간 제공
		\item 플랫폼을 이용하는 노동자의 플랫폼간 이동을 허용, 별점 평가 등을 포함한 노동자 데이터의 이동성을 보장
		\item 노동 의뢰인, 노동자, 노동 중개 플랫폼에 대해 효과적인 조세 제도 확립
		\end{itemize}			
	\end{itemize}
\end{itemize}	

\section{관련 정책 사례}
\begin{itemize}
\item 국가별 대응 방식과 범위는 다양 
	\begin{itemize}
	\item 사안 별 \citep{International-Labour-Office:2021uk}
		\begin{itemize}
		\item 직업 안전 및 보건
			\begin{itemize}
			\item 뉴질랜드, 브라질, 호주: 플랫폼 노동자도 적용을 받을 수 있도록 확장
			\end{itemize}
		\item 사회 보장
			\begin{itemize}
			\item 자영업자로 분류된 노동자에 대한 사고 보험 (프랑스), 사회 보장 제도 (다수의 라틴 아메리카 국가), 특정 플랫폼에서의 산업 재해 보상 (인도네시아, 말레이시아)
			\item 병가를 모든 노동자에게 적용하는 것으로 확대 (아일랜드), 플랫폼 노동자에게 고용 보험 적용 (핀란드)
			\end{itemize}
		\item 노사관계
			\begin{itemize}
			\item	 플랫폼 중개 노동자의 노동자 지위 인정이 핵심
			\end{itemize}
		\item 노동 시간과 임금
			\begin{itemize}
			\item 플랫폼 중개 노동자에게 적용할 수 있는 권리, 예를 들어, 온라인에 연결하지 않을 권리(the right to disconnect) 등을 법제화(프랑스)
			\end{itemize}	
		\item 분쟁 해결
			\begin{itemize}
			\item 계약 조항으로 분쟁 해결을 제약하여 노동자의 권리를 침해하는 경우가 있음
			\item 이러한 조항이 무효라는 판결을 하기도 함(캐나다 대법원)
			\end{itemize}
		\item 데이터 접근과 사생활 보호	
			\begin{itemize}
			\item 노동자의 보호와 사생활 보호에 대한 제도적 보장이 증가하는 추세 (나이지리아, 브라질, 유럽 연합, 인도)
			\end{itemize}
		\end{itemize}
	\item 제도 별 \citep{Garben:2019vs}: 표 \ref{tab:nationalresponsestoplatformwork}	
	\end{itemize}
	
		\begin{table}[htp]
		\caption{유럽 국가별 플랫폼 노동 관련 제도}
		\begin{center}
		\begin{tabular}{llll}
		\toprule
		 제도 & 국가 & 장점 & 단점 \\
		\midrule
		 현행 법 & 영국, 네덜란드, 벨기에,   & 사법 제도에 의한 사안 별 적용,  & 법적 불확실성, \\
		 적용 & 스웨덴, 아일랜드 등 & 법적 구속력 & 국가별 보호 수준의 차이 \\
		\midrule
		 새로운 법 & 프랑스 & 목적이 있음, & 노동자 지위 문제를 \\
		적용 & & 법적 구속력 & 해결하지 않음 \\
		\midrule
		 단체 교섭 & 덴마크 & 노동자와 고용자의 합의 & 해당 분야에만 적용, \\
		 & & & 안전과 보건 문제는 미적용 \\
		\midrule
		 자체 규제 & 독일 & 상향식 접근(bottom-up) & 법적 구속력 없음, \\
		 & & & 제한적인 적용 범위 \\
		\bottomrule
		\end{tabular}
		\end{center}
		\label{tab:nationalresponsestoplatformwork}
		\end{table}%
	
\item 유럽 의회 \citep{Expert-Group:2021wz, The-European-Parliament:2019wx}
	\begin{itemize}
	\item 유럽 사회 권리 기둥 (European Pillar of Social Rights)에 근거하여 작성
		\begin{itemize}
		\item 대상 국가: 27개 유럽 연합 회원국과 3개 유럽 자유 무역 협약국
		\item 각 국가의 이행 의무가 있는 것은 아니지만, 대상 국가는 2022년 8월까지 관련 법을 개정할 것을 권고
		\end{itemize}
	\item 플랫폼 노동으로 분류할 수 있는 노동자에게 적용
		\begin{itemize}
		\item 0시간 계약, 단순 노동, 가사 노동, 바우처 기반 노동 등
		\end{itemize}
	\item 보장해야할 권리
		\begin{itemize}
		\item 서면으로 된 근로 조건 정보 제공권
		\item 수습 기간 제한권
		\item 추가 직업 선택권(배타 및 양립 불가 조건의 노동 계약 조항이 있는 경우 제외)
		\item 노동 시작 전 합리적 노동 기간을 알 권리
		\item 0시간 계약 남용 방지
		\item 안정된 작업으로 이동을 요청하는 경우 서면 답변을 받을 권리
		\item 의무 훈련을 무료로 받을 권리
		\end{itemize}
	\end{itemize}
\item 프랑스 \citep{han-insang-sindong-yun:2019aa}
	\begin{itemize}
	\item 2016년, 노동과 사회적 대화의 현대화 그리고 직업적 경로의 보장에 관한 법
		\begin{itemize}
		\item 플랫폼 노동 종사자의 산재 보험 적용, 직업 교육, 노동 3권\footnote{노동 조합을 조직하거나 노동 조합에 가입할 수 있고(단결권), 대표를 통하여 집단적 이익을 주장할 수 있고(단체교섭권), 자신들의 요구 사항을 관철시키기 위해 권리를 남용하지 않는 범위에서 조직적으로 노동 제공을 거부할 수 있음(파업권)} 보장을 명시
		\end{itemize}
	\end{itemize}
\item 미국 \cite[7장]{jangjiyeon-ihogeun-joim-yeong-bag-eunjeong-gimgeunju:2020aa}
	\begin{itemize}
	\item 주 정부의 관련 법 또는 조례 제정
	\item 2019년 9월, 캘리포니아 주 AB5(Assembly Bill 5) 법 통과, 2020년 1월 1일부터 시행
		\begin{itemize}
		\item 독립 하청업자(independent contractor) 지위를 갖는 종속적 자영업자를 노동자로 간주할 수 있도록 함
		\item $\rightarrow$ 고용자가 ABC 테스트를 입증하지 못하는 한 플랫폼 노동 종사자는 노동자로 인정되어 최저임금, 산재보상, 실업보험, 유급병가, 가족휴가 등의 적용을 받음
			\begin{itemize}
			\item[A] 독립 하청업자는 업무 수행계약과 사실 상의 업무 수행과 관련하여 고용기관의 통제와 지시로부터 자유롭다.
			\item[B] 독립 하청업자가 고용기관 업무의 통상적인 과정 이외의 업무를 수행한다.
			\item[C] 독립 하청업자가 업무 수행과 관련된 것과 동일한 성격을 가진 독립적으로 성립된 거래, 직업 또는 사업에 관련되어 있다.
			\end{itemize} 
		\end{itemize}
	\end{itemize}
\end{itemize}

\pagebreak

\section*{정리하기}
\begin{enumerate}
\item 플랫폼 중개 노동은 기존의 노동 계약 등과 달라 개념에 대한 정의부터 필요하며, 따라서 관련 통계를 작성하는 데 많은 어려움이 있다.
\item 국제노동기구는 디지털 노동 중개 플랫폼 기업의 목록을 정리하고 이들 기업의 활동을 파악하고, 관련 노동자의 설문조사를 진행해 통계를 작성했다.
\item 우리나라는 두 개의 연구에서 서로 다른 방식으로 플랫폼 경제 종사자 또는 플랫폼 노동자를 정의하고 표본조사를 통해 관련 노동자 규모를 추정하고, 설문을 통해 플랫폼 노동에서의 특징을 확인했다.
\item 조사에서 확인되는 공통적인 특징은 플랫폼으로 중개되는 노동의 특성에 따라, 소득이나 노동 시간 등이 매우 다양하다는 것이다. 
\item 플랫폼 노동에서의 문제에 대해 현행 법 적용, 새로운 법 적용, 단체 교섭, 자체 규제 등의 다양한 방법으로 대응하고 있으나 각 방법에는 장단점이 있다.
\item 국제노동기구는 기본 원칙 및 권리(노동 3권 보장, 차별금지와 동일 임금, 강제 노동 및 아동 노동 철폐), 노동 기준(안전과 보건, 사회 보장, 고용과 일자리 정책, 노동 감사), 플랫폼 노동의 원칙(지불 시스템, 공정한 중단, 데이터 접근 및 사생활 보호, 명확한 계약 조항, 일자리 이동성, 불만 및 분쟁 해결)과 이에 근거한 구체적 권고를 제시하고 있다.
\end{enumerate}