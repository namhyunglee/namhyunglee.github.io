\chapter{시장 설계}\label{cha:mechanismdesign}

\section*{학습개요}
게임 이론과 경매 이론의 기본 모형을 학습한다. 그리고 이를 검색어 경매에 활용할 수 있음을 배운다.

\section*{학습목표}
\begin{enumerate}
\item 게임 이론의 기본 모형을 이해한다.
\item 경매 이론의 기본 모형을 이해한다.
\item 제2가격 입찰제의 기본 모형을 이해한다.
\item 제2가격 입찰제가 검색어 경매에 활용될 수 있음을 이해한다.
\item 전략적 행위자가 자신의 이익을 위해 행동하더라도 사회적으로 바람직한 목적을 달성하도록 보장할 수 있는 제도를 만들 수 있음을 이해한다.
\end{enumerate}

\section*{주요 용어}
상호의존성, 전략적 행동, 균형, 시장 설계, 제2가격 입찰제

\pagebreak

\section{게임 이론}
\begin{itemize}
\item 게임 이론
	\begin{itemize}
	\item 다수의 경제 주체 간에 상호 의존성이 있어 전략적 고려를 할 때의 합리적인 의사 결정을 연구
		\begin{itemize}
		\item 상호 의존성
			\begin{itemize}
			\item 자신의 의사결정이 자신의 효용뿐만 아니라 다른 경제주체의 효용에도 영향을 미침
			\item 동시에 다른 경제주체의 의사결정도 자신의 효용에 영향을 미침
			\end{itemize}
		\item 전략적 고려
			\begin{itemize}
			\item 경제주체가 상호의존성을 인식하고 자신에게 가장 유리한 의사결정을 하고자할 때
			\item 다른 경제주체의 의사결정이 자신의 효용에 미치는 영향을 고려
			\end{itemize}
		\end{itemize}
	\item 상품을 걸고 하는 여러 사람의 가위바위보
		\begin{itemize}
		\item 내가 가위를 내더라도, 상대방이 가위, 바위, 보 중 무엇을 내느냐에 따라 내가 이기고 지는 것이 달라짐 $\rightarrow$ 승패에 따라 얻는 것이 달라짐
		\item 내가 이기려면 상대가 무엇을 낼 지 생각해야 함 
		\item 상대도 마찬가지 
		\end{itemize}
	\item 기업의 신상품 출시, 생산량 결정 등 경쟁 전략, 투자, 투표 등도 같은 구조로 생각할 수 있음
	\end{itemize}
\item 게임의 표현
	\begin{itemize}
	\item 경기자(player): 게임 참여자, $n$명의 경기자가 참여할 경우
		\begin{align*}
		I = \{1, 2, \cdots, n \}
		\end{align*}
	\item 전략 (strategy): 경기자가 선택할 수 있는 대안 $s_{i}$
		\begin{itemize}
		\item 전략 집합: 경기자가 선택할 수 있는 전략의 집합 $s_{i} \in S_{i}$
		\item 전략 프로필: 각 경기자가 자신의 전략 집합에서 전략을 선택하여 나열한 것 $(s_{1}, s_{2})$
		\end{itemize}
	\item 보수함수 (payoff function): 가능한 모든 전략 프로필에 대해 경기자가 얻는 보상 $u_{i}$
	\item 2 명의 가위바위보
				\begin{table}[htp]
				\caption{2명의 가위바위보 게임}
				\begin{center}
				\begin{tabular}{cccc}
				\toprule
				 & 가위 & 바위 & 보\\
				\midrule
				 가위 & 0 ,0 & -1, 1 & 1, -1 \\
				 바위 & 1, -1 & 0, 0 & -1, 1 \\
				 보 & -1, 1 & 1, -1 & 0 ,0 \\
				\bottomrule
				\end{tabular}
				\end{center}
				\label{tab:twopersonsrockpaperscissorsgame}
				\end{table}%
	\end{itemize}
\item 기본 가정
	\begin{itemize}
	\item 게임의 공통 지식(Common Knowledge of Games)
		\begin{itemize}
		\item 나는 전략 집합, 전략 구성, 보수를 알고 있음 $\rightarrow$ 내가 알고 있는 것을 상대도 알고 있음 $\rightarrow$ 내가 알고 있는 것을 상대도 알고 있음을 나도 알고 있음 $\rightarrow$ $\cdots$
		\end{itemize}
	\item 공통의 합리성
	\end{itemize}
\item 협조게임 대 비협조게임
	\begin{itemize}
	\item 협조게임: 실행에 비용이 들지 않는 강한 규제가 존재(binding agreement)
		\begin{itemize}
		\item 규제가 존재한다는 의미는 규제를 어떻게 만들 것인가부터 정의되어야 하므로 분석이 어려움
		\end{itemize}
	\item 비협조게임: 강한 규제가 존재하지 않음(non-binding agreement)
		\begin{itemize}
		\item 경제학에서는 경기자가 그 어떠한 규제가 없이 자신이 손해 보지 않는 행동 만을 추구한다고 가정
		\item 친구와 가족 관계를 비협조게임으로 분석할 수 있음
		\end{itemize}
	\end{itemize}	
\item 완전 정보
	\begin{itemize}
	\item 자신을 포함한 모든 경기자들이 이제까지 어떤 선택을 했는지 알고 있는 게임
		\begin{itemize}
		\item 바둑, 장기, 체스는 모든 선택을 확인할 수 있으므로 완전정보게임
		\end{itemize}
	\end{itemize}	
\item 전략형 게임 대 전개형 게임
	\begin{itemize}
	\item 전략형 게임
		\begin{itemize}
		\item 동시선택게임
		\item 게임에 참여하는 모든 사람이 자신의 선택을 동시에 선택
		\item 시간 상 동시 선택의 의미는 아님 
		\item $\rightarrow$ 순차적으로 선택하더라도 앞에서 선택한 경기자의 전략을 뒤에 선택하는 경기자가 모른다면 동시에 전략을 선택하는 것과 동일
		\end{itemize}
	\item 전개형 게임
		\begin{itemize}
		\item 경기자의 순차적인 의사결정을 명시적으로 고려
		\item 한 경기자가 먼저 어떤 행동을 취한 다음 다른 경기자가 이를 보고 자신의 행동을 취하는 게임
		\item 모든 전략형 게임은 전개형 게임으로 표현할 수 있음
		\end{itemize}	
	\item 등을 대고 순차적으로 가위바위보를 내는 게임 $\rightarrow$ 전략형 게임	
	\end{itemize}	
\item 순수전략 대 혼합전략
	\begin{itemize}
	\item 순수전략
		\begin{itemize}
		\item 전략 집합에서 100\%의 확률로 하나의 전략을 선택
		\item[예)] 나는 항상 가위만 낸다.
		\end{itemize}
	\item 혼합전략
		\begin{itemize}
		\item 전략 집합에서 어떤 전략을 선택할 지의 확률이 존재
		\item[예)] 가위, 바위, 보를 낼 확률은 각각 33.3\% 이다.
		\end{itemize}
	\end{itemize}
\item 내시 균형(Nash Equilibrium)
	\begin{itemize}
	\item 각 경기자들이 선택한 전략에 의해 하나의 결과가 나타났을 때 
	\item 모든 경기자가 이에 만족하고 더 이상 전략을 변화시킬 의도가 없다면 균형에 도달
	\item 혼합 전략을 선택할 수 있으면 모든 게임에는 하나 이상의 내시 균형이 존재함이 증명됨
	\end{itemize}
\end{itemize}

\section{경매 이론}
\begin{itemize}
\item 시장 설계(mechanism design)
	\begin{itemize}
	\item 시장의 규칙을 만들어서, 의도하는 목표를 달성하려는 것
	\item 시장의 규칙이 거래의 결과에 어떤 결과에 미치는가? $\rightarrow$ 경매 이론으로 확인 가능
	\end{itemize}
\item 기본 가정 \cite[Lecture 2]{Roughgarden:2016aa}
	\begin{itemize}
	\item 판매자는 하나의 상품을 경매로 판매하려고 함
		\begin{itemize}
		\item 경매의 방식은 다양
		\end{itemize}
	\item 구매하려고 하는 사람은 모두 $n$ 명
	\item 입찰자 $i$는 상품에 음이 아닌 가치 (nonnegative value) $v_{i}$를 부여
	\item 입찰자의 가치 $v_{i}$는 판매자와 다른 입찰자가 알지 못하는 정보 (private information)
	\item 입찰자 $i$가 경매에서 낙찰받지 못한다면 효용은 0
	\item 입찰자가 가격 $p$를 지불하고 구입을 하면 그때 누리는 효용은 $v_{i} - p$
	\end{itemize}
\item 제도의 설계
	\begin{itemize}
	\item 입찰가 공개 대 비공개
		\begin{itemize}
		\item 가격이 없는 경우도 있음
		\end{itemize}
	\item 판매자는 누가 낙찰 받을지 결정
		\begin{itemize}
		\item 낙찰자가 없을 수도 있음
		\item 가장 높은 가격을 쓴 입찰자 1명? 아니면 여러 명? 여러 명이라면 몇 명까지?
		\item $\rightarrow$ 검색어 광고: 가장 높은 가격을 쓴 1명에게 낙찰하는 것과 다수의 낙찰자를 선정하고 평균 가격을 받는 것 중 어느 것이 이윤이 높을지
		\end{itemize}
	\item 판매자는 판매 가격을 결정
		\begin{itemize}
		\item 가장 높은 가격? 
		\item 아니면 0?
		\end{itemize}
	\end{itemize}
\item 최고 가격  비공개 입찰제 (first-price sealed-bid auction)
	\begin{itemize}
	\item 입찰자 $i$ 는 입찰가 $b_{i}$를 밀봉한 봉투에 담아 판매자에게 전달
	\item 가장 높은 가격을 쓴 입찰자에게 낙찰됨
	\item 입찰자는 자신이 제출한 입찰가 ( $p = b_{i}$)를 지불
	\item $\rightarrow$ 입찰자는 자신이 평가하고 있는 가치 $v_{i}$만 알고 있음
	\item $\rightarrow$ 그런데 경매에서 이기기 위해서는 입찰가를 높게 써야 함
	\item $\rightarrow$ 하지만 높은 가격을 쓰면 쓸 수록 경매에서 이긴 후 자신의 효용 $v_{i} - b_{i}$ 는 감소
		\begin{itemize}
		\item $\rightarrow$ 승자의 저주 (winner's curse)
		\end{itemize}
	\item $\rightarrow$ 경매의 모든 입찰자가 이 사실을 알고 있음 
	\item $\rightarrow$ 따라서, 자신이 쓸 수 있는 가장 높은 가격을 쓰기 보다 낮은 가격을 쓰려고 할 것
	\item $\rightarrow$ 점점 낮은 가격으로 입찰이 진행됨
		\begin{itemize}
		\item 즉, 높은 가격으로 입찰되리라는 보장이 없음
		\end{itemize}
	\item $\rightarrow$ 판매자에게는 바람직하지 않은 상황
	\end{itemize}		
\item 제2가격 비공개 입찰제 (second-price auction)
	\begin{itemize}
	\item 가장 높은 가격을 쓴 입찰자에게 낙찰됨
	\item 입찰자는 두 번째로 높은 입찰 가격을 지불
	\item $\rightarrow$ 입찰자가 평가하는 가치가 두 번째로 높은 입찰 가격보다 작다면 ($v_{i} < b_{j}$), 
	\item $\rightarrow$ 평가하는 가치와 같은 가격으로 솔직하게 입찰하면 낙찰받지 못하고 그 때의 효용은 0, 과대 보고 ($v_{i} <  b_{j} < b_{i}$  ) 할 때에만 경매에서 승리하고 그때는 음의 효용($v_{i} - b_{j} < 0$ )을 누림
	\item $\rightarrow$ 입찰자가 평가하는 가치가 두 번째로 높은 입찰 가격보다 크다면 ($v_{i} > b_{j}$),
	\item $\rightarrow$ 자신이 평가하는 가치보다 낮춰서 과소 보고하면 낙찰받지 못함 ($b_{i} < b_{j} < v_{i}$)
	\item $\rightarrow$ 낙찰 받기 위해서는 경쟁자의 입찰 가격보다 높아야 하고 ($b_{i} > b_{j}$), 그 때의 효용($v_{i} - b_{j}$)이 최대가 되려면, 입찰 가격은 가치와 같아야 함 ($b_{i} = v_{i} $) 
	\end{itemize}
\item 제2가격 비공개 입찰제의 특징 
	\begin{itemize}
	\item 입찰자는 자신이 평가하는 가치를 그대로 드러내는 것(true value revelation)이 다른 어떤 전략보다 항상 유리한 전략(강우월 전략, dominant strategy)이고 그렇게 하는 것이 음이 아닌 효용을 가져다 줌
	\item 판매자에게도 불리하지 않은 제도
	\item $\rightarrow$ 판매자와 입찰자 모두가 제2가격 입찰제의 결과에 만족하고 이를 바꿀 유인이 없음 
	\item $\rightarrow$ 사회적으로 바람직한 균형
	\end{itemize}
\end{itemize}

\section{검색어 경매}
\begin{itemize}
\item 검색 광고는 플랫폼 경제의 주요 수입
	\begin{itemize}
	\item 2020 회계년도 구글(Google)의 검색 수입은 약 1,469억 달러(2021년 현재 약 168조원)\footnote{출처: \url{https://abc.xyz/investor/static/pdf/20210203_alphabet_10K.pdf?cache=b44182d}, 참고로  2020년 삼성전자 수입은 약 2,006억 달러였음(\url{https://images.samsung.com/is/content/samsung/assets/global/ir/docs/2020_con_quarter04_soi.pdf})}
	\end{itemize}
\item 이 절에서는 앞에서의 경매 이론이 검색어 경매에 적용될 수 있음만 설명
	\begin{itemize}
	\item 검색어 경매는 앞에서의 간단한 경매 이론에 비해 복잡한 구조이기 때문
	\end{itemize}
\item 기본 모형
	\begin{itemize}
	\item 경매자는 검색 결과에 노출 위치 $k$ 를 판매
		\begin{itemize}
		\item 앞에서의 단순 경매 모형과 다른 점은 $k$ 가 하나가 아니라 다수
		\item 웹 페이지에서의 위치는 다양: 최상단, 중간, 아래, 팝업 등
		\item 위치가 아니라 검색 키워드도 생각할 수 있음: 방송대, 방통대, 사이버대, 온라인 학위, 방송통신대 등
		\item 위치와 키워드의 조합도 가능
		\item 앞으로는, 논의를 단순하게 하기 위해 노출 위치로만 생각
		\end{itemize}
	\item 입찰자는 노출 위치를 구매하고자 함
		\begin{itemize}
		\item 노출 위치에 따라 그 가치도 다양
		\item $\rightarrow$ 광고 클릭률 (CTRs: click-through rates): 광고를 본 사람이 클릭할 확률
			\begin{itemize}
			\item 로그 기록을 통해 측정 가능
			\item 위치 $j$의 CTR $\alpha_{j}$
			\end{itemize}
		\item $k$ 개의 노출 위치에 대해 $\alpha_{1} \geq \alpha_{2} \geq \cdots \geq \alpha_{k}$ 로 순서대로 나열할 수 있음
		\item 입찰자 $i$는 노출 위치에 대해 자신이 부여하는 가치 $v_{i}$가 있을 것
		\item 따라서, 입찰자 $i$가 노출 위치 $j$ 에서 기대하는 가치는 $v_{i}\alpha_{j}$ 
		\end{itemize}
	\end{itemize}
\item 검색어 경매 설계의 목표
	\begin{itemize}
	\item 입찰자가 자신이 평가하는 가치를 그대로 드러내고, 그 것이 다른 어떠한 전략보다 항상 유리한 전략이고, 그렇게 하는 것이 음이 아닌 효용을 가져다 주도록 해야 함
	\item 판매자와 입찰자 이득의 합을 극대화해야 함
	\item 효율적으로 계산될 수 있어야 함
		\begin{itemize}
		\item 매 초마다 검색이 이뤄질 것이므로
		\end{itemize}
	\end{itemize}
\end{itemize}

\pagebreak

\section*{정리하기}
\begin{enumerate}
\item 게임 이론은 다수의 경제 주체 간에 상호 의존성이 있어 전략적 고려를 할 때의 합리적인 의사결정을 연구한다.
\item 게임은 경기자, 전략, 보수함수로 구성된다.
\item 게임의 균형은 경기자들이 선택한 전략으로 어떤 결과가 나왔을 때, 모든 경기자가 이에 만족하고 더 이상 전략을 변화시킬 의도가 없는 상태이다. 
\item 시장 설계는 시장의 규칙을 만들어서 의도하는 목표를 달성하려는 것이다.
\item 제2가격 입찰제는 입찰자가 경매 대상에 대해 자신이 평가하고 있는 가치를 그대로 드러내도록 하는 것이 다른 어떤 전략보다 항상 유리한 전략이도록 만든다.
\item 또 제2가격 입찰제의 결과는 모든 참여자에게 음이 아닌 효용을 가져다주고 그 결과에 만족하고 이를 바꿀 유인이 없으므로 사회적으로 바람직한 균형이 된다.
\item 검색어 경매를 제2가격 입찰제와 유사한 목표와 구조로 설계할 수 있다.
\end{enumerate}