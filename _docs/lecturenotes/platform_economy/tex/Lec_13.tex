\chapter{플랫폼과 반독점: 현실과 정책}\label{cha:competitionpolicy}

\section*{학습개요}
거대 기술 기업의 반독점 혐의를 살펴보고, 이를 규제하기 위한 노력을 학습한다.

\section*{학습목표}
\begin{enumerate}
\item 현재 경쟁 정책의 플랫폼 산업에 대해 갖는 한계를 설명할 수 있다.
\item 거대 기술 기업에 대한 규제 방향을 설명할 수 있다.
\end{enumerate}

\section*{주요 용어}
경쟁 제한, 이해관계 충돌, 비차별 원칙, 데이터 이동성

\pagebreak

\section{플랫폼 기업의 현실}\label{sec:}
\subsection{플랫폼 기업 현황}
\begin{itemize}
\item GAFA \citep{Liaubet:2021tz}
	\begin{itemize}
	\item Google, Apple, Facebook, Amazon
	\item 2020년 현재, 네 기업의 매출 총합은 터키의 국내총생산에 맞먹음
	\item 의료, 게임, 영상, 음악, 지불, 커뮤니케이션, 운송 등 다방면의 사업
	\item 지난 2018--2020년 간, 네 기업이 인수합병한 기업은 모두 58개, 15억 5,900만 달러, 평균 2천 687만 달러
		\begin{itemize}
		\item 같은 기간, 미국 주식시장 에스엔피 500 (S\&P 500) 기업 중 1,000만 달러 이상의 인수는 3건
		\end{itemize}
	\end{itemize}
\item 2016--2019, 전세계 43개국, 마켓플레이스 등 특정 산업, 웹사이트 통신량(traffic) 기준 \citep{Costa:2021up}
	\begin{itemize}
	\item 대상 국가
		\begin{itemize}
		\item 2013--2014년: 아르헨티나 등 23개국
		\item 2015년: 헝가리 등 3개국 추가
		\item 2016--2019년: 한국 등 17개국 추가
		\end{itemize}
	\item 산업
		\begin{itemize}
		\item 마켓플레이스 소비자 대상 (X2C), 241개, 아마존(amazon), 이베이(ebay) 등
		\item 마켓플레이스 비즈니스 대상 (B2B), 115개, 알리바바(alibaba), 인디아마트(indiamart) 등
		\item 레스토랑 예약: 46개, 오픈테이블(opentable), 조마토(zomato) 등
		\item 레스토랑 배달: 149개, 딜리버루(deliveroo), 우버이츠(ubereats) 등
		\item 운송: 85개, 우버(uber), 그랩(grab) 등
		\item 숙박: 210개, 에어비앤비(airbnb), 부킹닷컴(booking.com) 등
		\item 전문 서비스: 컴퓨터 프로그래밍, 법률/회계, 건축 및 공학 등, 광고 및 시장 조사, 기타 전문 및 과학 기술 활동, 147개, 업워크(upwork), 프리랜서(freelancer) 등
		\item 개인 서비스: 특수 건축 활동, 건축 및 조경 활동, 컴퓨터와 개인 및 가구 재화 수리, 기타 개인 서비스 활동 등, 185개, 트리트웰(treatwell), 태스크래빗(taskrabbit) 등
		\item 모바일 지불: 150개, 페이팔(paypal), 라쿠텐(rakuten) 등
		\end{itemize}
	\item 총 플랫폼의 수와 1인당 플랫폼 웹사이트 통신량은 증가 중
	\item 하지만, 산업별, 국가별 큰 차이가 있음
	\end{itemize}
\end{itemize}

\subsection{거대 기술 기업의 반독점 혐의}\label{sec:}
\begin{itemize}
\item 페이스북 \citep{Subcommittee-on-Antitrust-Commercial-and-Administrative-Law:2020aa}
	\begin{itemize}
  \item 경쟁 회피 또는 경쟁자 제거를 목적으로 인스타그램, 왓츠앱을 포함하여 스타트 업 인수를 논의한 내부 이메일 확인
  \item 시장 우월적 데이터를 활용 잠재적 신규 기업을 확인하고, 관련 기업을 인수, 모방, 운영 중단시킴
	\end{itemize}	
\item 아마존
	\begin{itemize}
	\item 아마존 마켓플레이스를 이용하는 상당수의 중소 기업이 다른 대안이 없으므로 시장 지배력을 행사하고 있는 것으로 인정해야 함
  \item 성장 과정에서 다이어퍼스, 자포스(Diapers.com, Zappos) 등의 온라인 소매 기업을 인수
	\begin{itemize}
	\item $\rightarrow$ 기업뿐만 아니라 사용자도 인수함으로써 시장 지배력을 강화함
	\end{itemize}
  \item 내부 문서에 따르면, 마켓플레이스를 이용하는 판매자를 내부 경쟁자로 호칭
	\begin{itemize}
	\item $\rightarrow$ 직접 판매자이자 시장 관리자인 아마존의 이중적 지위로 볼 때, 이해관계의 충돌이 발생 
	\item $\rightarrow$ 판매자 정보에 대한 접근력을 활용할 유인이 됨
	\end{itemize}
  \item 성장 중인 음성 인식 시장에서 고착 효과를 누리고 있음
		\begin{itemize}
		\item 고착 효과(lock-in effect): 높은 전환 비용(switching cost)으로 다른 상품이나 서비스로 전환할 수 없음
		\end{itemize}
	\item 아마존 웹 서비스는 아마존과 경쟁 중인 다른 기업에 중요한 인프라를 제공하고 있어, 잠재적인 이해관계 충돌 가능성이 높음
		\begin{itemize}
		\item $\rightarrow$ 경쟁자에게 최적의 기술을 제공하지 않을 가능성이 있음
		\end{itemize}
	\end{itemize}	
\item 애플
	\begin{itemize}
 	\item 모바일 기기 운영 체제 시장에서 시장 지배력을 행사하고 있음
		\begin{itemize}
		\item $\rightarrow$ 높은 전환 비용, 높은 진입 장벽, 네트워크 효과로 인해 시장 지배력을 유지
		\item 경쟁 제한을 위한 장벽, 자사가 제공하는 서비스를 우선함으로써 경쟁자를 차별
		\end{itemize}
	\item 경쟁과 관련된 민감한 정보를 부적절하게 활용함으로써 앱 개발자를 차별
	\item 독점적 지위를 이용 앱 스토어 내에서 초 경쟁 가격을 부과
	\item 하드웨어 판매에서 더 많은 수입을 얻고 있지만, 앱스토어의 수수료가 수입에서 차지하는 비중이 빠르게 늘어나는 중
		\begin{itemize}
		\item $\rightarrow$ 앱스토어에서의 경쟁 약화는 앱의 혁신과 품질 향상을 가로 막고, 가격 상승과 소비자 선택의 폭을 좁히는 결과를 가져 올 것
		\end{itemize}
	\end{itemize}	
\item 구글
	\begin{itemize}
	\item 검색 시장의 독점력을 이용하여 제 3자에게 부적절한 콘텐츠를 제시하고, 구글의 열등한 수직 상품을 제공
		\begin{itemize}
		\item $\rightarrow$ 제3의 수직 공급자를 상대적으로 낮은 위치로 내림
		\end{itemize}
	\item 일반 검색의 독점력을 활용, 유료 광고 검색과 일반 검색의 경계를 흐리게 만들어, 광고 및 구글 자체의 콘텐츠를 검색 결과 페이지에 제공
	\item 스마트폰 제조사가 안드로이드 운영체제를 사용할 때, 구글의 애플리케이션이 사전 설치되고 기본 애플리이션으로 설정되도록 요구
	\end{itemize}
\end{itemize}


\section{경쟁 정책의 한계}\label{sec:}
\begin{itemize}
\item 경쟁 측정의 한계 \citep{Evans:2013vp,Jenny:2015wh}
	\begin{itemize}
	\item 한 집단의 수요의 가격 탄력성으로 측정 
		\begin{itemize}
		\item $\rightarrow$ 양면 시장(플랫폼)은 서로 다른 면에서의 수요에 대한 가격 탄력성이 서로 영향을 줌
		\end{itemize}
	\item 디지털 상품의 경우 한계 비용이 0에 가까움 
		\begin{itemize}
		\item $\rightarrow$ 마치 완전경쟁시장과 같은 결과
		\end{itemize}
	\item 어느 한 면의 사용자가 무료로 플랫폼의 서비스를 사용하는 경우, 결과가 왜곡됨
	\end{itemize}
\item 약탈 가격 적용의 어려움
	\begin{itemize}
	\item 아마존 등 전자 상거래 플랫폼의 경우 
	\item $\rightarrow$ 플랫폼을 이용하는 판매자를 가격 경쟁 시킴 
	\item $\rightarrow$ 소비자의 구매 가격 하락 
	\item $\rightarrow$ 약탈 가격의 논리가 성립하지 않음
	\end{itemize}
\item 우월적 지위를 남용한 인접 시장 확장	
	\begin{itemize}
	\item 아마존은 자사 상품의 판매와 동시에 전자 상거래 플랫폼을 운영
		\begin{itemize}
		\item 플랫폼 이용 판매자의 상품 모방 
			\begin{itemize}
			\item $\rightarrow$ 지적재산권 침해 문제이지 플랫폼의 문제는 아님
			\end{itemize}
		\item 홈페이지 등에서 자사 상품을 우선 배치, 판매 시 추가 비용 부담 등
			\begin{itemize}
			\item 입증이 불가능한 것은 아니지만 어려움
			\item 만약 플랫폼이 이를 하겠다고 의도하면 정교화할 것을 예상할 수 있음
			\end{itemize}
		\end{itemize}
	\end{itemize}
\end{itemize}

\section{경쟁 정책의 논의}
\subsection{유럽 연합}
\begin{itemize}
\item 디지털 시장 법(DMA: Digital Market Act) 
	\begin{itemize}
	\item 다음의 기업을 문지기(gatekeeper)로 추정 \citep{Cremer:2019aa,choegyeyeong:2020we,gimhyeonsu-gang-ingyu:2020aa}
		\begin{itemize}
		\item 적어도 3개 이상 회원국 에서 핵심 플랫폼 서비스 제공
		\item 최근 3년 회계연도에서 유럽 경제 지역(EEA: European Economic Area)에서 연간 매출액이 65억 유로 이상 또는 지난 1년 간 평균 주가 총액(또는 이에 상응하는 공정 시장 가치)이 650억 유로 이상
		\item 지난 회계연도 역내 활성 이용자 월 4천 5백만명 및 사업 이용자 1만명 이상
			\begin{itemize}
			\item 또는 지난 3년 회계연도 각각에서 위 기준을 충족할 경우
			\end{itemize}
		\end{itemize}
	\item 문지기 준수 의무
		\begin{itemize}
		\item 문지기는 다른 서비스나 제 3 사업자의 개인 데이터를 결합하는 행위를 금지하고 최종 이용자가 문지기의 다른 서비스에 가입하도록 하는 요구 금지
		\item 사업 이용자에게 문지기의 식별 서비스 이용 요구를 금지
		\item 사업자 및 최종 이용자에게 문지기의 다른 핵심 서비스 이용을 요구하는 행위 금지
		\item 사업 이용자의 제 3 사업자의 중개 이용을 허용
		\item 사업 이용자가 문지기를 통하지 않고 판매를 촉진할 수 있어야 하고 이에 대한 최종 이용자의 접근 및 이용이 가능해야 함
		\item 광고주 및 광고 퍼블리셔에 가격 등 요구 정보를 제공할 의무
		\item 공공 기관을 통한 이의 제기 가능
		\end{itemize}	
	\item 조정 사항
		\begin{itemize}
		\item 사전 탑재 어플리케이션의 제거(uninstall) 허용
		\item 문지기 핵심 서비스를 통하지 않고 앱이나 앱스토어를 허용
		\item 문지기의 앱스토어에 대한 사업 이용자의 공정하고 비차별적인 접근 허용
		\item 문지기 자체 및 관련 기업 우대 자제할 것
		\item 문지기의 운영체제를 이용해 다른 서비스나 다른 앱으로의 전환 또는 구독을 기술적으로 제한하는 행위를 자제할 것
		\item 사업 이용자와 보조 서비스 제공자는 문지기 보조 서비스에 사용하는 운영체제, 하드웨어, 소프트웨어 기능에 접근하고 상호 운용 허용
		\item 사업 이용자와의 경쟁을 위해 사업 이용자 및 그 최종 이용자로부터 산출된 데이터를 이용하는 행위 자제(refrain)
		\item 광고주 및 광고 퍼블리셔 요청 시 성과 측정 도구 접근 및 독립적 검증 수행에 필요한 정보 제공
		\item 사업 및 최종 이용자 활동으로 생성된 데이터에 지속적 실시간 접근을 포함하여 데이터 이동성 및 이를 원활하게 하는 도구 제공
		\item 사업 이용자(및 사업 이용자가 승인한 제 3자)에게 최종 이용자가 제3자 핵심 서비스를 이용하는 과정에서 제공 또는 생성된 집계 및 비집계 데이터 접근을 허용
			\begin{itemize}
			\item 개인 정보는 개인의 공유 허용을 전제
			\end{itemize}
		\item 검색 엔진에서 이용자가 생성한 랭킹, 쿼리, 클릭 및 보기 데이터에 대해 경쟁 검색자의 접근 요구를 따라야 함. 단 정보의 익명화는 전제되어야 함
		\end{itemize}	
	\end{itemize}
\end{itemize}

\subsection{미국}
\begin{itemize}
\item 미 하원 보고서, 디지털 시장에서의 경쟁에 대한 조사 \citep{Subcommittee-on-Antitrust-Commercial-and-Administrative-Law:2020aa,gimhyeonsu-gang-ingyu:2021aa}
	\begin{itemize}
	\item 비차별 원칙 $\rightarrow$ 자기 우대 제한, 동일 서비스에 대한 동일 계약, 결합 판매 금지 등
	\item 인접 시장 진출 제한과 기업 분할을 포함한 거대 플랫폼 구조 조정 등
	\item 전환 비용을 낮추기 위해 상호 운용성 및 데이터 이동성을 높이기 위한 개방적 서비스 접근 촉구
	\item 시장지배적 사업자의 인수 합병을 반경쟁적 행위로 추정
	\end{itemize}
\item 후속 조치로, 미 하원, 2021년 6월 플랫폼 규제 패키지 법안 발의
	\begin{itemize}
	\item 5개의 법안
		\begin{itemize}
		\item 플랫폼 독점 종식법 (Ending Platform Monopolies Act), 
		\item 진입방해 인수합병 금지 (Platform Competition and Opportunity Act),
		\item 자사제품 특혜제공 금지법 (American  Innovation and Choice Online Act),
		\item 소셜미디어 이동제한 금지법 (Augmenting Compatibility and Competition by Enabling Service Switching Act), 
		\item 합병신청 수수료 인상법 (Merger Filing Fee Modernization Act)
		\end{itemize}
	\item 대상
		\begin{itemize}
		\item 미국 내 활성 이용자 월 5,000만명, 활성 사업이용자 월 10만명
		\item 시가 총액 6,000억 달러 이상
		\item 온라인 플랫폼에서 재화와 서비스 판매를 위한 중요한 거래 상대방
		\end{itemize}
	\item 주요 내용
		\begin{itemize}
		\item 플랫폼 제공자의 자사 재화 및 서비스 판매 중단
		\item 인수 합병의 경쟁 제한성 없음에 대한 입증 책임을 플랫폼 기업에 부과
		\item 자사 제품 우대 금지
		\item 소셜 미디어 데이터 이동성 보장 등
		\end{itemize}
	\end{itemize}
\end{itemize}

%\cite{Shapiro:2019aa}
%. A tech titan putting up obstacles to customers seeking to also use rival products could easily face liability under this precedent. As a recent example of exclusionary conduct, Facebook blocked Vine, a video sharing app launched by Twitter in January 2013, from accessing Facebook user data (O’Sullivan and Gold 2018). This prevented Facebook users from inviting their Facebook friends to join Vine. Facebook was applying its policy of restricting access to apps that replicated Facebook’s core functionality. In response to the Vine episode becoming public, Facebook stated that it was dropping this policy (Facebook 2018), which appears difficult to defend. Twitter discontinued the Vine mobile app in October 2016
%
%Apple has been accused of discriminating against rivals who rely on the Apple platform to reach consumers. In March 2019, the music streaming service Spotify filed an antitrust complaint at the European Commission against Apple (Ek 2019). 
%Spotify objected to the 30 percent fee that Apple charges on certain purchases made through Apple’s payment system and claimed that Apple had locked Spotify out of Apple Watch. Spotify asserted that it should receive the same treatment at the Apple App Store given to Apple’s competing service, Apple Music. In response, Apple claimed that it had worked closely with Spotify for years and that Spotify was not willing to abide by the same rules that apply to all apps on the App Store, which Apple regards as necessary for the operation and security of the App Store. Apple further claimed that Apple approved Spotify for the Apple Watch and that Spotify has been the leading app in the Watch Music category. The European Commission is opening an investigation in response to Spotify’s complaint.
%The Spotify complaint illustrates the tensions that arise when the company controlling a platform also offers its own services on that platform. Indeed, the boundary between the “platform” and services running on that platform can be fuzzy and can change over time. Similar issues will surely arise for other applications that rely on Apple’s App Store to reach customers. For example, Apple recently removed several parental control apps from the App Store. These apps provide alternatives to Apple’s own screen-time control tools. Apple explained that it took this action to protect users’ privacy and security, but an antitrust complaint here would not be shocking (Apple 2019).
%
%We know a lot more about what a case of this type might look like against Google, because the European Commission issued an antitrust decision in June 2017 against Google involving Google Shopping, including a €2.42 billion (\$2.7 billion) fine.9 Google displays advertisements when users enter queries into the Google search engine that relate to commercial products. For example, Figure 1 shows what one sees on a desktop computer if one searches for “Nikon Cameras” on Google. 
%All of the images displayed in Figure 1 are sponsored search results; that is, they are advertisements paid for by online merchants. Google calls these “Product Listing Ads.” A user who clicks on one of these ads is directed to the website of the online merchant sponsoring that ad, and that sponsor pays a fee to Google. The first link to NikonUSA.com is also an advertisement, in text form. The second link to NikonUSA.com is a generic search result generated by Google’s algorithm, not an advertisement. More generic search results, not shown in Figure 1, follow.
%Many press reports have left the impression that the European Commission case was about Google biasing its search algorithm by demoting its rivals, but that is not correct. The European Commission fact sheet states, “The Commission Decision does not object to the design of Google’s generic search algorithms or to demotions as such, nor to the way that Google displays or organizes its search results pages (e.g., the display of a box with comparison shopping results displayed prominently in a rich, attractive format)” (European Commission 2017). Instead, the European Commission “objects to the fact that Google has leveraged its market dominance in general internet search into a separate market, comparison shopping. Google abused its market dominance as a search engine to promote its own comparison shopping service in search results, whilst demoting those  of rivals.” According to the European Commission, Google did this by displaying Product Listing Ads, such as those shown in Figure 1. This is a peculiar claim, because those Product Listing Ads are very much like the text ads that Google has shown for years, and the European Commission does not object to ads that use text rather than images. Plus, as the European Commission recognizes, there is nothing wrong from a competitive perspective when a content provider earns revenue by selling advertisements. The newspaper and radio industries have done that for a very long time. Furthermore, it is not apparent how the Product Listing Ads “promote” Google’s comparison shopping service, since a user who clicks on one of those ads is directed to the merchant’s website, not to the stand-alone Google Shopping site.

\pagebreak

\section*{정리하기}
\begin{enumerate}
\item 플랫폼 산업에서는 각 면에서의 수요에 대한 가격탄력성이 서로 영향을 주므로, 어느 한 면에서의 가격탄력성으로 경쟁 정도를 측정하는 방법은 결과를 왜곡시킬 수 있다.
\item 플랫폼 산업에서는 병목을 향한 경쟁처럼 어느 한 면의 사용자에게 경쟁을 유도할 수 있으므로, 소비자 잉여의 감소가 나타나지 않을 수 있다.
\item 우월적 지위를 남용한 인접 시장 확장의 방법도 플랫폼에 대한 반독점 정책이 아니라 지적 재산권 침해 등 다른 정책적인 대안으로 접근해야할 경우도 있다.
\item 디지털 플랫폼 기업에 대한 유럽 연합과 미국의 반독점 정책은 플랫폼 사업자 자신과 플랫폼을 이용하는 사용자 간의 비차별, 상호 운용성 및 데이터 이동성을 높이는 기술적 대안의 의무화 등을 큰 방향으로 설계되고 있다.
\end{enumerate}