\chapter{플랫폼 경제의 특징}\label{cha:whatisplatform}

\section*{학습개요}
플랫폼 경제의 특징을 설명하고, 매칭과 기술의 두 가지 성격으로 분류한다. 앞으로 배울 내용을 개관하며, 특히 다면(양면) 시장이론과 네트워크 효과와 이로 인한 플랫폼의 경제적 특성을 알아본다. 그리고 플랫폼의 사회적 특성으로서 과세, 반독점, 노동 이슈를 간략히 살펴본다.


\section*{학습목표}
\begin{enumerate}
\item 플랫폼의 특징을 설명할 수 있다.
\item 플랫폼이 최신 현상이 아님을 이해한다.
\item 인터넷 발달 이전과 이후 플랫폼에 나타난 변화를 설명할 수 있다.
\item 매칭 플랫폼과 기술 플랫폼의 특징과 수익 창출 방법을 설명할 수 있다.
\end{enumerate}

\section*{주요 용어}
플랫폼, 외부성, 수요의 가격 탄력성, 호환성, 기술 개방, 과세, 반독점, 주문형 노동


\pagebreak

\section{플랫폼의 기능과 분류}\label{sec:}
\begin{itemize}
\item 쇼핑몰, 신용카드, 텔레비전(신문), 닌텐도 게임기의 공통점은 무엇일까? \cite[pp. 991--992]{Rochet:2006jl}
	\begin{itemize}
	\item 상점 $\leftrightarrow$ 쇼핑몰 $\leftrightarrow$ 소비자
	\item 가맹점 $\leftrightarrow$ 신용카드 $\leftrightarrow$ 구매자
	\item 시청자 $\leftrightarrow$ 텔레비전 $\leftrightarrow$ 광고주
	\item 게임 개발사 $\leftrightarrow$ 게임기 제작사 $\leftrightarrow$ 게임 이용자
	\end{itemize}
\item 이들은 모두 플랫폼 \cite[p. 13]{Cusumano:2019aa} 
	\begin{quote}
	``서로 다른 종류의 사용자가 상호 활동할 수 있도록 만들어주는 것"
	\end{quote}
\item 경제학의 전통적인 관심
	\begin{itemize}
	\item 새로운 상품의 생산 $\rightarrow$ 더 낮은 가격으로 생산 $\rightarrow$ 운송 비용을 낮추어 더 효율적으로 교환
	\end{itemize}
\item 간과해온 질문: 판매자와 구매자는 어떻게 만나는가? \cite[p. 380]{Tirole:2017aa}
	\begin{itemize}
	\item 인터넷 이전: 제한된 상품과 서비스 카탈로그
		\begin{itemize}
		\item 음악을 듣기 위해서는 길거리에서 들리는 음악을 듣거나, 라디오나 TV에서 틀어주는 것을 듣거나
		\item 책을 읽기 위해서는 서점에 진열된 것을 보거나, 도서관에 소장된 것을 보거나
		\end{itemize}
	\item 인터넷 이후: 카탈로그의 길이가 무제한으로 늘어남
		\begin{itemize}
		\item $\rightarrow$ 무엇이 제공되고 있는지, 누구와 거래를 할 것인지 평가하는 것이 중요
		\item $\rightarrow$ 신호 비용(signaling costs)
		\item $\rightarrow$ 정보가 무한히 많으므로 중개인(플랫폼)을 두고 거래 상대방을 찾는 것이 효과적일 수 있음
		\end{itemize}
	\end{itemize}
\item 플랫폼의 첫 번째 유형: 매칭(matching) 플랫폼 \cite[p. 19]{Cusumano:2019aa}
	\begin{itemize}
	\item 플랫폼은 플랫폼 사용자에게 플랫폼에서 무엇을 제공하는지, 무엇이 가장 잘 맞는 지의 정보를 제공하는 것이 중요
	\item 공유 경제라고 불리는 비즈니스의 상당수가 여기에 해당
		\begin{itemize}
		\item 에어비앤비(Airbnb, 집), 우버(Uber, 차) 등
		\end{itemize}	
	\item 구글의 검색 엔진도 이 분류에 포함 \citep{Varian:2016vt}
		\begin{itemize}
		\item 검색자가 원하는 정보를 연결시켜주기 때문
		\end{itemize}
	\item 수익 창출
		\begin{itemize}
		\item 재화나 서비스의 판매 연결, 콘텐츠의 생산이나 공유 또는 광고 등
		\end{itemize}
	\end{itemize}
\item 플랫폼의 두 번째 유형: 기술(technological) 플랫폼
	\begin{itemize}
	\item 사용자 간 상호 작용을 가능하게 하는 기술적 기반(technical interface)을 제공
	\item 컴퓨터 운영체제(MS Windows, Apple MacOS, Linux 등), 모바일 기기 운영체제(Google Android, Apple iOS 등), 콘솔 게임기(Sony Playstation, MS Xbox, Nintendo 등) 등
	\item 수익 창출
		\begin{itemize}
		\item 재화나 서비스의 직접 판매 또는 대여, 보완재나 서비스의 개발 등
		\end{itemize}
	\end{itemize}
\item 혼합(Hybrid) 플랫폼? 
	\begin{itemize}
	\item 애플(Apple), 구글(Google): 모바일 기기 운영체제 $+$ 모바일 어플리케이션 판매
	\item 기업의 비즈니스를 중심으로 생각하면 가능한 분류. 하지만 분석을 위해 필요한 분류는 아님
	\end{itemize}
\end{itemize}

\section{플랫폼의 경제 이론}
\begin{itemize}
\item 다면 시장 이론(multi-sided market) \cite[pp. 383--388]{Tirole:2017aa}
	\begin{itemize}
	\item 플랫폼이 상대해야 하는 사용자를 면(side)으로 파악 $\rightarrow$ 기본은 양면 시장(two-sided market)
	\item 외부성과 수요의 가격 탄력성
		\begin{itemize}
		\item 외부성: 플랫폼의 사용자는 다른 어느 한 면의 활동으로부터 영향을 받음
			\begin{itemize}
			\item A면의 사용자가 B면의 상호 작용에서 더 많은 이득을 얻는다면, 플랫폼은 A면의 사용자에게 더 높은 이용료를 부과하고 B면의 사용자에게 상대적으로 낮은 이용료를 부과할 수 있음
			\item $\rightarrow$ 신문, 잡지, TV 등이 광고주로부터 돈을 받고, 독자나 시청자에게는 돈을 받지 않는지 설명 가능; 신용카드 회사가 가맹점으로부터만 수수료를 받고 구매자에게는 받지 않는 것도 설명 가능
			\end{itemize}
		\item 수요의 가격 탄력성: 가격 1퍼센트 상승으로 플랫폼이 몇 퍼센트의 소비자를 잃을 것인가를 보여줌 $\rightarrow$ 가격 결정의 기준
			\begin{itemize}
			\item 낮은 가격 탄력성: 가격을 높여도 수요 감소가 작으므로 수입이 증가할 수 있음
			\item 하지만, 이론적인 가능성일 뿐 $\rightarrow$ 소비자는 가격 상승에 대해 경쟁 플랫폼으로 옮길 수 있음
			\item $\rightarrow$ 신용카드 회사가 가맹점마다 수수료를 다르게 받는 이유를 설명 가능
			\end{itemize}	
		\end{itemize}
	\end{itemize}
\item 네트워크 이론(network theory)
	\begin{itemize}
	\item 네트워크와 네트워크 효과
		\begin{itemize}
		\item 네트워크: 노드(node)와 노드를 연결하는 링크(link)
		\item 네트워크 효과: 어떤 재화나 서비스를 사용하는 사람이 많아지면 많아질 수록 그 재화나 서비스의 가치가 더 커짐
		\item $\rightarrow$ 임계 질량(critical mass)과 승자 독식(winner takes all)
		\end{itemize}	
	\item 닭이 먼저일까 달걀이 먼저일까?
		\begin{itemize}
		\item 플랫폼은 B면의 사용자가 있기 전에 A면의 사용자 수가 충분하도록 투자할 필요가 있을 수 있음
		\item 게임기 제작사, 게임 사용자, 게임 개발사
			\begin{itemize}
			\item A 게임기의 사용자가 많지 않다면, 개발사는 A 게임기에서 작동하는 게임을 개발하지 않을 유인이 큼
			\item A 게임기 개발사는 게임 판매로 개발사가 얻는 수익을 시장 평균보다 더 높게 제시할 수 있음
			\item 동시에 A 게임기 개발사는 시장 평균 가격보다 낮은 가격으로 게임기를 판매하여 더 많은 소비자를 유인
			\item $\rightarrow$ 만약 A 게임기 개발사가 시장의 후발 주자라면, 성공적인 시장 진입을 위해 손해를 보면서 게임기를 판매할 수도 있음
			\end{itemize}
		\end{itemize}
	\item 호환성
		\begin{itemize}
		\item 소비자는 플랫폼을 선택할 수 있음 
		\item 플랫폼은 호환성을 위해 협조해야할까?
			\begin{itemize}
			\item 이동 통신에 대해서는 국가가 호환성을 강제
			\item 게임기, 신용카드 등은 강제가 아님
			\end{itemize}
		\item 플랫폼의 어떤 면(side)에서는 사용자가 다수의 플랫폼(multihoming) 또는 하나의 플랫폼(single homing)을 선택할 수 있음
			\begin{itemize}
			\item 여러 신용카드를 받는 가맹점 vs. 하나의 신용카드만 받는 가맹점
			\end{itemize}
		\end{itemize}
	\item 기술 개방
		\begin{itemize}
		\item 플랫폼은 여러 면의 경제 주체를 유인하기 위해 기술을 개방할 수도 개방하지 않을 수도 있음
		\item 모바일 기기 운영체제: 애플(Apple iOS) vs. 구글 안드로이드(Google Android)
		\end{itemize}
	\end{itemize}
\end{itemize}

\section{플랫폼 경제와 정책}
\begin{itemize}
\item 규제 
	\begin{itemize}
	\item 플랫폼에 대한 규제
		\begin{itemize}
		\item 과세: 국경을 초월한 기업 활동을 통한 수익을 어떻게 과세할 것인가?
		\item 반독점: 앞에서 본 것처럼 플랫폼은 양면에 비대칭적 가격 설정이 유리 $\rightarrow$ 전통적인 반독점 이론에서는 신규 기업의 시장 진입을 막는 약탈적 가격일 수 있음
			\begin{itemize}
			\item 하지만, 신규 진입 기업도 같은 전략을 사용할 것
			\item 과연 불공정 경쟁일까?
			\end{itemize}
		\end{itemize}
	\item 규제자로서의 플랫폼 $\rightarrow$ 플랫폼이 자선 사업가여서가 아니라 그렇게 하는 것이 유리하기 때문
		\begin{itemize}
		\item 판매자 간의 경쟁 촉진
		\item 가격 규제
		\item 품질 관리
		\item 정보 제공
		\end{itemize}
	\end{itemize}
\item 노동
	\begin{itemize}
	\item 주문형 노동
		\begin{itemize}
		\item 각종 대행(음식 배달, 쇼핑, 운전 등), 제시된 과업 수행 등
		\end{itemize}	
	\item 다수의 문제점 노출
		\begin{itemize}
		\item 불안정한 법적 지위와 낮은 소득
		\item 장시간 노동과 위험 감수
		\item 사회안전망으로부터의 사각 지대
		\end{itemize}
	\end{itemize}
\end{itemize}

\pagebreak

\section*{정리하기}
\begin{enumerate}
\item 경제학에서 정의하는 플랫폼은 "서로 다른 종류의 사용자가 상호 활동할 수 있도록 만들어주는 것"이다.
\item 플랫폼은 새로운 현상이 아니며, 쇼핑몰, 신용카드, (광고를 전달한다는 측면에서) 텔레비전/신문/라디오 등은 플랫폼의 속성을 갖고 있다.
\item 인터넷은 소비자가 확인할 수 있는 상품과 서비스의 목록을 크게 늘렸고, 이는 오히려 소비자가 원하는 정보를 찾고 평가해야하는 비용이 높아지는 상황을 초래했다. 따라서 플랫폼이 서로 원하는 거래 상대방을 찾아주는 것이 효과적일 수 있게 되었다.
\item 매칭 플랫폼은 플랫폼 사용자가 원하는 것을 추천, 제공하는 것이 일반적이다. 검색 엔진이 대표적인 서비스이다. 재화나 서비스의 판매 또는 이의 연결, 콘텐츠의 생산이나 공유 또는 광고 등을 통해 수익을 창출한다.
\item 기술 플랫폼은 플랫폼 사용자 간 상호 작용을 가능하게 하는 기술적 기반을 제공한다. 컴퓨터나 모바일 기기 등의 운영체제가 대표적이다. 재화나 서비스의 직접 판매 또는 대여, 보완재나 서비스의 개발을 통해 수익을 창출하는 것이 일반적이다. 
\end{enumerate}


