\chapter{게임: 닌텐도, 마이크로소프트, 소니}\label{cha:videogame}

\section*{학습개요}
게임 산업의 주요 특징을 살펴보고, 양면시장 이론, 네트워크 효과, 시소 원칙 등 앞에서 배운 개념으로 게임 산업을 경제학적으로 분석한다.


\section*{학습목표}
\begin{enumerate}
\item 게임 산업의 주요 특징을 이해한다.
\item 양면 시장 이론을 이용해 게임 산업의 특징을 설명할 수 있다.
\item 게임 산업 내 플랫폼 기업의 가격 전략을 설명할 수 있다.
\item 2020년대 초반 게임 산업의 변화를 설명하고, 학습한 내용으로 미래 변화를 전망해본다.
\end{enumerate}

\section*{주요 용어}
네트워크 효과, 호환성, 이식 비용

\pagebreak

\section{게임 산업}\label{sec:}
\begin{itemize}
\item 전세계 게임 산업 정의
	\begin{itemize}
	\item 비디오 게임: 소셜/캐주얼 게임(웹/브라우저 기반), 콘솔 게임(온라인/부분 유료 포함), PC 게임(온라인/부분 유료 포함) 등
	\item e스포츠: 티켓 판매 수익, 스폰서 십, 부가/이벤트 상품 수익 등
	\item 가상현실 게임: VR 게임 결제, 가입형 서비스, 디지털 결제, 오프라인 구매 등
	\end{itemize}
\item 전세계 게임 산업 규모
	\begin{itemize}
	\item 전세계적으로 보면 게임은 영화와 음악 산업을 합친 것보다 더 큰 규모로 성장 중
		
		\begin{table}[htp]
		\begin{center}
		\begin{threeparttable}
		\caption{세계 콘텐츠시장 규모 및 전망, 2017-2024}\label{tab:worldcontentsmarket}
		\begin{tabularx}{\textwidth}{lrrrrrrrrrr}
		\toprule
		& 2015	& 2016	& 2017	& 2018	& 2019	& 2020	& 2021	& 2022	& 2023	& 2024 \\
		\midrule
		출판&	2918	& 2893	& 2839	& 2798	& 2764	& 2482	& 2530	& 2531	& 2512 & 	2494 \\
		음악&	463 & 	486 & 	514	 & 543	& 580 & 	408 & 	519 & 	648 & 	671 & 	689 \\
		게임&	827	& 956 & 	1087	& 1201	& 1317	& 1429	& 1548	& 1638	& 1726	& 1815 \\
		영화&	387	& 401& 	419& 	435 & 	451& 	155& 	276& 	373& 	387& 	399 \\
		방송&	4609	& 4826	& 4826 & 	4895	& 4906	& 4622	& 4848 & 	5024	& 5124	& 5263 \\
		광고&	4708	& 5281	& 5281	& 5696 & 	6054 & 	5434	 & 5802	& 6199	& 6422	& 6635 \\
		\bottomrule
		\end{tabularx}
		\begin{tablenotes}
		\small
		\item 단위: 억 달러
		\item 출처: \cite{PWC:2020tx}, \cite{hangugkontencheujinheung-won:2020tl} p. 11 재인용.
		\end{tablenotes}
		\end{threeparttable}
		\end{center}
		\end{table}%
		
	\item 미국, 중국, 일본은 전 세계 게임 산업의 절반 가량을 차지
	\item 하지만 부문별 규모에서는 국가별 특징이 있음
		\begin{itemize}
		\item 주요 3개국 모두 스마트폰  등에서 하는 소셜/캐주얼 게임 시장이 빠르게 성장하는 것이 확인됨
		\item 하지만 미국은 전통적으로 콘솔 게임 중심으로 발달
		\item 일본은 콘솔 게임 중심으로 발달했지만, PC 게임 시장도 큰 규모를 차지
		\item 중국은 소셜/캐주얼 게임 중심으로 발달하고 PC 게임 시장 규모도 큰 반면, 콘솔 게임 시장은 미미
		\end{itemize}
	\end{itemize}	
	
		\begin{sidewaystable}
		\begin{center}
		\begin{threeparttable}
		\caption{주요 국가별 게임 시장 규모 및 전망, 2015--2024}\label{tab:gamemarket}
		\begin{tabularx}{\textwidth}{llrrrrrrrrrr}
		\toprule
		& & 2015	& 2016	& 2017	& 2018	& 2019	& 2020	& 2021	& 2022	& 2023	& 2024 \\
		\midrule
		미국 & 게임 광고 & 1135	& 1237	& 1331	& 1433	& 1535	& 1574	& 1613	& 1685	& 1755	& 1821\\
		&  소셜/캐주얼 게임 & 6332	& 7452	& 8850	& 10444	& 11668	& 12575 & 	13442	& 14284 & 	14968 & 	15631 \\
		& 콘솔 게임 & 7949 &	8219&	8488 &	8824	 & 9143 &	9451	& 9975	& 10303	 & 10327	 & 10470 \\
		& PC 게임 & 3406	& 3671 & 	3926	& 4158	& 4644	& 5245 & 	6070	& 6837	& 7742	& 8758 \\
		& e스포츠 & 88 &	127 	& 173 & 	218	 & 279	& 314	 & 379	& 425	 & 469	& 507 \\
		& 소계 & 18897	& 20842	& 22938	& 25314	& 27581	& 29547	& 31954	& 34074	& 35891	& 37939 \\
		\midrule
		중국 & 게임 광고 & 218	&271	&306&	338&	366	&383&	401	&428	&455	&482 \\
		&  소셜/캐주얼 게임 & 9312	&13092	&16791	&18316&	20646&	22726	&24386	&25418	&26189	&26758 \\
		& 콘솔 게임 & 4	&25	&127&	245&	386&	543&	681&	804&	915	&998 \\
		& PC 게임 & 5585	&5694&	6256&	6706&	7045&	7534	&8164	&8740	&9448	&10276 \\
		& e스포츠 & 50&	73	&112	&181	&340	&408&	485&	551&	609&	668 \\
		& 소계 &15161	&19159	&23638	&25887	&28938	&31801	&34394	&36301	&38074&	39768 \\
		\midrule
		일본  & 게임 광고 & 275	&280	&313	&345&	375	&392	&410	&432	&454	&475 \\
		&  소셜/캐주얼 게임 & 6051&	7866	&9181	&10649	&11772	&12726	&13669	&14075	&14406&	14630 \\
		& 콘솔 게임 & 2841	&2977	&3113	&3228	&3305	&3306&	3354&	3388&	3428	&3478 \\
		& PC 게임 & 2325&	2532	&2791	&2968	&3118	&3307	&3563	&3679	&3785	&3879 \\
		& e스포츠 & 4&	7&	11	&15	&22&	29	&40	&49	&57	&67  \\
		& 소계 & 11494	&13672	&15440	&17261&	18667	&19861	&21164	&21780&	22323	&22773\\
		\midrule
		\bottomrule
		\end{tabularx}
		\begin{tablenotes}
		\small
		\item 단위: 백만 달러
		\item 출처: \cite{PWC:2020tx}, \cite{hangugkontencheujinheung-won:2020tl} p. 46, p.139, p. 214 재인용.
		\end{tablenotes}
		\end{threeparttable}
		\end{center}
		\end{sidewaystable}
	
\item 한국 게임 산업 정의와 규모
	\begin{itemize}
	\item 콘텐츠산업조사: 앞서의 보고서와는 다소 다른 기준으로 작성되므로 직접 비교는 어려움	
	\item 전세계 특성과 유사하게 우리나라 게임 산업의 매출은 음악과 영화를 합친 것보다 큰 규모
	\item 특기할만한 점은 다른 콘텐츠 산업보다 수출액이 높다는 것
	\item 세부적으로 보면 모바일 게임의 매출 비중이 높고 콘솔 게임의 매출 비중은 낮은 특성을 보임
	\end{itemize}
\end{itemize}

			\begin{table}[htp]
			\begin{center}
			\begin{threeparttable}
			\caption{한국 콘텐츠 산업(일부) 요약, 2019}\label{tab:koreancontentsmarket}
			\begin{tabularx}{\textwidth}{lrrrrrrr}
			\toprule
			& 사업체 수 & 종사자 수 & 매출액 & 부가가치액 & 수출액 & 수입액 \\
			\midrule
			출판 & 25,220 & 185,270 & 21,341,176 & 8,875,983 & 214,732 & 275,426 \\
			음악 & 34,145 &  77,149 & 6,811,818 & 2,173,092 & 756,198 & 13,766 \\
			영화 & 1,223 & 32,566 & 6,432,393 & 1,354,550 & 37,877 & 38,432 \\
			게임 & 13,387 & 89,157 & 15,575,034 & 6,753,335 & 6,657,777 & 298,129 \\
			방송 & 1,062 & 51,006 & 20,843,012 & 6,816,636 & 474,359 & 95,812 \\
			광고 &  7,323 & 73,520 & 18,133,845 & 5,630,559 & 139,083 & 276,034 \\
			\bottomrule
			\end{tabularx}
			\begin{tablenotes}
			\small
			\item 단위: 개, 명, 백만원, 천 달러.
			\item 출처: \cite{munhwacheyuggwangwangbu:2021wo}, p. 3.
			\item 통계 중 만화, 애니메이션, 캐릭터, 지식정보, 콘텐츠 솔루션 산업은 제외
			\end{tablenotes}
			\end{threeparttable}
			\end{center}
			\end{table}%	
	
	
			\begin{table}[htp]
			\begin{center}
			\begin{threeparttable}
			\caption{한국 게임 산업 총괄}\label{tab:koreagameindustry}
			\begin{tabularx}{\textwidth}{lrrrrrrr}
			\toprule
			& 사업체 수 & 종사자 수 & 매출액 & 부가가치액 & 수출액 & 수입액 \\
			\midrule
			2015 & 13,844 & 80,388 & 10,722,284 & 5,047,597 & 3,214,627 & 177,492\\
			2016  & 12,363 & 73,993 & 10,894,508 & 4,848,056 & 3,277,346 & 147,362 \\
			2017  & 12,937 & 81,932 & 13,142,272& 5,795,742& 5,922,998 & 262,911\\
			2018  & 13,357 & 85,492 & 14,290,224 & 6,179,093 & 6,411,491 & 305,781\\
			2019  & 13,387 & 89,157 & 15,575,034 & 6,753,335 & 6,657,777 & 298,129 \\
			\bottomrule
			\end{tabularx}
			\begin{tablenotes}
			\small
			\item 단위: 개, 명, 백만원, 천 달러.
			\item 출처: \cite{munhwacheyuggwangwangbu:2021wo}, p. 124.
			\end{tablenotes}
			\end{threeparttable}
			\end{center}
			\end{table}%
			
				\begin{table}[htp]
				\begin{center}
				\begin{threeparttable}
				\caption{한국 게임산업 업종별 연도별 매출액 현황, 2015--2019}\label{tab:koreagameindustrybygenres}
				\begin{tabularx}{\textwidth}{lrrrrr}
				\toprule
				중분류 (소분류) & 2015 & 2016 & 2017 & 2018 & 2019 \\
				\midrule
				게임 제작 및 배급업 & 9,016,139 & 9,352,741 & 11,304,298 & 12,393,304 & 13,463,890 \\
				(PC 게임) & 5,318,214 & 4,678,630 & 4,540,886 & 5,023,616 & 4,805,846 \\
				(모바일 게임) & 3,484,406 & 4,330,096 & 6,210,237 & 6,655,757 & 7,739,931 \\
				(콘솔 게임) & 166,091 & 262,656 & 373,376 & 528,495 & 694,556 \\
				(아케이드 게임) & 47,428 & 81,359 & 179,800 & 185,436 & 223,557 \\
				\midrule
				게임 유통업 & 1,706,145 & 1,541,767 & 1,837,974 & 1,896,920 & 2,111,144 \\
				(컴퓨터 게임방 운영업) & 1,660,400 & 1,466,815 & 1,760,018 & 1,828,291 & 2,040,881 \\
				(전자 게임장 운영업) & 45,745 & 74,952 & 77,956 & 68,629 & 70,263 \\
				\midrule
				합계 & 10,722,284 & 10,894,508 & 13,142,272 & 14,290,224 & 15,575,034 \\
				\bottomrule
				\end{tabularx}
				\begin{tablenotes}
				\small
				\item 단위: 백만 원
				\item 출처: \cite{munhwacheyuggwangwangbu:2018wv}, \cite{munhwacheyuggwangwangbu:2021wo}.
				\end{tablenotes}
				\end{threeparttable}
				\end{center}
				\end{table}%		


\section{게임 산업의 경제학}
\subsection{게임 산업의 특성}
\begin{itemize}
\item 플랫폼
	\begin{itemize}
	\item 콘솔  \citep{Lee:2012ud}
		\begin{itemize}
		\item 보통 게임기라고 부르는 작은 박스
		\item 초창기에는 모니터 등이 있는 일체형이었지만, 현재는 휴대용 기기에만 모니터가 있고, 가정용 거치형은 모니터나 텔레비전을 연결해야 함
		\item 2021년 현재, 주요 제조사는 닌텐도(Nintendo), 소니(Sony), 마이크로소프트(Microsoft)의 3개 회사\footnote{최초의 카트리지 식 콘솔 게임기는 1976년 출시된 페어차일드 카메라(Fairchild Camera)의 채널 에프(Channel F). 하지만, 성공적으로 콘솔 게임기 시장을 연 것은 그 이듬해 출시된 아타리 VCS (Video Computer System). 닌텐도의 게임기 NES(Nintendo Entertainment System)는 1983년 일본에서 처음 출시되었고 2년 뒤 미국에 상륙. 소니의 플레이스테이션(Playstation)은 1995년 처음 발매. 마이크로소프트의 엑스박스(X-Box)는 2001년 처음 출시.}
		\end{itemize}
	\item 개인용 컴퓨터와 스마트폰
		\begin{itemize}
		\item 개인용 컴퓨터(PC: Personal Computer)와 스마트폰 같은 다기능 장치에서도 게임을 할 수 있음
		\end{itemize}
	\item 엔진
		\begin{itemize}
		\item 게임 개발에 필요한 소프트웨어 프레임워크로 보통 라이브러리와 지원 프로그램이 포함되어 있음
		\item 3D 또는  2D, 물리 엔진, 충격 반응, 소리, 애니메이션, 네트워킹, 메모리 관리, 쓰레딩 등 포함됨
		\item 게임 개발의 효율성 및 다양한 플랫폼으로의 이식(porting)을 지원함
		\item 블렌더(Blender), 언리얼(Unreal), 유니티(Unity) 등이 대표적
		\end{itemize}
	\end{itemize}
\item 플랫폼 사용자
	\begin{itemize}
	\item 게임 개발자 또는 개발 스튜디오: 게임을 만드는 창의적 활동
	\item 게임 퍼블리셔: 게임의 광고, 마케팅, 유통
	\item 게임 이용자
		\begin{itemize}
		\item	전세계: 2020년 현재 26억 9천만명으로 추산, 2023년까지 30억 7천만명으로 늘어날 것으로 예상\footnote{\url{https://www.statista.com/statistics/748044/number-video-gamers-world/}}
		\item 한국: 조사대상 3,084명 중 게임 이용자 2,174명(70.5\%) \citep{hangugkontencheujinheung-won:2020aa}
		\end{itemize}
	\end{itemize}
	
			\begin{table}[htp]
			\begin{center}
			\begin{threeparttable}
			\caption{한국 연령별/성별 게임 이용률, 2020}\label{tab:gameexperiencebygenderandage}
			\begin{tabularx}{\textwidth}{cccccccc}
			\toprule
			& 전체 & 10대 & 20대 & 30대 & 40대 & 50대 & 60 -- 65세 \\
			\midrule
			남성 & 73.6\% & 92.8\% & 93.2\% & 79.6\% & 78.5\% & 54.8\% & 37.0\% \\
			여성 & 67.3\% & 90.0\% & 76.2\% & 68.0\% & 74.5\% & 58.7\% & 33.1\% \\
			\midrule
			전체 & 70.5\% & 91.5\% & 85.1\% & 74.0\% & 76.6\% & 56.8\% & 35.0\% \\
			\bottomrule
			\end{tabularx}
			\begin{tablenotes}
			\small
			\item 응답자 3,084명, 단위: \% 
			\item 신뢰수준 95\% $\pm 1.8\% p$
			\item 출처:  \citep{hangugkontencheujinheung-won:2020aa}, p. 11.
			\end{tablenotes}
			\end{threeparttable}
			\end{center}
			\end{table}%
	
	
\item 상품 또는 서비스의 특징
	\begin{itemize}
	\item 수명이 짧음
		\begin{itemize}
		\item 콘솔: 하드웨어 성능의 발달 등에 따라 주기적으로 신제품을 출시(5-7년)		
		\item 게임: 게임의 속성에 따라 다르지만, 결말이 있거나, 흥미가 지속되지 않는 경우 게임을 계속하지 않음, 통상 출시 후 6개월 이내에 사용이 급감
			\begin{itemize}
			\item 게임 개발사는 지속적으로 새로운 게임을 출시하거나
			\item 추가 패키지를 출시
			\end{itemize}
		\end{itemize}
	\item 네트워크 효과
		\begin{itemize}
		\item 콘솔 게임기 등의 하드웨어 그 자체를 구매하려 하기 보다 소비자는 자신이 하고 싶은 게임이 있기 때문에 구매
		\item 게임 퍼블리셔는 이미 사용자 수가 많은 콘솔 게임기나 하드웨어에서 게임을 출시
		\item 보통 콘솔 게임기 제조사는 하드웨어 판매에서는 손해를 보고, 소프트웨어 판매나 로열티 수입에서 이득을 얻음
		\end{itemize}
	\item 호환성
		\begin{itemize}
		\item 이식(porting)
			\begin{itemize}
			\item 보통 특정 콘솔 또는 운영체제(윈도우즈 또는 맥/리눅스, 아이오에스 또는 안드로이드)에서 구동하는 게임은 다른 콘솔이나 운영체제에서는 구동하지 않았음
			\item 다른 콘솔이나 운영체제에서 구동하려면 게임을 이식(porting)하거나, 새로운 버전을 만들어야 함
			\item $\rightarrow$ 추가 개발 및 프로그래밍 시간 $+$ 재정 비용
			\end{itemize}
		\item 배타성(exclusivity)
			\begin{itemize}
			\item 1985년 닌텐도는 미국에 게임기를 출시하며 게임 개발사에 최소 2년간 자사의 게임기에만 게임을 공급하고, 고정된 가격에 게임을 판매하고 게임 판매에 대해 20\%의 로열티를 요구\footnote{1987년, 미국 연방 거래 위원회(FTC: Federal Trade Commission) 조사 이후, 고정된 가격으로의 판매하는 것은 금지되고, 닌텐도의 보안 칩이 걸린 카트리지가 탑재된 방식으로 게임이 판매되며 로열티를 징수하는 것은 유지됨}
			\item 콘솔 게임기 제조사가 직접 게임 개발을 해, 자사의 게임기에만 독점 공급하기도 함
			\end{itemize}
		\end{itemize}
	\end{itemize}		
\end{itemize}


\subsection{게임의 수요와 공급}
\begin{itemize}
\item 수요 $\rightarrow$ 소비자가 재미있어 하는 게임 또는 좋아하는 콘솔  $\rightarrow$ 재미있다 / 좋아하는 것의 의미는?
	\begin{itemize}
	\item 게임 그 자체의 흥미 또는 콘솔의 성능
	\item 그 게임을 하는 다른 사용자가 많아서
	\item $\rightarrow$ 더 많은 실증 연구가 필요
	\end{itemize}
\item 공급
	\begin{itemize}
	\item 게임 제작사가 게임 콘솔 제작사라면, 자사의 콘솔에 독점 공급하고자 할 것
	\item 게임 제작사가 콘솔 제작과 독립되어 있다면, 몇 개의 콘솔에서 게임을 개발할 것인지 결정해야 함 $\rightarrow$ 이식 비용
	\item 게임 개발 비용 $\rightarrow$ 직접 개발할 것인가 외주 개발할 것인가 $\rightarrow$ 더 많은 연구가 필요
	\end{itemize}
\end{itemize}

\subsection{플랫폼 경쟁}
\begin{itemize}
\item 플랫폼 가격 전략
	\begin{itemize}
	\item 콘솔 게임기 제작사와 게임 사용자
		\begin{itemize}
		\item 한계 비용 이하로 게임기를 판매 $\rightarrow$ 시장이 확대됨에 따라, 규모의 경제가 나타나면 게임기 생산 비용이 하락 $\rightarrow$ 장기적으로 비용이 가격보다 더 빠르게 하락하여 이윤 발생
		\end{itemize}
	\item 콘솔 게임기 제작사와 게임 개발사
		\begin{itemize}
		\item 게임 개발사로부터 로열티 징수
		\item 다수의 게임 콘솔 게임기 제작사가 경쟁하는 경우, 로열티를 낮추는 경쟁 발생할 수 있음 $\rightarrow$ 특히 시장에 진입하려는 기업이 선택할 수 있는 전략 %소니 플레이스테이션
		\end{itemize}
	\end{itemize}
\item 호환성과 이식 비용(porting costs)
	\begin{itemize}
	\item 콘솔 게임기 제작사와 게임 개발사
		\begin{itemize}
		\item 이식 비용을 낮추어 호환성을 높이는 것은, 시장에 신규 진입하려는 기업이 선택할 수 있는 전략
			\begin{itemize}
			\item 소니가 플레이스테이션을 출시하면서 개발 도구와 소프트웨어 라이브러리를 개발자에게 제공하고 닌텐도의 카트리지 방식보다 저렴한 씨디(CD)기반의 개발을 제공
			\item 마이크로소프트가 엑스박스를 출시하면서 다이렉트 엑스 기반의 게임 개발 도구를 공개함으로써 개인용 컴퓨터 기반의 게임을 콘솔로 이식하기 쉽도록 함
			\end{itemize}
		\item 또, 같은 회사의 게임기라고 하더라도 차세대 게임기를 출시하면서 구세대 게임기와의 호환성 정도를 고려할 수 있음
		\end{itemize}
	\end{itemize}
\item 독점 출시
	\begin{itemize}
	\item 게임 개발사와 계약을 맺거나 또는 콘솔 게임기 제작사가 직접 개발하거나 
	\item 게임 개발 비용은 상승하지만 이식 비용이 하락한다면, 게임 개발사는 다수의 플랫폼에서 게임을 구동하도록 하는 것이 유리
		\begin{itemize}
		\item 독점 출시 계약을 맺더라도 단기 계약(예) 6개월)으로 체결
		\end{itemize}
	\item 게임기 제작사가 직접 개발을 하는 경우, 수직 통합(vertical integration)으로 이해할 수 있음
	\item 소비자에게는 이득일 수도 손해일 수도
		\begin{itemize}
		\item 경쟁자의 시장 진입을 막을 수 있음 $\rightarrow$ 네트워크 효과가 있는 경우 그 효과가 더 커짐 $\rightarrow$ 소비자 선택을 좁히게 됨
		\item 하지만 안정적인 개발과 투자가 약속됨 
			\begin{itemize}
			\item 게임기 제작사가 게임 개발도 하는 경우, 시장 진입에 유리할 수 있음
			\item $\rightarrow$ 플랫폼 경쟁을 촉발할 수 있음
			\end{itemize}
		\end{itemize}
	\end{itemize}
\end{itemize}

\section{게임 산업의 변화}
\begin{itemize}
\item 구독형 상품과 클라우드 \citep{Plante:2021uu, Lai:2019wk}
	\begin{itemize}
	\item 애플 아케이드(Apple Arcade)와 엑스박스 게임 패스(Xbox Game Pass)
		\begin{itemize}
		\item 월 정액 요금을 지불하고, 제공된 목록에서 자유롭게 게임을 선택
		\item $\rightarrow$ 게임 사용자는 하나의 플랫폼을 사용, 게임 개발사는 다수의 플랫폼에서 개발하게 될 것
		\end{itemize}
	\item 구글 스태디아(Google Stadia)
		\begin{itemize}
		\item 콘솔이나 PC 없이 조작 장치와 네트워크 연결로 클라우드에서 게임을 할 수 있음
		\end{itemize}
	\item 이전에 비해 비용을 낮추므로 소비자를 늘릴 수 있음
	\item $\rightarrow$ 게임 사용자와 게임 개발사 모두 하나의 플랫폼만 사용하게 될 것
	\end{itemize}
\item 교차 플랫폼(cross-platform) \citep{Lai:2019wk}
	\begin{itemize}
	\item 전통적으로 콘솔, PC, 모바일 중 어느 하나로만 게임이 가능했음 $\rightarrow$ 소비자는 다수 플랫폼을 사용(multi-homing)
		\begin{itemize}
		\item 개인용 컴퓨터 등에 비해 게임 하드웨어의 표준화 요구는 낮았음
		\item 여러 대의 컴퓨터에서 문서 작업을 하는 것이 아니라 하나의 게임기에서 하나의 게임을 하는 개념에 가까웠음
		\item 하지만, 네트워크의 발달로 협력형 게임이 인기를 얻게됨에 따라 플랫폼 간 연계성이 중요해짐
		\end{itemize}
	\item 하지만 콘솔과 PC 그리고 서로 다른 콘솔 간의 연결이 가능해짐
		\begin{itemize}
		\item[예)] 포트나이트(Fortnite) 
		\end{itemize}
	\item 친구와 게임을 같이 하기 위해 동일한 콘솔을 구입하지 않아도 됨 $\rightarrow$ 진입 비용의 하락 $\rightarrow$ 소비자는 하나의 플랫폼을 사용하지만, 개발사는 다수의 플랫폼을 사용하게 될 것
	\end{itemize}
\item 확률형 아이템을 도박으로 규정하여 규제하려는 움직임 있음(영국 등, \cite{Davies:2020wn}) $\rightarrow$ 수익 모형의 변화
\item 코로나19 대유행 이후, 게임 만을 즐기는 것이 아니라, 사교의 공간(`동물의 숲'), 음악 공연장(`포트나이트') 등으로 그 기능을 강화 \citep{Slotkin:2020tq}	
\end{itemize}

\pagebreak

\section*{정리하기}
\begin{enumerate}
\item 게임 산업은 전세계나 우리나라에서나 영화와 음악 산업을 합친 것보다 규모가 더 크고, 다른 콘텐츠 산업보다 높은 성장률을 보일 것으로 예측된다.
\item 게임 산업에서의 플랫폼은 콘솔 게임기 제작사, 개인용 컴퓨터와 스마트폰 및 운영체제 제작사, 그리고 최근에 주목받는 것으로 소프트웨어 프레임워크 개발사가 있다.
\item 플랫폼을 사용하는 한 면에는 게임을 직접 만드는 개발자(또는 스튜디오)와 광고, 마케팅, 유통 등을 맡는 퍼블리셔가 있고, 다른 한 면에는 게임 이용자가 있다.
\item 상품이나 서비스로서 게임기와 게임의 특징은 수명이 짧은 점, 네트워크 효과가 있는 점, 호환성을 고려해야 한다는 점을 꼽을 수 있다.
\item 게임의 수요와 공급에 대해서는 더 많은 실증 연구가 필요하다.
\item 콘솔 게임 산업에서 콘솔 게임기 제조사는 플랫폼으로서 시소 원칙에 따라 가격 전략을 구사하는데, 게임 사용자에게 한계 비용보다 낮은 가격으로 콘솔 게임기를 판매하고, 게임 개발자에게 로열티를 받아 이득을 얻는 것이 일반적이다.
\item 이식 비용은 호환성 또는 다수 플랫폼 사용에 영향을 줄 수 있으며, 후발 게임기 제조사는 이식 비용을 낮추어 많은 게임 개발사의 개발 참여를 유도할 유인이 있다.
\item 어떤 게임이 특정 콘솔 게임기에서 독점적으로 작동한다면, 네트워크 효과가 작동하여 경쟁자의 진입을 막는다는 면에서는 소비자에게 손해가 될 수 있지만, 안정적인 개발과 투자가 보장되어 시장 진입을 촉진할 수도 있다는 점에서 소비자에게 이득이 될 수도 있다.
\item 2020년대 초반 현재, 다수의 게임을 월 정액으로 이용하는 구독 상품, 콘솔 등의 기기 없이 클라우드에 접속하는 게임, 교차 플랫폼 게임 등 다양한 변화가 나타나고 있다.
\end{enumerate}


%\cite{Maruyamaa:2011aa,Hart:2017aa,Marchand:2013uv}