\chapter{미디어: 애플과 넷플릭스}\label{cha:media}

\section*{학습개요}
음악 및 영상 산업의 특징과 디지털 및 플랫폼으로의 전환이 가져온 효과를 학습한다.

\section*{학습목표}
\begin{enumerate}
\item 온라인 미디어 산업의 특징을 설명할 수 있다.
\item 묶음 판매가 온라인 미디어 산업에 가져온 효과를 설명할 수 있다.
\item 스트리밍 음악(및 영상) 서비스의 특징을 설명할 수 있다.
\item 영상의 디지털 배급이 가져온 효과를 설명할 수 있다.
\item 미디어 플랫폼 관련 이슈를 설명할 수 있다.
\end{enumerate}

\section*{주요 용어}
음악 스트리밍, OTT, 묶음 판매, 가격 차별, 망 중립성 

\pagebreak

\section{미디어 산업과 특징}\label{sec:}
\begin{itemize}
\item 전세계 영화, 음악, 방송 산업 규모: 표 \ref{tab:worldcontentsmarket}
\item 미국, 중국, 일본 미디어 산업 시장 규모: 표 \ref{tab:usmediamarket} -- 표 \ref{tab:jpmediamarket}
	\begin{itemize}
	\item 산업별로 보면, 음악 시장은
		\begin{itemize}
		\item 미국은 공연, 중국은 스트리밍, 일본은 실물음반 매출 비중이 높았음
		\item 코로나19 감염증 확산으로 공연 시장은 크게 위축되었으나, 2022년 이후 회복할 것으로 전망
		\item 3국 모두에서 디지털 음원 매출은 증가하지만, 세부적으로는 음악 스트리밍 매출이 증가하고, 다운로드는 감소할 것으로 예상되고 있음
		\item 미국의 경우 2019년 3월 기준, 음악 스트리밍 시장 사업자는 애플 뮤직(Apple Music, 445만명), 스포티파이(Spotify, 442만명), 판도라 라디오(Pandora Radio, 314만 7천명)가 대표적
		\item 중국의 경우 2019년 12월 기준, QQ뮤직(QQ音乐, 3억 1,644만명), 쿠거우 뮤직(酷狗音乐, 2억 6,799만명), 쿠워뮤직(酷我音乐, 1억 1,658만명)이 대표적인 서비스임. 시장 규모에도 불구하고, 유료 가입자는 낮은 수준이지만, 매년 유료 지불 의사 비중과 금액이 상승하는 추세에 있음
		\item 일본의 경우, 2019년 이후 실물 음반 판매의 감소와 함께, 디지털 스트리밍 동시 발매 사례 등이 나타나고 있고, 주요 아티스트들이 스트리밍 서비스에 음원을 공개하기 시작했다. 스트리밍 서비스 앱 사용자 비중은 아마존 뮤직과 스포티파이가 21.4\%로 가장 높고, 애플 뮤직(15.7\%)이 그 다음임
		\end{itemize}

		\begin{sidewaystable}
		\begin{center}
		\begin{threeparttable}
		\caption{미국 미디어 시장 규모 및 전망, 2015--2024}\label{tab:usmediamarket}
		\begin{tabularx}{\textwidth}{llrrrrrrrrrr}
		\toprule
		& & 2015	& 2016	& 2017	& 2018	& 2019	& 2020	& 2021	& 2022	& 2023	& 2024 \\
		\midrule
		음악 & & 16,394 & 17,504 & 19,006 & 20,292 & 21,997 & 15,950 & 20,358 & 25,303 & 26,302 & 27,163 \\
		& (공연) & 9,284 & 9,593 & 10,012 & 10,446 & 10,885 & 3,912 & 7,269 & 11,190 & 11,540 & 11,901 \\
		& (다운로드) & 2,394 & 1,858 & 1,385 & 1,014 & 834 & 722 & 545 & 384 & 280 & 198 \\
		& (스트리밍) & 1,620 & 3,221 & 5,135 & 6,414 & 7,923 & 9,396 & 10,493 & 11,413 & 12,233 & 12,889 \\
		& (실물음반) & 1,973 & 1,624 & 1,532 & 1,155 & 1,148 & 931 & 901 & 969 & 883 & 796 \\
		영화 & & 11,171 & 11,423 & 11,221& 11,762 & 11,360 & 3,895 & 6,996 & 9,678 & 9,858 & 10,041 \\
		방송 & & 208,784 & 212,890 & 210,079 &208,588 & 205,918 & 190,017 & 199,170 & 204,195 & 205,390 & 209,579 \\
		& (유료 TV) & 101,052 & 100,824 & 98,548 & 94,340 & 89,575 & 81,556 & 81,266 & 80,750 & 79,711 & 79,581 \\ 
		& (OTT 비디오) & 8,132 & 10,399 & 12,546 & 14,940 & 18,175 & 22,593 & 25,382 & 26,609 & 29,107 & 30,929 \\
		\bottomrule
		\end{tabularx}
		\begin{tablenotes}
		\small
		\item 괄호 안의 소분류 업종은 일부만 제시
		\item 단위: 백만 달러
		\item 출처: \cite{PWC:2020tx}, \cite{hangugkontencheujinheung-won:2020tl} p. 25, p. 36, p.  57 재인용.
		\end{tablenotes}
		\end{threeparttable}
		\end{center}
		\end{sidewaystable}

		
	\item 영화 시장은
		\begin{itemize}
		\item 3국 모두, 코로나19 감염증 확산으로 급격한 위축을 경험
		\item 통상 영화관 $\rightarrow$ 가정 주문형 동영상 서비스 $\rightarrow$ OTT (Over-The-Top)로 단계적으로 공개되는 전략에서 벗어나, 영화관과 OTT의 동시 개봉 또는 OTT 단독 개봉도 나타남
		\item 미국의 경우, 영화관 관람 빈도와 PC, 스마트폰, 영상 스트리밍 기기, 게임 콘솔 등 영화를 볼 수 있는 기기를 보유 대수가 정비례하는 것으로 나타남 $\rightarrow$ IT 기기를 통한 시청이 영화관 시청을 대체하는 것이 아니라, 미디어 이용에 관심이 많은 이용자가 영화관 방문을 더 많이 하는 것으로 추측할 수 있음 
		\item 중국의 경우, 자국 영화의 매출 비중이 상승하는 추세에 있으며, 2019년 아이치이와 텐센트동영상의 회원 규모가 1억명을 넘어서는 등 인터넷동영상 서비스 가입자가 증가하는 추세에 있음
		\item 일본의 경우, 자국 애니메이션 매출이 높은 비중을 차지하는 특징을 보이고 있으며, 함께 노래를 부르며 영화를 보는 라이브 이벤트 상영 등의 기획으로 주 소비층인 애니메이션 팬덤을 겨냥한 전략이 시도되고 있음
		\end{itemize}	
		
		\begin{sidewaystable}
		\begin{center}
		\begin{threeparttable}
		\caption{중국 미디어 시장 규모 및 전망, 2015--2024}\label{tab:cnmediamarket}
		\begin{tabularx}{\textwidth}{llrrrrrrrrrr}
		\toprule
		& & 2015	& 2016	& 2017	& 2018	& 2019	& 2020	& 2021	& 2022	& 2023	& 2024 \\
		\midrule
		음악 & & 429 & 536 & 685 & 1,014 & 1,273 & 1,438 & 1,745 & 1,832 & 1,949 & 2,023 \\
		& (공연) &  196 & 211 & 226 & 243 & 258 & 106 & 190 & 230 & 241 & 247 \\
		& (다운로드) &  17 & 20 & 19 & 34 & 40 & 48 & 52 & 36 & 29 & 22\\
		& (스트리밍) & 98 & 185 & 240 & 517 & 759 & 1,076 & 1,331 & 1,404 & 1,554 & 1,666 \\
		& (실물음반) & 24 & 13 & 13 & 6 & 4 & 3 & 2 & 2 & 2 & 2 \\
		영화 & & 6,907 & 7,309 & 8,908 & 9,707 & 10,316 & 2,263 & 5,485 & 7,540 & 7,797 & 8,061 \\
		방송 & & 32,344 & 35,239 & 38,665 & 40,578 & 41,182 & 39,840 & 42,160 & 45,664 & 48,805 & 51,708 \\
		& (유료 TV) & 15,196 & 17,629 & 18,323 & 17,612 & 16,592 & 15,449 & 15,871 & 16,568 & 17,783 & 19,006 \\ 
		& (OTT 비디오) & 497 & 1,377 & 3,709 & 6,047 & 7,687 & 10,152 & 11,235 & 12,635 & 14,531 & 16,103 \\
		\bottomrule
		\end{tabularx}
		\begin{tablenotes}
		\small
		\item 괄호 안의 소분류 업종은 일부만 제시
		\item 단위: 백만 달러
		\item 출처: \cite{PWC:2020tx}, \cite{hangugkontencheujinheung-won:2020tl} p. 116, p. 130, p. 147  재인용.
		\end{tablenotes}
		\end{threeparttable}
		\end{center}
		\end{sidewaystable}
		
		
	\item 방송 시장은
		\begin{itemize}
		\item 크게 보아, 광고와 함께 무료로 제공되는 지상파, 서비스 이용료를 지불하는 다채널 방송, 주문형 비디오를 제공하는 유료 방송, 인터넷 기반의 구독형 비디오(OTT 비디오) 로 나눌 수 있음
		\item 미국의 경우, 다채널 방송과 유료 방송의 가입자는 감소하는 추세이며, OTT 비디오 시장의 가입자와 매출이 증가하고 있으나 시장 전체 규모를 키울 수 있는 정도로 성장하고 있지는 못하고 있음
			\begin{itemize}
			\item 2019년 기준(statista.com), 월 평균 사용자 넷플릭스 (4,655만명), 훌루 (Hulu, 2,648만명), 아마존 프라임 비디오 (Amazon Prime Video, 1,646만명)이 대표적
			\end{itemize}
		\item 중국의 경우, 방송 시장이 지속적으로 성장 중. 성장의 핵심은 유료 방송과 OTT 비디오로 예상됨
			\begin{itemize}
			\item 단, 주요 OTT 서비스인 아이치이(爱奇艺, 바이두(百度)), 요우쿠투도우(优酷土豆, 알리바바(阿里巴巴) 그룹), QQ비디오(텐센트) 등이 2020년 6월 기준 3억명이 넘는 월간 활성 사용자 수를 기록할 정도로 사용자를 확보하고 있지만, 광고 기반 무료 서비스로, 구독형 서비스에 비해 수익원이 안정적이지 않다는 한계가 있음 %腾讯视频
			\end{itemize}
		\item 일본은 공영방송과 유료 방송이 성장하지 않고 정체 상태를 보이는 선진국 방송 시장의 전형적인 모습을 보이고 있음. 단, OTT 비디오 시장은 2020년 이후 빠르게 성장할 것으로 예상되고 있음
			\begin{itemize}
			\item OTT 비디오 시장의 서비스 이용자 순위는 아마존 프라임 비디오(37\%), 넷플릭스(21\%), 훌루(Hulu, 11\%) 순
			\item 수익으로는 넷플릭스(13.8\%), 다즌(DAZN, 11.2\%), 아마존 프라임 비디오(10.9\%) 순
			\end{itemize}
		\end{itemize}	

		\begin{sidewaystable}
		\begin{center}
		\begin{threeparttable}
		\caption{일본 미디어 시장 규모 및 전망, 2015--2024}\label{tab:jpmediamarket}
		\begin{tabularx}{\textwidth}{llrrrrrrrrrr}
		\toprule
		& & 2015	& 2016	& 2017	& 2018	& 2019	& 2020	& 2021	& 2022	& 2023	& 2024 \\
		\midrule
		음악 & & 6,654 & 6,784 & 6,677 & 7,162 & 7,414 & 5,410 & 6,272 & 7,606 & 7,532 & 7,455 \\
		& (공연) & 2,363 & 2,393 & 2,424 & 2,710 & 2,907 & 1,406 & 2,256 & 3,259 & 3,288 & 3,330 \\
		& (다운로드) &  473 & 435 & 430 & 415 & 370 & 335 & 263 & 206 & 154 & 114 \\
		& (스트리밍) &  189 & 346 & 424 & 567 & 723 & 915 & 1,042 & 1,146 & 1,249 & 1,328 \\
		& (실물음반) &  3,357 & 3,350 & 3,147 & 3,228 & 3,174 & 2,544 & 2,491 & 2,764 & 2,613 & 2,457 \\
		영화 & & 1,990 & 2,159 & 2,095 & 2,044 & 2,341 & 1,019 & 1,742 & 2,174 & 2,201 & 2,228 \\
		방송 & &  28,531 & 30,365 & 30,946 & 31,674 & 31,524 & 30,332 & 31,690 & 32,293 & 32,470 & 32,745 \\
		& (공영 TV) & 5,704 & 6,003 & 6,293 & 6,246 & 6,209 & 6,161 & 6,116 & 6,075 & 6,039 & 6,010 \\
		& (유료 TV) & 3,954 & 4,756 & 4,759 & 4,775 & 4,809 & 4,581 & 4,674 & 4,693 & 4,707 & 4,727 \\ 
		& (OTT 비디오) &  1,234 & 1,606 & 1,931 & 2,289  & 2,602 & 3,341 & 3,666 & 4,102 & 4,551 & 4,985 \\
		\bottomrule
		\end{tabularx}
		\begin{tablenotes}
		\small
		\item 괄호 안의 소분류 업종은 일부만 제시
		\item 단위: 백만 달러
		\item 출처: \cite{PWC:2020tx}, \cite{hangugkontencheujinheung-won:2020tl} p. 203, p. 224, p. 233  재인용.
		\end{tablenotes}
		\end{threeparttable}
		\end{center}
		\end{sidewaystable}
		
		
	\end{itemize}

\item 한국 콘텐츠 시장: 표 \ref{tab:koreancontentsmarket}
	\begin{itemize}
	\item 한국 미디어 산업 시장 규모: 표 \ref{tab:krmediamarket}
	\item 음악 시장은 성장 중이며, 음악 제작과 인터넷/모바일 음원 유통의 증가가 두드러짐
	\item 영화 시장도 성장 중이지만, 코로나19 감염증 확산의 영향을 확인할 수 있는 데이터는 아직 부족
	\item 방송 시장도 성장 중이며, 지상파 TV와 종합 유선 방송의 매출은 감소하고 있지만, 방송채널사용사업, IPTV, 독립제작사의 매출은 증가 중
		\begin{itemize}
		\item 방송채널사용사업(program provider): 종합 유선 방송 사업자 또는 위성 방송 사업자와 특정 채널의 전부 또는 일부 시간에 대한 전용 사용 계약을 체결하여 그 채널을 사용하는 사업자. 2021년 6월 현재 461개 사업자 등록(승인) 중\footnote{과학기술정보통신부, 방송채널사용사업 승인 등록 현황, \url{https://www.msit.go.kr/publicinfo/detailList.do?sCode=user&mId=63&mPid=62&publictSeqNo=650&publictListSeqNo=16&formMode=L&referKey=650}}
		\end{itemize}

		\begin{sidewaystable}
		\begin{center}
		\begin{threeparttable}
		\caption{한국 미디어 시장 매출액, 2015--2019}\label{tab:krmediamarket}
		\begin{tabularx}{\textwidth}{llrrrrr}
		\toprule
		& & 2015	& 2016	& 2017	& 2018	& 2019 \\
		\midrule
		음악 & &  4,975,196 & 5,308,240 & 5,804,307 & 6,097,913 & 6,811,818 \\
		& (음악제작업)  & 964,125 & 1,062,210 & 1,173,207 & 1,347,418 & 1,859,987 \\
		& (인터넷/모바일) &  1,136,983 & 1,245,425 & 1,441,804 & 1,518,803 & 1,601,034\\
		& (공연 기획) & 778,825 & 864,217 & 944,134 &  981,066 & 1,037,056\\
		& (노래연습장) &   1,492,618 & 1,516,622 & 1,503,115 &  1,445,015 & 1,429,534\\
		영화 & & 5,112,219 & 5,256,081\tnote{a} & 5,494,670 & 5,889,832 & 6,432,393 \\
		& (기획 및 제작) & 814,475 & 778,494 & 722,207 & 740,082 & 781,161 \\ 
		& (영화 배급) & 910,741 & 913,265 & 937,504 & 890,581 & 945,624 \\
		& (극장 상영) & 2,335,365 & 2,442,068 & 2,608,179 & 2,851,750 & 3,125,480 \\
		& (온라인 배급) & 154,322 & 166,737 & 175,785 & 209,795 & 241,284 \\
		& (온라인 상영) & 261,636 & 313,701 & 385,658 & 456,003 & 516,184 \\
		방송 & &  16,462,982 & 17,331,138 & 18,043, 595 & 19,762,210 & 20,843,012 \\
		& (지상파 TV) &  3,724,472 & 3,612,780 & 3,320,939 & 3,451,432 & 3,215,076 \\
		& (종합유선방송) &  2,259,023 & 2,169,185 & 2,130,726 & 2,089,809 & 2,022,703 \\ 
		& (방송채널사용사업) &  6,222,446 & 6,380,071 & 6,639,622 & 6,840,197 & 7,091,841 \\
		& (IPTV) & 1,908,798 & 2,427,660 & 2,925,123 & 3,435,828 & 3,856,648 \\
		& (독립제작사) & 1,143,498 & 1,428,813 & 1,531,422 & 2,456,536 & 3,171,316 \\
		\bottomrule
		\end{tabularx}
		\begin{tablenotes}
		\small
		\item[a] 2015년까지는 상영대금(로열티) $+$ 프린트, 2016년부터 프린트 제외
		\item 괄호 안의 소분류 업종은 일부만 제시
		\item 단위: 백만원 
		\item 출처: \cite{munhwacheyuggwangwangbu:2018wv}, \cite{munhwacheyuggwangwangbu:2021wo}.
		\end{tablenotes}
		\end{threeparttable}
		\end{center}
		\end{sidewaystable}
		
		
	\end{itemize}

\item 온라인 미디어 산업의 특징 \citep{Sherman:2016ur}
	\begin{itemize}
	\item 효율적인 복사 및 공유
		\begin{itemize}
		\item 전통적인 음악/영화/방송 산업(문지기, gatekeeping model)에 비해 낮은 비용으로 전달이 가능
			\begin{itemize}
			\item $\{$배우, 작가, 감독 등$\}$ $\rightarrow$ 제작 / 배급 $\rightarrow$ 시청자 
			\item $\{$가수, 연주자, 작곡가, 작사가 등$\}$ $\rightarrow$ 제작 / 배급 $\rightarrow$ 소비자(시청자)
			\item 제작을 담당하는 음반사나 영화사가 가수나 작가 등과 계약을 체결(문지기 역할) $\rightarrow$ 숙련된 노동 인력과 장비를 이용하여 제작 $\rightarrow$ 라디오, 텔레비전, 극장 등으로 공개 $\rightarrow$ 물리 매체(음반, VHS, DVD, BD)  또는 2차 매체 (유료 케이블 채널 등) 판매
			\item 제작 과정이 디지털화 됨:  이전보다 저렴한 소프트웨어와 하드웨어로 제작 가능(스마트 폰 등을 이용한  촬영 등)
			\item 전달 과정이 디지털화 됨: 개인의 직접 배포가 가능해짐
			\end{itemize}
		\item 하지만, 저작권과 통신망 사용 문제가 제기됨
			\begin{itemize}
			\item 2000년대 초반, 냅스터(Napster)와 소리바다의 등장 $\rightarrow$ 음악 산업 급격한 매출 감소
			\item 엠피쓰리(mp3)의 유통이 음악 산업의 규모를 축소시켰는가를 추측할 수는 있어도 인과관계를 확인하기는 사실 어려운 문제
			\item 무료 다운로드 $\rightarrow$ 미리 듣기와 같은 역할 $\rightarrow$ 소비자는 자신의 취향에 맞는 음악을 듣고 구입을 결정할 수 있으므로, 소비가 늘어날 가능성이 있음
			\item 무료 다운로드의 데이터를 확인하기 어려움
			\item 무료 다운로드와 유료 다운로드의 데이터가 있더라도 인과 관계를 확인하기 어려움 $\rightarrow$  인기가 있는 음악, 영화, 드라마 등의 다운로드가 많을 것 $\rightarrow$ 무료 다운로드가 유료 다운로드를 추가적으로 얼마나 늘렸는지 확인하기 어려움
			\item 그럼에도 무료 다운로드가 음악 산업의 매출 감소에 큰 원인이라는 데 합의가 형성됨 \citep{Waldfogel:2017aa}
			\end{itemize}
		\end{itemize}		
	\item 제약의 감소
		\begin{itemize}
		\item 매장이나 대여점의 공간 한계를 넘어, 다수의 미디어 보유 가능
		\item $\rightarrow$ 다양성의 증가
		\end{itemize}
	\item 비용 효율적인 전달
		\begin{itemize}
		\item 높은 고정 비용 발생
			\begin{itemize}
			\item 플랫폼 사업자: 미디어 라이선스 계약, 서버 관리 및 트래픽 비용 등
			\item 사용자: 기기 구입 비용 등
			\end{itemize}
		\item 그러나 단위 당 비용은 상대적으로 낮음
			\begin{itemize}
			\item 사용자: 물리 매체에 비해 청취나 시청 1회당 비용이 낮을 수 있음
			\end{itemize}
		\end{itemize}
	\item 상호성
		\begin{itemize}
		\item 쌍방향 통신에 의한 데이터 축적
		\item 혁신적인 광고 시스템 또는 사용자 인터페이스
		\item 사용자 간 공유
		\end{itemize}
%	\item 휴대성: 휴대용 기기 또는 브라우저 기반 재생 장치
	\item 묶음 판매 및 효율적인 가격 차별
	\end{itemize}	
\item 묶음 판매(bundling)
	\begin{itemize}
	\item 전통적으로 음악을 듣는 것 $\rightarrow$ 음반을 사는 것(하나 이상의 곡이 한 장의 음반에 포함되어 있음)
		\begin{itemize}
		\item 음악과 유사하게 영화, 방송 프로그램을 보는 것 $\rightarrow$ 한 편의 영화, 방송 프로그램을 보는 것	
		\end{itemize}
	\item 음악, 영화, 방송 프로그램의 묶음
		\begin{itemize}
		\item 하나의 물리 매체(씨디 한 장, 디비디/블루레이 한 장 등)를 사는 것이 아니라
		\item 애플 뮤직(Apple Music), 스포티파이(Spotify), 멜론(Melon), 넷플릭스(Netflix), 웨이브(Wavve), 왓챠(Watcha) 등의 회원 가입
		\item  월 정액 요금을 내고, 각 서비스가 보유한 음악, 영화, 방송 프로그램을 무제한으로 이용
		\end{itemize}
	\end{itemize}
\item 온라인 미디어의 묶음 판매 
	\begin{itemize}
	\item 미디어의 디지털화 때문에 가능해짐
		\begin{itemize}
		\item 한 곡씩, 한 편씩 분리하고 다시 이를 하나로 묶는 비용이 낮아짐
		\end{itemize}	
	\item 신규 서비스: 한 곳에서 모든 것을 볼 수 있게 만들고자 함
		\begin{itemize}
		\item 명성이 있는 기존의 방송 채널이나 영화사는 그럴 유인이 낮음 $\rightarrow$ 자사 보유 목록을 활용
		\item[예)] 디즈니 플러스
		\end{itemize}
	\item 서비스 설계, 유지, 행정 비용 등 높은 고정 비용 $\rightarrow$ 규모의 경제를 누리기 위해서는 많은 미디어를 동시에 제공하는 것이 유리
	\end{itemize}
\item 묶음 판매의 경제학 \citep{Waldfogel:2020aa}
			\begin{table}[htp]
			\caption{지불 가능 금액과 묶음 판매}
			\begin{center}
			\begin{tabular}{cccc}
			\toprule
			 &  사랑이 뭐길래 & 사랑의 불시착 & 총합 \\
			\midrule
			가 & 500원 & 700원 & 1,200원 \\
			나 & 800원 & 500원 & 1,300원 \\
			\bottomrule
			\end{tabular}
			\end{center}
			\label{tab:bundle}
			\end{table}%
	\begin{itemize}
	\item 소비자
		\begin{itemize}
		\item ``사랑이 뭐길래"와 ``사랑의 불시착"을 각각 700원에 판매한다면
			\begin{itemize}
			\item 가는 ``사랑의 불시착"만, 나는 ``사랑이 뭐길래"만 구매하므로 수입: 700원 $\times$ 2 $=$ 1,400원
			\end{itemize}
		\item ``사랑이 뭐길래"와 ``사랑의 불시착"을 각각 500원에 판매한다면
			\begin{itemize}
			\item 가와 나 모두, 각 드라마를 구매하므로 수입: 500원 $\times$ 4 $ = $ 2,000원
			\end{itemize}
		\item ``사랑이 뭐길래"와 ``사랑의 불시착"을 합쳐서 1,200원에 판매한다면
			\begin{itemize}
			\item 가와 나 모두 구매하므로 수입: 1,200원 $\times$ 2 $ = $ 2,400원
			\end{itemize}
		\item 숫자 예가 중요한 것이 아니라, 묶음 판매가 수입을 늘릴 수 있다는 것이 중요
		\end{itemize}
	\item 플랫폼
		\begin{itemize}
		\item 묶음 판매는 상품이나 서비스를 한 단위 더 생산/전달하는 비용이 낮을 수록($=$한계 비용이 낮을 수록 $=$ 늘어나는 소비에 대응하기 위한 생산/전달의 비용이 낮을 수록) 유리한 전략
			\begin{itemize}
			\item 최초로 생산/전달하는 비용은 높지만 한 단위 더 생산/전달하는 비용이 낮은 상품이나 서비스라면
			\item[예)] 노래 한 곡, 영화 한 편, 드라마 한 편의 제작비는 많이 들지만, 이를 반복 재생하여 전달하는 비용은 낮음 
			\end{itemize}
		\item $\rightarrow$ 소비자의 수가 많은 것이 유리
		\item $\rightarrow$ 소비자를 유인할 필요가 있음
		\item $\rightarrow$ 낮은 가격, 다양한 음악, 영화, 방송 프로그램을 제시하는 것이 유리
			\begin{itemize}
			\item 두 개의 드라마만 예를 들었지만, 특정 음악, 영화, 방송 프로그램에 대한 지불 가능 금액은 소비자마다 다양
			\item 소비자의 다양한 취향을 맞출 수 있는 목록을 보유할 필요가 있음
			\end{itemize}
		\end{itemize}
	\item 콘텐츠 생산자
		\begin{itemize}
		\item 소비자가 노래 한 곡, 영화 한 편, 드라마 한 편을 구매할 때마다 일정 비율의 저작권 수입을 얻는 것이 아니라, 청취 또는 시청이 있을 때에 수입이 발생
		\item 예를 들어, 소비자가 노래 한 곡(1,000원)을 살 때마다 창작자($=$ 콘텐츠 생산자)가 100원을 받는다고 가정
		\item 월 10,000원의 회원 가입을 하면 소비자가 무제한으로 음악을 들을 수 있고, 이제 노래 한 곡을 들을 때 마다 창작자에게 10원이 지급된다고 가정
		\item $\rightarrow$ 소비자가 한 달에 몇 곡의 노래를 샀을까? 하루에 한 곡만 듣는다면 30곡을 샀을까? 그 이하일 가능성이 높을 것, 즉 소수의 창작자에게만 수입이 발생
		\item $\rightarrow$ 월 정액제에 가입을 하면 하루에 한 곡만 들을까? 그 이상일 수도 있을 것, 즉 다수의 창작자에게 수입이 발생
		\item 콘텐츠 생산자의 수입 $=$ 가격 $\times$ 수량 이므로, 낱개 판매에서 묶음 판매로 바뀌면서 단위 당 가격이 낮아지더라도, 수량이 늘어나면 수입이 늘어날 수 있음
		\end{itemize}
	\end{itemize}	
\item 가격차별(price discrimination)
	\begin{itemize}
	\item 기본적으로는 동일한 상품 또는 서비스를, 소비자의 지불구매 금액에 따라 다른 가격으로 판매하는 전략
	\item 보통은 기능 또는 성능 등의 차이를 두어 가격 차이를 만듬
		\begin{itemize}
		\item[예)] 개인 사용 대 가족 공동 사용, 저화질 대 고화질 등
		\end{itemize}
	\end{itemize}	
\end{itemize}

\section{음악, 방송, 영화 산업의 경제학}
\subsection{음악 산업: 애플}
\begin{itemize}
\item 애플 사의 짧은 역사
	\begin{itemize}
	\item 1976년 4월 1일, 스티브 잡스(Steve Jobs), 스티브 워즈니악(Steve Wozniak), 로널드 웨인(Ronald Wayne) 이 창업
		\begin{itemize}
		\item 애플 I (Apple I) 컴퓨터 출시
		\end{itemize}
	\item 1977년 1월 3일, 주식회사로 변경, 로널드 웨인은 자기 주식을 800달러에 두 스티브에게 매각하고 사업에서 철수, 마이크 마큘러(Mike Markkula)가 25만 달러를 투자
	\item 1977년 4월 16일, 애플 II (Apple II) 컴퓨터 출시
		\begin{itemize}
		\item 1977년부터 9월부터 1980년 9월까지 연평균 533\% 성장, 매출 77만 5천 달러에서 1억 1,800만 달러로 증가
		\end{itemize}
	\item 1980년 12월 12일, 주식 시장 상장
		\begin{itemize}
		\item 460만 주가 주당 22달러로 거래되기 시작 29달러로 마감: 1956년 포드 자동차의 상장 이후 최대 규모
		\end{itemize}
	\item 1983년, CEO로 존 스컬리 (John Sculley) 영입
	\item 1984년, 매킨토시 (Macintosh) 개인용 컴퓨터 출시
	\item 1985년 9월, 스티브 잡스 해고
		\begin{itemize}
		\item 스티브 잡스는 퇴사 후 넥스트 (NeXT) 설립
		\item 동년 초, 스티브 워즈니악 퇴사
		\end{itemize}
	\item 1991년, 파워북 (PowerBook) 노트북 출시
	\item 1994년, 아이비엠 (IBM), 모토롤라 (Motorola)와 함께, 에아이엠 동맹 (AIM alliance) 결성, 파워 피씨 플랫폼 (PowerPC Reference Platform) 제작 
	\item 1997년, 넥스트 사를 인수하고, 넥스트 스텝 운영 체제(NeXTSTEP operating system)와 스티브 잡스를 재영입
	\item 1997년 7월, 스티브 잡스 임시 CEO로 복귀
	\item 1997년 8월, 마이크로소프트 사의 투표권이 없는 1억 5천만 달러 주식 투자와 애플 컴퓨터에서 구동하는 마이크로소프트 오피스 발표
	\item 1997년 11월, 애플 스토어 (Apple Store) 웹 사이트 공개
	\item 1998년 8월, 일체형 컴퓨터, 아이맥 (iMac) 출시
	\item 2001년 3월 24일, 새로운 운영체제인 맥 오에스 텐 (Mac OS X) 공개
	\item 2001년 10월 23일, 휴대용 디지털 오디오 플레이어 아이팟 (iPod) 공개
	\item 2003년, 아이튠즈 스토어 (iTunes Store) 공개
		\begin{itemize}
		\item 곡당 0.99달러 판매, 아이팟과 연동
		\item 2008년 6월 19일, 다운로드 수 50억 회 돌파 
		\end{itemize}
	\item 2005년, 2006년 부터 인텔 (Intel) 기반 컴퓨터로의 전환을 발표
	\item 2007년, 아이폰 (iPhone), 애플 티비 (Apple TV) 공개
	\item 2007년 2월 6일, 아이튠즈 스토어 (iTunes Store)에서 디지털 권리 관리 (DRM: Digital Rights Management)없이 판매하겠다고 발표
	\item 2008년 7월, 아이폰과 아이팟 터치의 앱 스토어 공개
	\item 2010년 1월 27일, 아이패드(iPad) 공개
	\item 2011년 1월 6일, 맥 컴퓨터의 앱 스토어 공개
	\item 2014년 9월 9일, 스마트 워치 애플 워치 (Apple Watch) 공개
	\item 2015년 6월 8일, 스트리밍 기반 애플 뮤직 (Apple Music) 발표
		\begin{itemize}
		\item 2016년 동영상도 포함됨
		\item 월 정액 요금
		\end{itemize}
	\item 2017년 6월, 스마트 스피커 홈 팟 (Homepod) 공개
	\end{itemize}
\item 2001년 아이팟 $+$ 2003년 아이튠즈 스토어	
	\begin{itemize}
	\item 음악 재생 기기와 음원을 같이 판매	
	\item 아이폰, 아이패드와 같은 기기 사용의 편리성을 위한 서비스에 가까움
		\begin{itemize}
		\item 음악뿐만 아니라, 자체 제작 영화, 드라마 및 게임과 뉴스 구독 등 다른 서비스와 묶음 판매
		\end{itemize}	
	\end{itemize}
	
			\begin{table}[htp]
			\begin{center}
			\begin{threeparttable}
			\caption{애플의 경영 성과, 2015--2020}\label{tab:apple}
			\begin{tabularx}{\textwidth}{lrrrrrr}
			\toprule
			& 2015 & 2016 & 2017 & 2018 & 2019 & 2020 \\
			\midrule
			아이폰 & 155,041 & 136,700 & 141,319 & 166,699 & 142,381 & 137,781 \\
			아이패드 & 23,227 & 20,628 & 19,222 & 18,805 & 21,280 & 23,724\\
			맥 & 25,471 & 24,348 & 29,980 & 25,484 & 25,740 & 28,622 \\
			웨어러블, 홈, 액세서리\tnote{a} & &  & 12,826 & 17,381 & 24,482 & 30,620 \\
			서비스\tnote{b} & &  & 32,700 & 39,748 & 46,291 & 53,768 \\
			\bottomrule
			\end{tabularx}
			\begin{tablenotes}
			\small
			\item[a] 에어팟, 애플 TV, 애플 워치, 비츠 제품군, 홈팟, 아이팟 터치, 애플 및 기타 브랜드 액세서리 포함 
			\item[b] 디지털 콘텐츠와 서비스, 애플 케어, 애플 페이, 라이센스, 기타 수입을 포함 
			\item 단위: 백만 달러 
			\item 출처: Apple 10-K, 각 연도, \url{https://investor.apple.com/investor-relations/default.aspx} 
			\end{tablenotes}
			\end{threeparttable}
			\end{center}
			\end{table}%
	
\item 애플 음악 서비스의 요금 구조 (2021년, 미국 기준)
	\begin{itemize}
	\item 아이튠즈: 곡당 0.99달러 부터, 앨범당 4.99달러 부터
	\item 애플 뮤직: 광고 없는 스트리밍, 학생 (월 4.99달러), 개인 (월 9.99달러), 가족(최대 6명, 월 14.99달러)
	\item 애플 원
		\begin{itemize}
		\item 개인, 월 14.95달러: 애플 뮤직 $+$ 애플 티비 플러스 $+$ 애플 아케이드(게임) $+$ 애플 아이클라우드 플러스 50기가바이트
		\item 가족, 최대 6명, 월 19.95달러: 애플 뮤직 $+$ 애플 티비 플러스 $+$ 애플 아케이드(게임) $+$ 애플 아이클라우드 플러스 200기가바이트
		\item 프리미어, 최대 6명, 월 29.95달러: 가족 $+$ 애플 뉴스 플러스 $+$ 애플 피트니스 플러스
		\end{itemize}
	\end{itemize}
\item 스트리밍 음악 서비스\footnote{스트리밍 음악 서비스의 특징을 설명하지만, 상호성/비상호성, 구독형/광고형, 규모의 경제, 서비스 간 경쟁 등의 특징은 스트리밍 영상 서비스에도 동일하게 적용된다고 할 수 있음} \cite[8장]{Krueger:2021aa}	
	\begin{itemize}
	\item 상호성 대 비상호성 (interactive vs. non-interactive)
		\begin{itemize}
		\item 상호성: 청취자는 음악가, 앨범, 노래를 선택할 수 있음
		\item 비상호성: 음악이 사전에 프로그래밍되어 있음
		\end{itemize}
	\item 구독형 대 광고형
		\begin{itemize}
		\item 구독형: 광고가 없는 대신 일정 금액을 지불
			\begin{itemize}
			\item 사용자 수, 음악 품질 등에 따른 가격 차별 가능
			\end{itemize}
		\item 광고형: 무료로 사용하는 대신, 광고가 있음
		\item 생산자 입장에서는 보통 구독 수입 배분율이 광고 수입 배분율보다 좋은 것으로 알려져 있음
		\item 광고 $\rightarrow$ 소비자는 시간을 대가로 지불하는 셈
			\begin{itemize}
			\item 판도라 라디오, 2014년 6월 -- 2016년 4월, 광고량과 구독형 서비스 전환에 대한 실험 \citep{Huang:2018vq}
			\item 9개의 처치 그룹으로 무작위 할당: 21개월 동안 광고 수, 광고의 주기 등을 통제
			\item 1개의 통제 그룹: 한시간에 약 3분의 광고 청취
			\item 통제 그룹에 비해 광고를 가장 많이 듣는 그룹의 청취 시간은 감소, 반면 통제 그룹에 비해 광고를 가장 적게 듣는 그룹의 청취 시간은 증가
			\item 광고가 증가함에 따라 청취 시간은 감소 (사용 일수 감소 및 사용 중단이 확인됨) $\rightarrow$ 광고와 사용자 간의 상충 관계(trade-off)가 있음을 의미
			\item 하지만, 광고로 인한 수입 증가도 가능함이 확인됨 $\rightarrow$ 광고 증가로 사용자 감소가 나타날 수있지만, 광고 수입 증가가 사용자 감소를 상쇄할 수 있기 때문
			\item 연령대가 높아질 수록, 광고가 아닌 유료 구독 서비스를 선택하는 것이 확인됨
			\end{itemize}
		\end{itemize}
	\item 규모의 경제
		\begin{itemize}
		\item 웹 개발, 추천 알고리듬 개발 등은 고정 비용 $\rightarrow$ 하지만 사용자 수의 증가에 따라 상승하지는 않음
		\item 사용자 수가 많을 수록 추천은 정교해지고, 음반사와의 협상력은 높아짐
		\item $\rightarrow$ 사용자 수를 늘리는 것이 중요
		\item $\rightarrow$ 세계 시장으로의 확장
		\end{itemize}
	\item 스트리밍 음악 서비스와 음악 생산자의 계약은 보통 비공개
		\begin{itemize}
		\item 음악 생산자: 가수, 연주자, 작곡가, 작사가, 음반사(기획사) 등
		\item 스트리밍 음악 서비스 기업은 보통 음반사(기획사)와 비공개 계약을 맺음
			\begin{itemize}
			\item 수입의 65--70\%를 저작권료로 지급하는 것으로 추정
			\item 구체적인 계약에 따라 복잡해질 수 있음: 프로모션에서의 배분율, 기본 배분율 $+$ 수입 비중 당 배분율 또는 재생 횟수 당 배분율 또는 송출 횟수 당 배분율 등 다양한 계약이 가능하기 때문
			\end{itemize}
		\item 애플 뮤직은 프로모션으로 가입 시 최초 3개월 무료 $\rightarrow$ 이 기간 동안의 음악 청취에 대해서는 저작권료를 지불하지 않았음 $\rightarrow$ 가수 테일러 스위프트(Talyor Swift)의 문제 제기 $\rightarrow$ 지불 하기로 함 \citep{Helman:2015ur}
		\item 저작권료 이외에 데이터 기반의 보완 서비스 제공
			\begin{itemize}
			\item 스트리밍 정보 $\rightarrow$ 중요 청취자, 공연 지역 선택 등
			\end{itemize}
		\end{itemize}
	\item 서비스 간 경쟁
		\begin{itemize}
		\item $\rightarrow$ 보유하고 있는 음악 카탈로그가 비슷해질 수 있음
		\item $\rightarrow$ 차별화: 재생 목록 선곡 또는 추천, 음성 인식 기능 등
			\begin{itemize}
			\item $\rightarrow$ 사용자 개인의 취향에 대한 고유한 정보를 획득 
			\item $\rightarrow$ 사용자 맞춤형 서비스를 제공할 수록 다른 서비스로의 이동을 막을 수 있음
			\item $\rightarrow$ 다른 서비스로 이동하지 않는 사용자에 대해 부가 서비스를 제공하고 추가 과금을 설계할 수 있음
			\end{itemize}
		\end{itemize}	
	\end{itemize}
\end{itemize}


\subsection{방송, 영화 산업: 넷플릭스}
\begin{itemize}
\item 넷플릭스의 짧은 역사
	\begin{itemize}
	\item 1997년 8월, 마크 랜돌프(Marc Randolph)와 리드 헤이스팅스(Reed Hastings)가 미국, 캘리포니아 스콧츠 밸리에서 창업
		\begin{itemize}
		\item DVD 우편 대여 서비스로 시작
		\item 미국 시장에 DVD가 출시된 것은 1997년 3월 24일	
		\end{itemize}
	\item 1999년, 월 회원제 무제한 대여 서비스 시작 $\rightarrow$ 2000년 초 개별 대여 서비스 중단
	\item 2002년 5월, 주식 시장 상장
	\item 2003년 흑자 전환, 매출 2억 7,200만 달러, 이윤 650만 달러
	\item 2007년 1월, 스트리밍 서비스 시작
	\item 2009년, 스트리밍이 DVD 대여를 추월
	\item 2010년 1월, 워너 브라더스(Warner Bros.) 스튜디오와 신작 영화 공개 28일 후 대여 계약 체결
		\begin{itemize}
		\item	동년 4월, 유니버셜 픽쳐스(Universal Pictures)와 20세기 폭스(20th Century Fox)와 유사한 계약을 체결
		\end{itemize}
	\item 2010년 8월, 파라마운트(Paramount), 라이온스게이트(Lionsgate), 메트로-골드윈-메이어(Metro-Goldwyn-Mayer) 제작 영화의 5년 간 스트리밍 계약 체결 
	\item 2010년 9월, 캐나다에서 미국 외 국가로는 처음 서비스를 시작
		\begin{itemize}
		\item 2011년 9월, 라틴 아메리카
		\item 2012년 1월, 영국과 아일랜드
		\item 2013년 9월, 네덜란드
		\item 2014년 9월, 독일, 오스트리아, 스위스, 프랑스, 벨기에, 룩셈부르크
		\item 2015년 3월, 호주, 뉴질랜드
		\item 2015년 10월, 스페인, 포르투갈, 이탈리아
		\item 2016년 1월, 그외 전 세계(중국, 크림 반도, 북한, 시리아 제외)
		\end{itemize}
	\item 2010년, 브레이킹 배드(Breaking Bad)의 권리를 획득 
		\begin{itemize}
		\item 넷플릭스 효과 (Netflix Effect)의 시초
		\item 넷플릭스 효과: 어떤 드라마 시리즈 또는 영화가 갑작스럽게 인기를 얻게 됨에 따라, 과거 시리즈 또는 배우의 과거 출연작의 시청자가 하룻 밤 사이에 급속하게 증가하는 현상
		\end{itemize}
	\item 2011년 1월, 오리지널 시리즈 하우스 오브 카드(House of Cards)	 제작 권리 획득
		\begin{itemize}
		\item 2013년 2월, 첫 방송
		\end{itemize}
	\item 2012년 12월, 디즈니(Disney)와 다년간의 회원제 텔레비전 방송의 미국 내 최초 공개 계약 체결
		\begin{itemize}
		\item 2019년, 디즈니가 자체 서비스인 디즈니 플러스(Dinsney$+$)를 시작하면서 종료
		\end{itemize}
	\item 2015년 10월 16일, 자체 제작 영화, 비스트 오브 노 네이션 (Beasts of no nation) 공개
		\begin{itemize}
		\item 이후, 극장 개봉없이 스트리밍으로만 공개하는 영화에 대한 영화제 초대 및 수상에 대한 논란이 제기됨
		\end{itemize}
	\item 2017년 3월, 자체 제작 콘텐츠가 1,000시간이 넘는다고 발표
	\item 2021년 7월, 모바일 게임 제작 발표(구독 요금으로 사용가능하도록 할 예정)		
	\end{itemize}

		\begin{table}[htp]
		\begin{center}
		\begin{threeparttable}
		\caption{넷플릭스의 경영 성과, 2015--2020}\label{tab:netflix}
		\begin{tabularx}{\textwidth}{lrrrrrr}
		\toprule
		& 2015 & 2016 & 2017 & 2018 & 2019 & 2020 \\
		\midrule
		매출 & 6,779,511 & 8,830,669 & 11,692,713 & 15,794,341 & 20,156,447 & 24,996,056\\
		(스트리밍) & & & 11,242,216 & 15,428,752 & 19,859,230 & 24,756,675\\ 
		(DVD) & & & 450,497 & 365,589 & 297,217 & 239,381\\
		\midrule
		스트리밍 가입자\tnote{a} & 70,839 & 89,090 & 110,644 & 139,259 & 167,090 &  203,663 \\
		월 평균 가입비 & 8.15 & 8.61 & 9.43 & 10.31 & 10.82 & 10.91\\  
%		스트리밍 서비스 연말 가입자 수 & 74,762 & 93,796 & 117,582 &  \\
%		연말 가입자 수 & 44,738 & 49,431 & 54,750 &  \\
%		연말 유료 가입자 수 & 43,401 & 47,905 & 52,810 & \\
%		월 평균 가입비 & 8.50 & 9.21 & 10.18 & \\  
		\bottomrule
		\end{tabularx}
		\begin{tablenotes}
		\small
		\item[a] 연말 유료 기준 
		\item 단위: 천 달러, 천명, 달러
		\item 출처: 넷플릭스 10-K, 각 년도, \url{https://ir.netflix.net/ir-overview/profile/default.aspx}
		\end{tablenotes}
		\end{threeparttable}
		\end{center}
		\end{table}%	
	
\item 영상 산업은 상대적으로 음악 산업에 비해 무료 다운로드의 충격을 덜 받음
	\begin{itemize}
	\item 2000년 초의 통신 기술로, 대용량 동영상 파일의 전송 한계가 있었기 때문
	\item 음악 산업의 경험을 토대로 빠르게 새로운 수입 창출 방법을 찾음
	\end{itemize}	
\item OTT 경쟁 격화
	\begin{itemize}
	\item 미국의 경우, 넷플릭스(Netflix), 훌루(Hulu), 아마존 프라임 비디오(Amazon Prime Video) 이후 디즈니 플러스(Disney$+$), 애플티비 플러스(AppleTV$+$), 피콕(Peacock, NBC 유니버셜 산하), 에치비오 맥스(HBO Max, 워너 미디어 산하) 등 서비스 개시
	\item 일본의 경우, 글로벌 기업 외, 유--넥스트(U--NEXT), 디티비(dTV) 등 일본 현지 서비스도 시장 확대 중
	\item 한국의 경우, 웨이브(Wavve, KBS, MBC, SBS, SKT), 티빙(Tving, CJ ENM와 JTBC) 등 서비스 중
	\end{itemize}	
\item 디지털화로 인해 전세계 배급이 과거보다 용이해짐에 따라
	\begin{itemize}
	\item 가능한 많은 복권을 뽑는 것이 유리한 전략 \cite[p. 202]{Waldfogel:2017aa}
		\begin{enumerate}
		\item 신규 제작 수를 늘림
		\item 전통적인 문지기 구조에서는 만들지 않았을 작품을 제작  
		\item 과거 출시작 중 현재에도 볼만한 작품을 많이 보유하는 것이 유리
		\end{enumerate}
	\end{itemize}
\end{itemize}



\section{미디어 플랫폼 관련 이슈}
\begin{itemize}
\item 망 중립성 (net neutrality) \citep{Crawford:2015aa}
	\begin{itemize}
	\item 모든 인터넷 통신량은(traffic)은 동등하게 취급되어야 한다는 공개성(openness)의 원칙
		\begin{itemize}
		\item 최종적인 콘텐츠 공급자에게 사용자가 접근하는 것을 막아서는 안되며
		\item 콘텐츠 공급자 간의 사용자 접근에 대해 가격 차별을 두어서는 안됨
		\end{itemize}	
	\item 유튜브, 넷플릭스 등 온라인 동영상 시청의 증가 $\rightarrow$ 네트워크 사용량 증가로 망 중립성, 즉 망 사용료 문제가 문제가 됨
		\begin{itemize}
		\item 중국, 투도우(土豆) 2011년 1억 9,300만 명의 활성 사용자 기록, 매출의 42.1\%를 서버 및 네트워크 사용 비용으로 지불 $\rightarrow$ 알리바바의 요우쿠(优酷)와 합병
		\item 한국, 2021년 현재, 넷플릭스의 망사용료에 대한 소송 진행 중
		\end{itemize}
	\item 인터넷 네트워크 서비스 공급자(ISP)가 미디어 콘텐츠 공급도 하는 경우, 경쟁 미디어 콘텐츠 플랫폼과의 품질 차이가 없도록 하는 것도 포함됨
	\end{itemize}
%\item 수익 배분
%	\begin{itemize}
%	\item 
%	\end{itemize}
\item 보완재
	\begin{itemize}
	\item 온라인 실시간 공연
		\begin{itemize}
		\item 디지털 음원의 충격으로 음악 산업이 찾은 새로운 수입원은 공연(특히, 미국의 경우)
		\item 디지털 음원을 마치 공연에 대한 광고처럼 활용
		\item 코로나19 이후 온라인 실시간 공연을 어떻게 전달할 것인가의 문제가 대두됨
		\end{itemize}
	\item 추천 알고리듬: 재생 목록 또는 추천 영화/드라마
		\begin{itemize}
		\item 새로운 문지기 \citep{Aguiar:2018ts,Aguiar:2021vb}
		\item 2017년, 스포티파이의 금요일 신곡(New Music Friday) 목록과 이후 스포티파이에서 발표하는 일일 재생 순위 200위 노래의 관계를 분석 
		\item 금요일 신곡 목록에 등장한 노래는 스포티파이의 다른 재생 목록에 등장할 가능성이 높아지며, 신곡의 발견과 성공에 영향을 미치는 것으로 나타남
		\item $\rightarrow$ 과거에 발표된 음악 재생에도 영향을 미칠뿐만 아니라 미래에 발표될 음악에도 영향을 미칠 가능성이 있음
		\end{itemize}
	\item 배경 음악 또는 관련 영상
		\begin{itemize}
		\item 아직 매출 등에서 높은 비중을 차지하고 있지 않음
		\item 사용자 제작 콘텐츠가 늘어난다면, 새로운 시장이 될 가능성이 있음
		\end{itemize}	
	\end{itemize}
\item 대체재
	\begin{itemize}
	\item 온라인 실시간 방송
		\begin{itemize}
		\item 게임 방송, 상품 판매 등
		\item 더 폭 넓게는 유튜브 등으로 대표되는 사용자 제작 콘텐츠
		\end{itemize}
	\item 일명 스크린 전쟁 $\rightarrow$ 넷플릭스의 게임 출시 선언
	\end{itemize}
\end{itemize}

\pagebreak

\section*{정리하기}
\begin{enumerate}
\item 미디어 산업이 온라인, 디지털화 됨에 따라 전통적인 미디어 산업에 비해 낮은 비용으로 생산 및 전달이 가능해졌지만, 이로 인해, 저작권 등의 문제가 나타났다.
\item 온라인 미디어는 전통적인 미디어에 비해 다양한 방식으로 묶음 판매와 가격 차별이 가능해졌다.
\item 스트리밍 음악 및 영상 서비스는 사용자가 음악, 영화, 방송 프로그램 등을 선택할 수 있는 경우 상호성이 있다고 한다. 구독형 서비스는 광고가 없는 대신 일정 금액을 지불해야 하고 광고형 서비스는 무료로 사용하는 대신 광고가 있다. 
\item 규모의 경제와 협상력 증대를 위해 온라인 배급 플랫폼은 사용자 수를 많이 확보하는 것이 유리하고, 이를 위해 세계 시장으로 확장하는 전략을 갖게 된다.
\item 서비스 간 경쟁으로 보유하고 있는 음악이나 영상의 목록이 유사해지면, 추천 서비스 등의 부가 서비스로 차별화를 꾀하게 된다.
\item 어떤 작품이 성공할 지 모르는 불확실성이 항상 존재하므로, 가능한 많은 복권을 뽑는 것이 유리한 전략이다.
\item 동영상 콘텐츠의 인터넷 사용량이 늘어나면서, 망 사용료 부담, 넓게 보아 망 중립성에 대한 문제가 다시 제기되고 있다.
\item 보완재의 관점에서 보면, 온라인 실시간 공연, 추천 알고리듬과 사용자 정보, 사용자 제작 콘텐츠 증가에 따른 배경 음악 또는 영상 제공 수입 등의 증가를 생각해볼 수 있다.
\item 대체재의 관점에서 보면, 온라인 실시간 방송 및 게임과 같은 대체재와의 경쟁을 생각해볼 수 있다.
\end{enumerate}