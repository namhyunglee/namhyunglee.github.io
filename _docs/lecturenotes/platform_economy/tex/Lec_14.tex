\chapter{플랫폼과 노동: 이론}\label{cha:platforminlabormarket}

\section*{학습개요}
플랫폼을 통해 중개되는 노동의 특징을 학습한다.

\section*{학습목표}
\begin{enumerate}
\item 플랫폼을 통한 노동 중개의 개념을 설명할 수 있다.
\item 디지털 기반의 플랫폼 노동 중개의 분류와 각각의 특징을 설명할 수 있다.
\item 노동에 대한 플랫폼 중개의 장점과 단점을 기업과 노동자의 입장에서 각각 설명할 수 있다.
\item 노동의 플랫폼 중개에 대한 다양한 평가를 설명할 수 있다.
\end{enumerate}

\section*{주요 용어}
온라인 웹 기반 노동 중개 플랫폼, 위치 기반 노동 중개 플랫폼, 알고리듬에 의한 관리

\pagebreak

\section{플랫폼과 노동}\label{sec:}
\begin{itemize}
\item 플랫폼 관련 노동의 유형
	\begin{itemize}
	\item 플랫폼 기업에 종사하는 노동
	\item 플랫폼을 통한 노동 중개
		\begin{itemize}
		\item 노동 의뢰인 -- 플랫폼 -- 노동자
		\item 노동자도 다시 두 부류로 나눌 수 있음
			\begin{itemize}
			\item 플랫폼 기업이 직접 고용한 노동자
			\item 플랫폼의 중개를 받아 일을 하는 노동자 $\rightarrow$ 보통, 자영업자 (self-employed)나 독립 계약자 (independent contractor)로 분류
			\end{itemize}
		\end{itemize}
		
		\begin{figure}[htbp]
		\begin{center}
		\includegraphics[scale=0.2]{platform_labor.png}
		\caption{플랫폼을 통한 노동 중개}
		\label{fig:platform_labor}
		\end{center}
		\end{figure}		
		
	\item 앞으로 다루고자 하는 것은 후자, 특히 디지털 기반의 플랫폼을 통한 노동 중개 문제
	\end{itemize}
\item 디지털 기반 플랫폼의 노동 중개 \citep{International-Labour-Office:2021uk}	 \footnote{이 외에도 \cite{Schmidt:2017ws}는 클라우드 업무(cloud work)와 긱 업무(gig work), 그리고 과업 수행자가 특정 개인에게 주어지는가 아니면 대중에게 주어지는가 등으로 분류, \cite{Fernandez-Macias:2018wq}는 과업을 수행하기 위한 기술 수준, 노동 전달 방식(온라인/오프라인/혼합), 선택 과정(플랫폼/의뢰인/노동자/혼합)류 매칭 방식(제안, 경연) 등으로 분류, \cite{Groen:2021uc}는 \cite{International-Labour-Office:2021uk}의 분류 중 온라인 웹 기반 플랫폼에 의료 조언(medical consultation), 위치 기반 플랫폼에 가정 및 돌봄 서비스(home and care services), 가내 업무(domestic work)를 추가}
	\begin{enumerate}
	\item 온라인 웹 기반 (online web-based digital labor platform)
		\begin{itemize}
		\item 과업(task) 또는 업무(work)가 온라인 또는 원격으로 수행됨
		\item 프리랜서 (freelance)
			\begin{itemize}
			\item 일종의 노동 거래 시장, 의뢰인이 제시한 구체적인 과업을 수행할 수 있는 노동자를 매칭
			\item[예)] 번역, 법률/재무/특허 서비스 등, 
			\end{itemize}	
		\item 경연 기반 (contest-based)
			\begin{itemize}
			\item 창작물, 예술품 등의 제작 의뢰, 창작자는 가격을 확인하고 경쟁하거나 일정 가입비를 내고 회원 상태를 유지
			\end{itemize}
		\item 경쟁 프로그래밍 (competitive programming)
			\begin{itemize}
			\item 소프트웨어 개발자 등에게 인공지능, 데이터 애널리틱스, 소프트웨어 개발 등의 목표를 제시하고, 의뢰인이 우승자를 선발
			\end{itemize}
		\item 미세과업 (microtask)
			\begin{itemize}
			\item[예] 이미지 분류, 콘텐츠 관리, 동영상 자막 제작 등
			\end{itemize}
		\end{itemize}
	\item 위치 기반 (location-based digital labor platform)
		\begin{itemize}
		\item 특정 물리적 공간에서 노동을 수행
		\item 택시 (taxi)
		\item 배달 (delivery)	
		\end{itemize}
	\end{enumerate}
\item 플랫폼 중개 노동의 특징 \citep{Fernandez-Macias:2018wq}
	\begin{itemize}
	\item 플랫폼 기업의 수입
		\begin{itemize}
		\item 서비스 구독료와 수수료
		\item 의뢰인과 노동자 모두에게 부과
		\end{itemize}
	\item 알고리듬에 의한 자동 통제
		\begin{itemize}
		\item 과업의 분할 및 업무 특성에 따라 의뢰인과 노동자를 매칭
			\begin{itemize}
			\item 전통적으로는 교육 훈련 수준 등에 의해 결정
			\item $\rightarrow$ 별점 평가, 리뷰, 취소율 및 수락률, 노동자 기록에 의한 기계적 매칭으로 대체
			\end{itemize}
		\item 노동 현황을 실시간으로 확인 가능
			\begin{itemize}
			\item 데이터 기반의 세분화된 노동 과정의 통제
			\item[예)] 택시 및 배달: 호출 -- 이동 -- 승차 또는 수취 -- 이동 -- 도착
			\end{itemize}
		\end{itemize}	
	\item 플랫폼 사용 약관 동의에 의한 노동 관계가 일반적
		\begin{itemize}
		\item $\rightarrow$ 근무 환경, 일자리 안정성, 소득, 노동 시간에 영향
		\end{itemize}
	\item 노동자의 개인화: 자영업자, 독립 계약자인 경우, 많은 국가에서 제도적으로
		\begin{itemize}
		\item 단체 교섭(collective bargaining)의 대상이 되지 않음
		\item 또 연금, 건강보험, 고용보험, 산재보험 등 사회 보장 제도의 적용을 받지 않을 수 있음
		\end{itemize}
	\end{itemize}	
\item 플랫폼 관련 노동자의 개념적 분류 \citep{Vallas:2020aa}
	\begin{itemize}
	\item 창업자, 고숙련 노동자, 독립 계약자
		\begin{itemize}
		\item 플랫폼 및 디지털 인프라스트럭쳐의 설계와 보수를 담당
		\item 플랫폼 기업의 직접 고용이 대부분이지만, 일부는 아웃소싱할 수도 있음
		\item 지리적 분산 높음 $+$ 기술 수준 높음
		\end{itemize}
	\item 플랫폼을 활용하는 온라인 기반 전문 컨설턴트 또는 프리랜서
		\begin{itemize}
		\item 프로젝트 기반의 특정 업무를 담당: 그래픽 디자인, 컴퓨터 프로그래밍, 저널리스트 등 
		\item 지리적 분산 높음 $+$ 기술 수준 높음
%		\item 충분한 수의 의뢰인을 꾸준히 유지하는 경우 $\rightarrow$ 안정적인 고수익
%		\item 기업이 아웃소싱을 하여 플랫폼을 통한 인력 매칭을 하는 가와 플랫폼이 기존의 인력 매칭 서비스를 대체하는가를 나누어 생각할 필요가 있음
		\end{itemize}	
	\item 플랫폼을 활용하는 오프라인 기반 노동자
		\begin{itemize}
		\item 택시, 음식 배달, 가사 관리, 돌봄 서비스 제공 등
		\item 지리적 분산 낮음 $+$ 기술 수준 낮음
		\end{itemize}
	\item 플랫폼을 활용하는 온라인 기반 노동자
		\begin{itemize}
		\item 미세과업자(microtasking): 이미지 분류, 콘텐츠 관리, 동영상 자막 제작 등
		\item 지리적 분산 높음 $+$ 기술 수준 낮음
		\end{itemize}
	\item 플랫폼을 활용하는 창작자
		\begin{itemize}
		\item 소셜 미디어 등의 콘텐츠 제작자 등
		\item 지리적 분산 낮음 $+$ 기술 수준 높음
		\end{itemize}	
	\end{itemize}	
\end{itemize}	

\section{플랫폼 중개 노동의 명암}
\begin{itemize}
\item 플랫폼 중개 노동의 장점
	\begin{itemize}
	\item 기업
		\begin{itemize}
		\item 채용 과정의 생략 또는 단순화 $\rightarrow$ 비용 하락 및 효율성 향상
		\item 소비자 확대 및 시장 경쟁에 탄력적으로 대응 가능
			\begin{itemize}
			\item[예)] 코로나19 상황에서 요식업 등의 배달 서비스 활용
			\end{itemize}
		\item 인공지능, 업무 자동화, 데이터 분석 등 관련 기술 기업의 성장
		\end{itemize}
	\item 노동자
		\begin{itemize}
		\item 노동 기회 확대
		\item 소득 증가의 기회
		\item 노동 시간의 유연화와 자율성
		\end{itemize}
	\end{itemize}
\item 플랫폼 중개 노동의 위험
	\begin{itemize}
	\item 기업
		\begin{itemize}
		\item 특히, 온라인 웹 기반 플랫폼을 이용하는 경우, 업무 관리에서 다양한 어려움이 나타날 수 있음
		\item 디지털 인프라가 잘 구축되어 있지 않은 경우, 오히려 수익을 악화시킬 수 있음
		\item 플랫폼의 불투명성, 특히 데이터, 평가, 과금 체계 등의 비공개가 플랫폼과의 협상력을 약화시킬 수 있음
		\item 갈등 해결 제도가 취약한 경우 추가 비용 발생
		\end{itemize}
	\item 노동자
		\begin{itemize}
		\item 비용, 위험의 자기 부담
			\begin{itemize}
			\item 택시나 배달 서비스 등에 필요한 차량 등의 구입 비용 발생
			\item 안전 훈련 및 안전 장비 비용의 자기 부담 
			\item 안전 장비의 수량과 품질이 미흡하여 자기 부담을 해야하는 경우도 포함
			\end{itemize}
		\item 소비자 요구에 따른 근무는 오히려 노동 시간을 늘리거나 불안정하게 만들어 노동자의 자율성을 해칠 수 있음
		\end{itemize}
	\end{itemize}
\end{itemize}

\section{플랫폼 중개 노동에 대한 평가}
\begin{itemize}
\item 작은 기업가 (microentrepreneurs) \citep{Sundararajan:2016aa}
	\begin{itemize}
	\item 비도심 거주자, 장애인, 돌봄 의무자 등이 노동 기회를 갖게 됨
	\item 플랫폼에서의 신뢰 형성 시스템을 이용하여 구직이 가능해짐
	\end{itemize}
\item 디지털 통제 (digital cage)
	\begin{itemize}
	\item 정보비대칭을 활용한 노동 관리
		\begin{itemize}
		\item 택시 및 배달 서비스 $\rightarrow$ 노동을 제공하려는 사람들에게, 현재 플랫폼 이용자 수, 업무 할당 알고리듬 등을 공개하지 않음 $\rightarrow$ 노동자의 자율권과 소득을 통제
		\item 데이터를 이용한 작업장 규칙 규율 $\rightarrow$ 필수 노동량, 노동 대기 시간, 별점 평가 수준 등을 통제
		\end{itemize}
	\item 노동자 간의 관계를 최소화 $\rightarrow$ 연대의 기회를 줄이고 이로 인해 관리에 대한 저항을 통제
	\item 노동자 간의 경쟁을 강화
	\end{itemize}
\item 허용하는 군주 (permissive potentates)
	\begin{itemize}
	\item 작업 방법, 작업 일정, 성과 평가 등을 하지 않음
		\begin{itemize}
		\item 플랫폼의 성격 상, 원격으로 노동 과정을 통제하기는 어려움
		\item 플랫폼에는 다앙한 노동자가 접근할 수 있음
		\end{itemize}
	\item 하지만, 업무의 할당, 데이터 수집, 노동 가격 부과, 수입 등에 대해서는 강력한 통제권을 행사
	\end{itemize}	
\item 고용 및 소득 불안정성의 증대 (accelerants of precarity)
	\begin{itemize}
	\item 기업 또는 국가가 부담한 위험을 노동자에게 전가
		\begin{itemize}
		\item 최저임금, 보건안전 규칙, 퇴직 연금, 의료 보험, 노동 보상 등의 전통적인 보호를 받지 못하는 법적 또는 재정적 지위
		\item 플랫폼 노동의 문제라기 보다 더 광범위하고 오랫동안 지속되어온 사회경제적 흐름
		\end{itemize}
	\item 다수의 연구에서, 플랫폼을 이용하는 노동자는 하나 이상의 직업을 가진 경우가 많음이 확인됨
		\begin{itemize}
		\item 플랫폼 노동의 수입이 소득 불안정성을 낮추고, 주 직업의 낮은 소득 보상, 채무 상환, 저축 증대 등의 긍정적 효과를 줄 수도 있음
		\item 다른 한편, 플랫폼을 이용하는 노동자가 생활을 유지하기 위한 기본적인 지출을 플랫폼의 노동에 얼마나 의존하고 있는가가 중요하다는 지적도 있음  $\rightarrow$ 플랫폼 노동에의 의존도가 낮다면, 소득이 낮은 업무를 거부하고 노동 시장에서 더 나은 지위를 가질 수 있기 때문 \citep{Schor:2020vg}
		\end{itemize}
	\end{itemize}
\item 제도적 미비의 문제 \citep{Thelen:2018tl}
	\begin{itemize}
	\item 동일한 플랫폼 노동자라고 하더라도, 독일, 스웨덴, 미국 등 각국의 제도적 조건에 따라 노동자의 상황이 다름
	\item 미국의 경우, 플랫폼 노동자가 독립 계약에 의한 자영업자로 간주됨에 따라, 다양한 문제, 특히 사회 보장 제도의 보호를 받지 못하는 문제가 발생
	\item 하지만, 독일이나 스웨덴의 경우, 고용 지위 상의 문제가 아니라, 기존의 사회 서비스(예를 들어, 대중교통) 또는 조세 체계에 더 큰 위협이 됨
	\end{itemize}
\end{itemize}	

\pagebreak

\section*{정리하기}
\begin{enumerate}
\item 플랫폼을 통한 노동 중개는 기본적으로 법인 또는 개인의 의뢰인은 플랫폼에 수행되어야할 과업이나 업무를 의뢰하고 플랫폼은 이를 할 수 있는 개인에게 매칭하는 구조이다.
\item 디지털 기반 노동 중개 플랫폼은 과업 또는 업무가 온라인 또는 원격으로 수행되는 온라인 웹 기반 플랫폼과 특정 물리적 공간에서 노동이 수행되는 위치기반 플랫폼으로 구분할 수 있다.
\item 노동을 중개하는 플랫폼은 의뢰인과 노동자 모두로부터 서비스 구독료와 수수료 또는 모두를 수입으로 받는다. 매칭, 노동 관리 등의 많은 부분을 알고리듬에 의존하는 것이 특징이며, 노동의뢰인과 노동자는 플랫폼 사용 약관 동의로 계약을 맺는 것이 일반적이다. 이러한 형태의 노동자는 많은 국가에서는 아직 제도적으로 자영업자나 독립 계약자로 분류되어 단체 교섭이나 사회 보장 제도의 적용을 받지 않는 경우가 많다.
\item 플랫폼 중개 노동으로 기업은 기존 채용 과정의 비용을 낮추고 효율성을 높일 수 있으며, 소비자 확대와 시장 경쟁에 탄력적으로 대응할 수 있다. 하지만, 온라인 업무 관리에서 다양한 어려움이 나타날 수 있으며, 플랫폼의 평가 및 과금 체계 등이 불투명한 경우 협상력이 약화되고 갈등 해결 제도가 취약한 경우 추가 비용을 부담할 것이다.
\item 노동자는 노동 기회 및 소득을 증가시킬 수 있고 노동 시간과 업무의 자율성을 얻을 수 있다. 하지만, 관련 비용과 위험의 자기 부담이 발생할 수 있고, 소득과 노동 시간의 불안전성이 높아져 오히려 자율성을 해칠 수 있다.
\item 플랫폼 중개 노동에 대해서는 이전에 갖지 못한 노동 기회를 갖게 됨으로써 작은 기업가가 될 수 있다는 긍정적인 견해부터 디지털 통제가 강화되고 고용 및 소득의 불안정성이 높아진다는 부정적 견해까지 다양하다. 한편으로 플랫폼 노동 그 자체의 문제라기보다 노동 관련 제도의 문제라는 입장도 있다. 
\end{enumerate}


%\cite{Shapiro:2019aa}
%It seems clear cut that many labor markets depart rather significantly from the textbook model of perfect competition, in which employers are wage takers and face a highly elastic supply of labor. Labor markets are generally defined according to an occupation and a local geographic area (as emphasized by Moretti 2011). 
%With costs of job search and costs of geographical mobility, employers will have some degree of buyer power. Manning (2011) surveys the literature and concludes that “labor markets are pervasively imperfectly competitive.” Employers commonly share  relationship-specific rents with workers, so employees working at more productive firms earn higher wages (see, for example, Kline et al. 2017; Card et al. 2018). 
%Some local US labor markets are highly concentrated on the employer side, but that is not the situation for most workers. Azar, Marinescu, and Steinbaum (2017) use data from CareerBuilder.com to calculate labor market concentration in some 8,000 selected labor markets in the United States. They define these labor markets according to occupation and geography, such as “legal secretaries in the Denver area.” On average, 20 employers post job vacancies on CareerBuilder.com in a given market in a given quarter. They calculate an employer’s share in the labor market based on the number of vacancies listed by that employer at CareerBuilder.com in a given quarter, and they measure market concentration based on the Herfindahl–Hirschman index (HHI) on the employer side of the market. Weighting the geographic markets by population, the overall mean HHI is 1,691, which antitrust economists would classify as moderately concentrated. This method is likely to overestimate labor market concentration, because only about 35 percent of job openings nationally are listed on CareerBuilder.com. 
%Antitrust enforcement in labor markets has historically been extremely limited. 
%As discussed by Naidu, Posner, and Weyl (2018), this most likely reflects the view that most labor markets are reasonably competitive and that most employers face effective competition to attract and retain workers, combined with a view that some combination of unions, regulations, and lawsuits will help protect workers. That overall conclusion is probably true, but antitrust can still play a role in labor markets in two ways: by considering employer power in labor markets in selected mergers and by addressing anticompetitive agreements in labor markets.
%
%A merger that may substantially lessen competition among employers to hire workers is illegal under the Clayton Act. Marinescu and Hovenkamp (2018, 1) note that no merger has ever been blocked on these grounds and infer that “the antitrust law against anticompetitive mergers affecting employment markets is certainly underenforced, very likely by a significant amount.” Prager and Schmitt (2019) find that hospital mergers resulting in large increases in concentration in markets for skilled workers, including nurses and pharmacy workers, lead to lower wages. 
%The Horizontal Merger Guidelines (Department of Justice and Federal Trade Commission 2010, sec. 12) explain how the government evaluates mergers that may enhance buyer power. The government could define a relevant labor market and demonstrate that the merger in question would cause that market to become significantly more concentrated. The merging parties might then try to show that the affected workers have many alternative options for employment. For further details on this type of analysis, see Marinescu and Hovenkamp (2018) and Naidu, Posner, and Weyl (2018). 
%Two thorny issues are likely to arise if the government begins challenging mergers on the basis of harm to competition in labor markets. First, in cases where the merging parties assert that the merger will reduce their labor costs, the court may need to determine whether to credit these reduced costs as an efficiency gain or instead treat them as the exercise of buyer power in labor markets.11 Second, if a merger is expected to benefit consumers but harm workers, the court may need to determine whether and how to balance the interests of these two groups. Marinescu and Hovenkamp (2018) argue that under current law, a merger that harms workers by lessening competition in the labor market would not be saved by also offering benefits to consumers. 
%If the antitrust authorities seriously want to explore the possibility of challenging mergers on the basis of harm to competition in labor markets, developing a quick and efficient means of identifying mergers that involve a significant overlap in plausible labor markets would be a good first step.
%
%Section 1 of the Sherman Act prohibits agreements among employers to refrain from competing to hire workers, just as it prohibits traditional cartels among product-market rivals. This raises questions about no-poach and no-hire agreements that arise in certain labor markets. 
%In a prominent “no-poach” case, the Department of Justice (2010) sued Adobe, Apple, Google, Intel, Intuit, and Pixar for entering into agreements not to recruit certain workers from each other.12 When Apple CEO Steve Jobs learned that Google was trying to recruit employees from Apple’s Safari team, Jobs threatened Google co-founder Sergey Brin, stating that “if you hire a single one of these people, that means war.” In response, Google’s CEO Eric Schmidt stopped all efforts at Google to recruit anyone from Apple. When this was conveyed to Apple, Apple reciprocated (Koh 2014). Later, when a Google recruiter contacted an Apple employee, Jobs complained to Schmidt, who apologized and made a public example out of that recruiter, who was terminated within the hour.
%The Department of Justice and the Federal Trade Commission later released Antitrust Guidance for Human Resources Professionals, stating that “[g]oing forward, the DOJ intends to proceed criminally against naked wage-fixing or no-poaching agreements. These types of agreements eliminate competition in the same irredeemable way as agreements to fix product prices or allocate customers, which have traditionally been criminally investigated and prosecuted as hardcore cartel conduct” (Department of Justice and Federal Trade Commission 2016). Notice that this guidance refers to “naked wage-fixing or no-poaching agreements.” A no-poach agreement between two or more companies could be justified if those companies are engaged in legitimate joint activity, such as a joint venture to develop new products, and if the no-poach agreement is confined to employees involved in that joint activity, especially if the joint activity involves training these employees or providing them with access to confidential information. 
%No-hire agreements are common in the franchise sector. Krueger and Ashenfelter (2018) report that in 58 percent of major franchisors’ contracts with franchisees, including McDonald’s, Burger King, and Jiffy Lube, one franchisee is prohibited from hiring workers from another franchisee in the same chain. They find that no-hire agreements are more common in low-wage, high-turnover industries and have become more common over the past 20 years.
%Some limited no-hire provisions of this type could be justified if they provide an incentive for franchisees to invest in workers, giving them human capital that is specific to the franchisor but not to the franchisee. As a result, these agreements are more difficult to challenge under antitrust than are “naked” no-hire agreements. 
%Krueger and Posner (2018) describe a court case involving Jack-in-the-Box in which such a challenge failed. Under the Rule of Reason analysis typically used in antitrust to analyze agreements of this type, two important considerations will be how significantly these agreements restrict the number of employment options available to workers and whether they have depressed wages. A quick look may be sufficient to determine that a no-hire provision has no real efficiency justification and tends to suppress wages.

