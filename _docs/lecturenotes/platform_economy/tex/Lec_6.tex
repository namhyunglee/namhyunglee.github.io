\chapter{전자상거래: 아마존}\label{cha:ecommerce}

\section*{학습개요}
네트워크 이론과 양면 시장 이론, 데이터 기반 의사 결정 등 이전 강의에서 학습한 이론적 틀로 소매 플랫폼의 대표 기업인 아마존의 사업 방식을 파악한다. 

\section*{학습목표}
\begin{enumerate}
\item 소매 플랫폼 사업의 특징을 설명할 수 있다.
\item 소매업에서 데이터 서비스 사업의 특징을 설명할 수 있다.
\item 전자상거래의 미래를 생각해본다.
\end{enumerate}

\section*{주요 용어}
소매 플랫폼, 클릭 스트림, 데이터 서비스, 온라인 거래

\pagebreak

\section{전자상거래 산업}\label{sec:ecommerce}
\begin{itemize}
\item 모든 전자상거래 산업이 플랫폼은 아님. 대략의 규모를 가늠하는 차원에서
\item 2018년 전세계 전자상거래는 약 26조 달러 규모로 추정 \citep{UNCTAD:2021wk}
	\begin{itemize}
	\item B2B 시장이 21조 달러로 가장 큰 비중(83\%)을 차지
	\item B2C 시장은 4.4조 달러\footnote{동영상 강의에서의 설명을 반대로 잘못했고, B2B 시장의 비중이 높은 것이 맞음}
		\begin{itemize}
		\item 중국, 미국, 영국 순의 규모
		\end{itemize}
	\end{itemize}
\item 소수의 글로벌 기업이 특정 지역에서 강세를 보임
	\begin{itemize}
	\item 미국과 유럽: 아마존(Amazon)
	\item 중국: 알리바바(Alibaba), 제이디닷컴(JD.com), 핀두오두오(Pinduoduo)
	\item 동남아시아: 라자다(Lazada, 알리바바 소유), 쇼피(Shopee)
	\item 라틴 아메리카: 메르카도 리브레(MercadoLibre)
	\end{itemize}
\end{itemize}

\section{아마존의 경제학}\label{sec:amazon}
\subsection{아마존의 성장}
\begin{itemize}
\item 1994년 7월 5일, 제프 베조스(Jeff Bezos)\footnote{2021년 2월 CEO 사임을 발표하고, 2021년 3분기 중으로 앤디 재시(Andy Jassy)가 차기 CEO가 될 것을 공개}가 미국 워싱턴 주 벨레뷰(Bellevue, Washington)에서 창업 \citep{Stone:2013aa}
	\begin{itemize}
	\item 온라인으로 책을 판매하는 것에서 출발 $\rightarrow$ 가전제품, 소프트웨어, 비디오 게임, 의류, 가구 등으로 확장
	\end{itemize}
\item 1997년, 주식 시장 상장
\item 2017년, 홀 푸드 마켓(Whole Foods Market)을 인수
	\begin{itemize}
	\item 오프라인 시장으로의 적극적인 진출
	\end{itemize}
\item 직접 판매자이자 플랫폼으로 기능: 직접 판매 및 자체 브랜드 상품 판매 $+$ 외부 판매자 중개 
	\begin{itemize}
	\item 소비자 데이터 분석
	\item 단기 이윤보다 효율성과 고객 서비스를 우선시
	\item 장기 성장을 목표로
	\end{itemize}
\item 아마존 총 수입과 영업 이익: 표 \ref{tab:amazon}
	\begin{itemize}
	\item 장기 성장 $+$ 상당 기간 동안 적자를 기록 
		\begin{itemize}
		\item 북미 시장에서의 흑자는 최근의 일이며, 그외 전세계 시장에서는 아직 적자
		\item 재고 및 운송 등에서 규모의 경제가 나타나기까지 시간 소요 $+$ 초기 자본 투자 필요
		\item 소비자 수로 인한 네트워크 효과가 나타나기까지도 시간 소요
		\end{itemize}
	\item 현재에도 흑자는 아마존 웹 서비스에서 주로 기록
	\end{itemize}
			\begin{table}[htp]
			\begin{center}
			\begin{threeparttable}
			\caption{아마존의 경영성과, 1998--2020}\label{tab:amazon}
			\begin{tabularx}{\textwidth}{lrrrrrr}
			\toprule
			& 총 수입 & 북미 영업이익 & 전세계 영업이익 & AWS 영업이익 & 고용 \\
			\midrule
			1998 & 0.61 & -0.13 & & & \\
			1999 &1.64  & -0.72 & & & \\
			2000 & 2.76  & -1.42 & & &\\
			2001 & 3.13 & -0.55 & & &\\
			2002 & 3.94 & -0.15 & & &\\
			2003 & 5.26 & 0.04 & & &\\
			2004 & 6.92 & 0.59 & & &\\
			2005 & 8.94 & 0.33 & & &\\
			2006 & 10.72 & 0.19 & & &\\
			2007 & 14.84 & 0.48 & & & 17,000 \\
			2008 & 19.16 & 0.65 & & & 20,700\\
			2009 & 24.51 & 0.9 & & & 24,300\\
			2010 & 34.21 & 1.16 & & & 33,700\\
			2011 & 48.08 & 0.63 & & & 56,200\\
			2012 & 61.1 & -0.03 & & & 88,400\\
			2013 & 74.45 & 1.16 & 0.15  & 0.67 & 117,300\\
			2014 & 88.99 & 0.36 & -0.64 & 0.45 & 154,100\\
			2015 & 107.00 & 1.42 & -0.69 & 1.50 & 230,800\\
			2016 & 135.98 & 2.37 & -1.28 & 3.10 & 341,400\\
			2017 & 177.86 & 2.83 & -3.06 & 4.33 & 566,000\\
			2018 & 232.88 & 7.26 & -2.14 & 7.29 & 647,500\\
			2019 & 280.52 & 7.03 & -1.69 & 9.20 & 798,000\\
			2020\tnote{a} & 386.1 & 8.65 & 0.71 & 13.53 & 1,298,000\\
			\bottomrule
			\end{tabularx}
			\begin{tablenotes}
			\small
			\item 단위: 10억 달러, 명
			\item 출처: 아마존 10-K, \url{https://amazonir.gcs-web.com/quarterly-results}
			\item[a]  \url{https://press.aboutamazon.com/news-releases/news-release-details/amazoncom-announces-financial-results-and-ceo-transition}
			\end{tablenotes}
			\end{threeparttable}
			\end{center}
			\end{table}%
\end{itemize}

\subsection{소매 플랫폼에서 기술 플랫폼으로의 진화}
\subsubsection{소매 플랫폼}
\begin{itemize}
\item 소매 플랫폼: 아마존 마켓플레이스(Amazon Marketplace)
	\begin{itemize}
	\item 소비자 -- 아마존 마켓플레이스 -- 판매자
		\begin{itemize}
		\item 소비자: 선택의 폭이 넓어짐 $+$ 가격 하락
		\item 판매자: 소비자 접근의 시간 및 비용 절감
		\end{itemize}
	\item 소비자 $\rightarrow$ 아마존: 아마존 프라임(Amazon Prime: 배송비 무료 등) 등의 회원 가입 비용
	\item 판매자 $\rightarrow$ 아마존: 수수료 (판매액의 약 30\% \citep{Mims:2018wy})	
	\item 마켓플레이스의 아마존 내 소매 판매 비중 1999년 3\% $\rightarrow$ 2018년 58\% (Amazon, Annual Report 2018)
	\end{itemize}
\item 아마존의 역할
	\begin{itemize}
	\item 온라인 상의 상점 역할(storefront)
	\item 판매자의 재고 보관, 주문 및 배송 처리(Amazon Fulfilment)
		\begin{itemize}
		\item 하지만 도매상은 아님
		\end{itemize}
	\item 판매자에 대해 구독제 서비스만 운영
		\begin{itemize}
		\item 소량 판매: 개인 회원 $\rightarrow$ 낮은 구독비 $+$ 분류 제한 또는 등록시 마다 비용 부과
		\item 대량 판매: 기업 회원 $\rightarrow$ 월 40달러의 구독비 $+$ 분류 제한 없음
		\end{itemize}
	\end{itemize}
\item 판매자
	\begin{itemize}
	\item 판매 가격 설정, 소비자로부터의 수입, 아마존으로 수수료 지불
	\item 아마존은 대량 물류 계약(또는 직접 물류 서비스를 운영)하여 운송 비용을 낮추므로, 판매자가 직접 물류 서비스를 이용할 때 보다 낮은 가격으로 이용 가능
	\end{itemize}
\item 이외에도 음악, 영상(아마존 프라임 비디오, 아마존 뮤직) 등의 플랫폼 서비스 제공
\end{itemize}

\subsubsection{기술 플랫폼}
\begin{itemize}
\item 킨들(Kindle)
	\begin{itemize}
	\item 2007년 시작, 전자책 콘텐츠 $+$ 하드웨어
	\end{itemize}
\item 미케니컬 터크(Mechanical Turk)
	\begin{itemize}
	\item 2005년 시작, 기계가 하기는 복잡하지만 사람이 하기는 쉽고 단순한 업무를 제공
		\begin{itemize}
		\item[예)] 사진/동영상 처리, 데이터 확인, 정보 수집 및 처리
		\end{itemize}
	\item 요구하는 업무(HIT: Human Intelligence Task)를 제시, 아마존은 수수료 20\% 수입
	\end{itemize}
\item 에코/알렉사(Echo/Alexa)
	\begin{itemize}
	\item 음성 인식 $\rightarrow$ 개인 비서, 주문 처리 등
	\end{itemize}
\item 아마존 웹 서비스(AWS: Amazon Web Services)
	\begin{itemize}
	\item 클라우드 컴퓨팅 플랫폼
	\item 2002년 시작: 아마존이 연말의 일시적인 수요 폭증에 대응하기 위한 방법을 모색 중 탄생
	\item 2006년 3월 S3 저장, 8월 EC2 컴퓨팅 서비스 출시 
	\item 사용량에 따른 지불(pay-as-you-go) 방식
	\item 사용자는 필요에 따라 유연하게 IT 서비스를 증설 또는 축소 $\rightarrow$ 하드웨어 및 소프트웨어 구입 / 보안 / 관리 비용 절감
	\end{itemize}
\end{itemize}


\subsection{데이터! 데이터! 데이터!}
\begin{itemize}
\item 거래에서의 데이터 생성
	\begin{itemize}
	\item 상품 검색: 아마존 자체 연구에 따르면 미국 소비자의 55\%가 아마존에서 가장 먼저 상품 검색
	\item 뿐만 아니라 구매 시점, 구매 가격 등 구매 흐름의 관한 정보를 획득 $\rightarrow$ 클릭 스트림(click stream)
	\item 소비자의 구매 이력도 파악 
	\item 또 온라인에서의 구매를 결정하도록 하는 상품 정보가 무엇인지도 파악: 상품 크기, 사진 및 동영상, 사용자 리뷰/평가, 
	\end{itemize}
\item 판매자에게 데이터 서비스 제공
	\begin{itemize}
	\item 매출(전통적인 매장에서도 확인 가능) $+$ 검색 기록, 클릭 흐름, 구매 이력, 리뷰/평가, 위치 정보 등
	\item 프리미엄 데이터 서비스: 수요 및 트렌드 예측, 소비자의 가격 탄력성 등을 기반으로 데이터 기반 가격 전략 제공 $\rightarrow$ 기업 회원 상대 연간 10만 달러 부터 시작하는 것으로 알려짐 \citep{Bond:2018ud}
	\end{itemize}
\end{itemize}


\section{아마존과 전자상거래의 미래}\label{sec:aftercovid19}
\begin{itemize}
\item 아마존은 창립때부터 저비용의 유연한 자산 관리 구조를 갖는 것이 목표였음
	\begin{itemize}
	\item 아마존 프라임: 미국 아마존 가입자의 63\%가 이용. 연간 고정 회원비를 지불하고 2일 이내 배송 서비스를 받음
	\item 물류 비용 하락에 지속 투자
		\begin{itemize}
		\item 물류 센터(로봇)와 운송(드론)의 자동화
		\item 미국 인구의 44\%를 상대할 수 있을 것으로 예상되는 주요 거점 지역에 물류 센터 건설
		\item 오프라인 상점 진출로 최종 전달 체계의 접근성을 높임
		\end{itemize}
	\end{itemize}
\item 전자상거래 산업 전체로 보면, 코로나19로 인해 비대면 상황이 지속되면서 온라인 거래가 증가
	\begin{itemize}
	\item $\rightarrow$ 많은 연구는 소비자의 상당수가 코로나19 이후에도 온라인 구입을 지속할 것을 보임	
	\end{itemize}
\end{itemize}

\pagebreak

\section*{정리하기}
\begin{enumerate}
\item 전자상거래 중 B2B 시장의 비중이 높으며, 소수의 글로벌 기업이 특정 지역에서 강세를 보이는 현상이 관찰되고 있다.
\item 소매 플랫폼은 소비자와 판매자를 중개하는 역할을 한다.
\item 소매 플랫폼을 통해 소비자는 선택의 폭이 확대되고 가격 하락의 이점을 경험할 수 있으며, 판매자는 소비자 접근의 시간 및 비용을 절감할 수 있다.
\item 소매 플랫폼은 소비자로부터 회원 가입 비용을, 판매자로부터 수수료를 받을 수 있다.
\item 또한 소매 플랫폼은 거래에서 생성되는 소비자 위치, 상품 검색, 클릭 스트림, 구매 이력, 구매에 영향을 미치는 상품 정보 등의  데이터를 수집할 수 있다.
\item 소매 플랫폼은 이러한 데이터를 토대로 수요 및 트렌드 예측, 데이터 기반 가격 전략을 판매자에게 제공하고 이에 대한 수입을 얻을 수 있다.
\item 코로나19 감염증 확산 이후 전세계적으로 온라인 소매 판매가 크게 늘어났으며, 소비자의 온라인 구입은 지속될 가능성이 높다.
\end{enumerate}